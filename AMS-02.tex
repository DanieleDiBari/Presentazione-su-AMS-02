%\documentclass[notes]{beamer}       		% compila sia i frame che le note
\documentclass{beamer}              				% compila solo i frame
%\documentclass[notes=only]{beamer}  	% compila solo le note

% ita language and encoding
\usepackage[utf8]{inputenc}
%\usepackage[italian]{babel}

% formattazione
\usepackage{ragged2e} 							% pacchetto che contiene il comando \justifying
%\apptocmd{\frame}{}{\justifying}{}  	% --> applica la giustificazione a tutti i frame del documento
%\justifying											% --> applica la giustificazione a tutto il documento

% graphics style
\usepackage{graphicx}
\usepackage{xcolor}
\usepackage{shadowtext}

%image paths
\graphicspath{	
	{./Immagini/}
	{./Immagini/TRD/}
	{./Immagini/ToF/}
	{./Immagini/Magnetic_Spectrometer/}
	{./Immagini/Magnetic_Spectrometer/Magnet/}
	{./Immagini/Magnetic_Spectrometer/Tracker/}
	{./Immagini/Magnetic_Spectrometer/Tracker/Ladder_Fabrication/}
}

% impostazioni base delle slide - tema, colori, caratteri, ... (deic.uab.es/~iblanes/beamer_gallery/)
\mode<presentation>
{
	\usetheme{CambridgeUS}      % or try Darmstadt, Madrid, Warsaw, ...
  	\usecolortheme{rose} 			% or try albatross, beaver, crane, orchid, rose ...
	\usefonttheme{structurebold} % or try structureitalicserif, structuresmallcapsserif
}

% settagio dati principali del file - titolo, autore, ...
\title[Presentation on the AMS-02 experiment]{The AMS-02 experiment}
\subtitle{General review of the apparatus}
\author{Daniele Di Bari}
\institute{University of Perugia}
\date{27th April 2017}
\subject{Course of Particle Detectors}

% By default the beamer class adds navigation buttons in the bottom right corner. To remove them one can place
\beamertemplatenavigationsymbolsempty

% settaggio prima slide
\setbeamertemplate{title page}
{
	\shadowcolor{white!30!black}
	\vspace{-0.5cm}	
	
    \shadowtext{\textcolor{white}{\textbf{\footnotesize{Alpha Magnetic Spectrometer}}}}
   
    \vspace{0.2cm}  	
    
   	\shadowtext{\textcolor{white}{\textbf{\LARGE \inserttitle}}}\par
      
   	\vspace{0.1cm}  

   	\shadowtext{\textcolor{white}{\emph{\large \insertsubtitle}}}\par
   	
   	\vspace{2.5cm}
	
	\titlepagetext{lightgray}{The general particle physics}\\
	\titlepagetext{lightgray}{experiment in space, on board}\\
	\titlepagetext{lightgray}{the International Space Station}\\
	\titlepagetext{lightgray}{since 19th May 2011.}		
}
				
% colors
\definecolor{itemred}{RGB}{163,0,0}
\definecolor{itemblue}{RGB}{52,57,176}
\definecolor{notegrey}{RGB}{90,90,90}
\definecolor{titlebgd_grey}{RGB}{242,242,242}
\definecolor{titleframe_red}{RGB}{204,0,0}

% COMANDI
% Formato testo generico in titolpage
\newcommand\titlepagetext[2]{\textcolor{#1}{\footnotesize{\textit{{#2}}}}}
\newcommand\colortextbf[2]{\textcolor{#1}{\textbf{#2}}}
\newcommand\bluetextbf[1]{\textcolor{itemblue}{\textbf{#1}}}
\newcommand\bluetextit[1]{\textcolor{itemblue}{\textit{#1}}}
\newcommand\textitem[2]{\textcolor{itemblue}{\textbf{#1} (\textit{#2})}}

% Testo giustificato
\newenvironment<>{justify}{\justifying{}}


% 	MODIFICA DEL TEMA CambridgeUS -- HEAD AND FOOTLINES
%		Il tema CambridgeUS usa il outer-tema "infolines" outer theme che imposta head and footlines (vedi il file .sty: ../tex/latex/beamer/themes/theme/beamerthemeCambridgeUS.sty). 
%		Ad esempio, in "beamerouterthemeinfolines.sty" è definito il "footline template" in cui sono dichiarate tre caselle - la prima per l'autore, la seconda per il titolo e l'ultima per la data e il numero.
%		Utilizzo delle impostazioni di base di "infolines" e modificando diversi parametri per ottenere il layout il tema personalizzato
\setbeamercolor{institute in head/foot}{parent=palette secondary}
\setbeamercolor{subject in head/foot}{parent=palette tertiary}
\setbeamercolor{title in head/foot}{parent=palette primary}
\setbeamertemplate{footline}
{
	\leavevmode%
	\hbox{%
  		% "ht" è lo spazio sopra il carattere partendo dalla base del carattere, "dp"  è lo spazio sotto il carattere partendo sempre dalla base del carattere
  		\begin{beamercolorbox}[wd=.333333\paperwidth,ht=2.5ex,dp=1ex,left,leftskip=4ex]{author in head/foot}%
   			\usebeamerfont{author in head/foot}\insertshortauthor
  		\end{beamercolorbox}%
  		\begin{beamercolorbox}[wd=.333333\paperwidth,ht=2.5ex,dp=1ex,center]{institute in head/foot}%
    		\usebeamerfont{title in head/foot}\insertshortinstitute
 		 \end{beamercolorbox}%
 		 \begin{beamercolorbox}[wd=.333333\paperwidth,ht=2.5ex,dp=1ex,right,rightskip=4ex]{date in head/foot}%
 		 	\usebeamerfont{date in head/foot}\insertshortdate{}\hspace*{2em}\insertframenumber{} / \inserttotalframenumber
 		 \end{beamercolorbox}
 		 }%
 	\vskip0pt%
}
\makeatletter
\setbeamertemplate{headline}
{
  \leavevmode%
  \hbox{%
  \begin{beamercolorbox}[wd=.5\paperwidth,ht=2.8ex,dp=1.35ex,left]{subject in head/foot}	
    \usebeamerfont{section in head/foot}\hspace*{4ex}Course of Particle Detectors			% --> N.B. Qui c'è l'unica scritta che non usa nessun template tipo \title o \institute
  \end{beamercolorbox}%
  \begin{beamercolorbox}[wd=.5\paperwidth,ht=2.8ex,dp=1.35ex,right]{title in head/foot}%
    \usebeamerfont{subsection in head/foot}\insertshorttitle\hspace*{4ex}
  \end{beamercolorbox}}%
  \vskip0pt%
}
\makeatother
%\makeatletter
%\setbeamertemplate{headline}
%{
%  \leavevmode%
%  \hbox{%
%  \begin{beamercolorbox}[wd=.5\paperwidth,ht=2.65ex,dp=1.5ex,right]{subject in head/foot}%
%    \usebeamerfont{section in head/foot}\expandafter\beamer@ifempty\expandafter{\insertsectionhead}{Course of Particle Detectors}{\insertsectionhead}\hspace*{2ex}
%  \end{beamercolorbox}%
%  \begin{beamercolorbox}[wd=.5\paperwidth,ht=2.65ex,dp=1.5ex,left]{title in head/foot}%
%    \usebeamerfont{subsection in head/foot}\hspace*{2ex}\expandafter\beamer@ifempty\expandafter{\insertsubsectionhead}{\insertshorttitle}{\insertsubsectionhead}
%  \end{beamercolorbox}}%
%  \vskip0pt%
%}
%\makeatother
%% FINE MODIFICA

\begin{document}
	% impostare una foto come sfondo del frame: 
	\usebackgroundtemplate{\includegraphics[height=\paperheight,keepaspectratio]{on_ISS-08_cutMIDDLEUP-3088.jpg}}	
	% 	possibili impostazioni per l'img: width=\paperwidth  --  height=\paperheight  --  keepaspectratio

\begin{frame}\pdfbookmark[2]{Title}{Title}
	\titlepage
\end{frame}

\note{ 
\footnotesize{
The absorption thickness of the Earth's atmosphere, corresponding to an average of 25 radiation lengths (X0), screens the ground from primary CRs, which interact before reaching the detector. 

Direct detection of electromagnetically interacting CRs is therefore carried in space. In order to identify the nature of the detected particles, space borne instruments exploit typical high energy physics detection techniques, like um precision tracking using silicon detector technology or calorimetric energy measurements. 

Despite the detection concept is very similar to the modern accelerator experiments, the technological realization differs significantly. The requirements of a space borne experiment are in fact very challenging. Weight, dimension and power consumption constraints limit the size of the detector (thus their acceptance) to the $\sim$ 10 m$^3$ range. The limited bandwidth for the data transfer (to ground), the extreme thermal environment and the transport from ground to space also shape critically the detector concept.

In a typical direct CR detection experiment, particles traversing the instrument are fully characterized via the simultaneous measurement of the energy E, mass M, charge Z and charge sign. In a minimalistic experiment, a calorimeter is used to measure the energy E, dE=dX detectors are used to measure Z and a time of flight system is used to trigger the data acquisition and to measure the velocity and hence the mass M. A magnet can be used to detect the particle trajectory and infer the charge sign.

Some experiments are also equipped with additional detectors dedicated to the identification of CR rare species, like transition radiation detectors or neutron detectors that improve the identification of $e^{\pm}$. The measurement of CRs in space began in the 1970s with the measurement of nuclear isotopes in the energy range $\leqslant$ 1 GeV with the IMP satellites. The field flourished in 2000s, when the FERMI-LAT satellite observatory provided for the first time high precision rays direct measurements and the PAMELA satellite mission (see Figure 1.8) provided the direct measurement of charged CRs up to the 100 GeV range. The state-of-the-art space born experiment is the AMS-02 detector. AMS is the first particle detector located on the International Space Station (ISS) where it has been collecting high precision data up to the TeV range since 2011. The AMS detector will be exhaustively described in Chapter 2.

AMS successfully confirmed the possibility of operating a particle physics detector on the ISS, hence stimulating the development of the next (and ``next-to-next'') generation experiments to be operated on space stations, like the JEM-EUSO observatory or the DAMPE/HERD instrument.
}
}

\usebackgroundtemplate{}

\begin{frame}[allowframebreaks]\pdfbookmark[2]{Contents}{Contents}
\frametitle{Table of Contents}
\tableofcontents
\end{frame}

\begin{frame}<presentation:0>[noframenumbering]
	\frametitle{Technical Question about the Detector}
	AMS is a particle physics experiment in space, so it focuses on the detection of particles, but:
	\begin{itemize}
	\item	Which kind of particle do we have to detect?
	\item	What is the required dimension of the detector?
	\item	Which ``property'' of the particle do we have to know?
	\begin{itemize}
		\item[$\circ$]	Position - Trajectory
		\item[$\circ$]	Time
		\item[$\circ$]	Number
		\item[$\circ$]	Energy
		\item[$\circ$]	Momentum
	\end{itemize}
	\item	What is the required resolution?
	\item	What is the maximum count rate?
	\item	What is the time distribution of the events?
	\item	And last, but not least, how much does it cost? 
	\end{itemize}
\end{frame}

\section{Introduction}
\begin{frame}
	\setbeamertemplate{background}{white}
	\frametitle{Introduction}
	\framesubtitle{The Alpha Magnetic Spectrometer}
	\justifying
	AMS-02 is a general purpose \textbf{high energy particle detector}  designed to operate as an external module on the
	International Space Station.
	
	\vspace{0.25cm}
	\begin{block}{Purpose of the AMS experiment}
		\justifying
		To perform accurate, high statistics, long duration measurements of the spectra of energetic primary charged 
		cosmic rays (CRs) in space. 
	\end{block}
		
	\vspace{0.25cm}
	Some of the \textbf{physics goals} are:
	\begin{enumerate}
		\item \bluetextbf{Dark Matter}
		\item \bluetextbf{Matter/Antimatter Asymmetry}
		\item \bluetextbf{Cosmic Ray Physics}
	\end{enumerate}
\end{frame}
\note{
	AMS-02 (Alpha Magnetic Spectrometer ) is a general purpose high energy particle detector which was successfully deployed on 
	the International Space Station (ISS) on May 19, 2011 to conduct a unique long duration mission of fundamental physics 
	research in space. Among the physics objectives of AMS are the searches for an understanding of Dark Matter, Anti-matter, the 
	origin of cosmic rays and the exploration of new physics phenomena not possible to study with ground based experiments
	Lo scopo dell'esperimento è quello di effettuare, per un periodo di lunga durata così da ottenere un'elevata statistica, delle 
	misure molto accurate dello spettro di energia (per energie fino all'ordine dei TeV) dei raggi cosmici carichi primari direttamente 
	nello spazio.
}

\section{Technical Requirements}
\subsection{from Physics Goals}
\begin{frame}
    \frametitle{Technical Requirements}
    \framesubtitle{from Physics Goals}
    \justifying
    
	Physical goals involve \textbf{technical requirements for the AMS detector}:
	
	\begin{itemize}\justifying
		\item	\bluetextbf{Dark Matter $\Rightarrow$}
					to get $e^+/p$ rejection of $\sim10^{-6}$ for the measurement of the positron fraction.
		\item	\bluetextbf{Matter/Antimatter Asymmetry $\Rightarrow$}
					to reach a sensitivity in the search for anti-matter nuclei of $10^{-10}$ (ratio of anti-helium nuclei to helium nuclei).
		\item	\bluetextbf{Cosmic Ray Physics $\Rightarrow$}
					to measure the composition and spectra of charged particles with an accuracy of 1\%.
	\end{itemize}
	
	Moreover, \bluetextbf{for each of these ones}, it is very important to extend, as far as possible, the energy range of the 
	measurements.
	
	For AMS, the required energy range is from 0.5 to $\sim$ 2000 $GeV$.

	%\footnotesize{\underline{\textbf{Note}} These requirements represent a considerable sensitivity improvement compared to the previous space-borne experiments.}

\end{frame}

\note{
	Esiste un forte interesse nell'effettuare delle misure di precisione della frazione di positroni dei raggi cosmici nella regione 
	energetica da 10 a 1000 GeV, in quanto le misurazioni di $e^+ / (e^+ + e^-)$, da AMS-01, HEAT, PAMELA e FERMI indicano 
	una grande deviazione di questo rapporto dalla produzione di e+ e e- previsto da un modello che comprende solo collisioni 
	ordinarie del raggio cosmico.
   	 
   	These available measurements are both at too low an energy and of too limited statistics to shed the light on the origin of this 
   	significant excess.
   	 
	AMS-02 is expected to provide definitive answers concerning the nature of this deviation.
}

\begin{frame}
	\frametitle{Technical Requirements}
    \framesubtitle{from Physics Goals}	
    
    \begin{block}{The technical challenge of AMS-02}
		\justifying
		The illustrated requirements represent a considerable sensitivity improvement compared to the previous space-borne 
		experiments.
	\end{block}
   	
   	\vspace{0.25cm}
	\small{\bluetextbf{NOTE}}
	\vspace{-0.2cm}
    \begin{itemize}
    	\item[\footnotesize{$\blacktriangleright$}] 
    		\footnotesize{\justifying
    		There is a strong demand for precision measurements of CRs in the energy region from 10 to 1000 
    		$GeV$ as the measurements of positron fraction, i.e.  $e^+ / (e^+ + e^-)$, by AMS-01, HEAT, PAMELA and FERMI 
    		indicate a large deviation of this ratio from the production of $e^+$ and $e^-$ predicted by a model that includes only 
    		ordinary CR collisions.
    		
    		These available measurements are both at too low an energy and of too limited statistics to shed the light on the origin of 
    		this significant excess.
    		
    		AMS-02 is expected to provide definitive answers concerning the nature of this deviation.}
   	\end{itemize}  
   	 
%
%%	\vspace{0.25cm}
%	There is a strong demand for precision measurements of cosmic rays in the energy region from 10 to 1000 GeV as the 
%	measurements of e+/(e+ + e-) by AMS-01, HEAT, PAMELA and FERMI indicate a large deviation of this ratio from the 
%	production of e+ and e- predicted by a model that includes only ordinary cosmic ray collisions. 
%	
%	These available measurements are both at too low an energy and of too limited statistics to shed the light on the origin of this 
%	significant excess. 
%	
%	AMS-02 is	expected to provide definitive answers concerning the nature of this deviation.
\end{frame}

%\begin{frame}
%	\frametitle{Technical Requirements}
%    \framesubtitle{From Physics Goals}
%	\begin{columns}
%		\begin{column}{0.5\textwidth}
%			\begin{center}
%				\includegraphics[width=0.95\columnwidth]{Schem_Requirements-DM.png}
%			\end{center}
%		\end{column}
%
%		\begin{column}{0.5\textwidth}
%			\begin{center}
%				\includegraphics[width=0.95\columnwidth]{Schem_Requirements-Antimatter.png}
%			\end{center}
%		\end{column}
%	\end{columns}
%\end{frame}

\begin{frame}[label=AMS-apparatus_img]
	\frametitle{Technical Requirement}
	\framesubtitle{general structure of the AMS apparatus}
	\begin{center}
		\vspace{-0.15cm}
		\hspace{-0.35cm}\includegraphics[height=0.7\paperheight]{Schem_structure_of_AMS.png}

		\vspace{-0.35cm}

		\hspace{0.25cm}\hyperlink{AMS-apparatus_list}{\textbf{\beamerbutton{link to AMS apparatus list}}}
	\end{center}
	%\vspace{-0.25cm}
\end{frame}

\begin{frame}
	\frametitle{Technical Requirement}
	\framesubtitle{from the research on the Dark Matter}
	\begin{center}
		\vspace{-0.1cm}
		\includegraphics[height=0.72\paperheight]{Schem_Requirements-DM.png}
	\end{center}
\end{frame}

\begin{frame}
	\frametitle{Technical Requirement}
	\framesubtitle{from the research on the Matter/Antimatter Asymmetry}
	\begin{center}
		\vspace{-0.1cm}
		\includegraphics[height=0.65\paperheight]{Schem_Requirements-Antimatter.png}
	\end{center}
\end{frame}

\begin{frame}
	\frametitle{Technical Requirement}
	\framesubtitle{from the research on the Cosmic Ray}
	\begin{center}
		\vspace{-0.5cm}
		\includegraphics[height=0.55\paperheight]{Schem_Requirements-Charge02.png}
	\end{center}
\end{frame}

%\begin{frame}
%	\frametitle{Technical Requirement}
%	\framesubtitle{From the research on the Cosmic Ray}
%	\begin{center}
%		\includegraphics[height=0.5\paperheight]{Schem_Requirements-Charge.png}
%	\end{center}
%\end{frame}

\subsection{a space-borne experiment}
\begin{frame}
	\frametitle{Technical Requirement} 
	\framesubtitle{why a particle detector in the space?}
	\justifying
	
	The absorption thickness of the Earth's atmosphere, corresponding to an average of 25 radiation lengths ($X_0$), screens the 
	ground from primary CRs, which interact before reaching the detector. 
	
	\vspace{0.25cm}
	\textbf{Direct detection of electromagnetically interacting CRs is therefore carried in space.} 
	
	\vspace{0.25cm}
	In order to identify the nature of the detected particles, space borne instruments exploit typical high energy physics detection 	
	techniques, like $\mu m$ precision tracking using silicon detector technology or calorimetric energy measurements. 

\end{frame}

\begin{frame}
	\frametitle{Technical Requirement} 
	\framesubtitle{of a space-borne experiment}
	\justifying
	
	\small{Despite the detection concept is very similar to the modern accelerator experiments, the technological realization differs 
	significantly.}
	
	\vspace{0.15cm}
	\small{\textbf{The requirements of a space borne experiment are in fact very challenging.}}
	
	\vspace{0.1cm}
	\begin{block}{The challenging requirements of a space born experiment}
		\justifying
		\small{
		Weight, dimension and power consumption constraints limit the size of the detector (thus their acceptance) to the 
		$\sim 10 m^3$ range.
		The limited bandwidth for the data transfer (to ground), the extreme thermal environment and the transport from ground to 
		space also shape critically the detector concept.
		
		Moreover, there is practically no way to fix any kind of problem with detectors and subsystem of AMS-02 once that it 
		is on the ISS\footnotemark[1].
%		Moreover, there is practically no way to fix any problem with AMS-02 detectors and subsystems when it's on the ISS
%		\footnotemark[1].
		}
	\end{block}
	
%	\footnotetext[1]{It is necessary a high degree of redundancy for the detector design to ensure that the functioning of the 
%	experiment remain on the required parameters for all the life time of the ISS (at least 20 years).}
	\footnotetext[1]{An high degree of redundancy in detector design is required to ensure that the operation of the experiment 
	remains within the required physical parameter for the entire lifetime of the ISS (at least 20 years)}
\end{frame}

%\begin{frame}
%	\frametitle{Technical Requirement} 
%	\framesubtitle{of a space-borne experiment}
%	New environment for HEP experiments
%	\begin{itemize}
%		\item	Operation in vacuum
%		\item	Acceleration during start and landing up to 9g
%		\item 	Temperature variations from $-180 \: ^\circ{C}$ to $+50 \: ^\circ{C}$
%		\item	Deposition limits on ISS $< 10^{-14} \frac{g}{s \cdot cm^2}$
%		\item	Weight limited to 13500 lb
%		\item	Datarate $1 Mbyte/s$ via 1 datalink
%		\item	Power consumption limited to 2kW
%		\item	Powersupply at 120 V via 1 powercable
%		\item	Redundancy
%	\end{itemize}
%\end{frame}

\section{The protype (AMS-01)}
\begin{frame}
	\frametitle{AMS-01} 
	\framesubtitle{The protype of AMS-02}
	\justifying	
	
	%	In order to ensure that technologies used in the detector construction work reliably in space, a scaled down detector 
	%	(AMS-01) was built and flown in 1998 onboard the STS-91 mission for 10 days.	
	Before installation on the Space Station was built a \textbf{scaled down detector of AMS} that,  in 1998, perform an 
	engineering flight of 10 days on the Space Shuttle Discovery (STS-91 flight) with the purpose of ensure that:
	
	\begin{itemize}
		\justifying	
		\item	The AMS experiment can function properly in space; in vacuum with orbital temperature changes from -65 to 40 
					$^{\circ}{C}$ and in the intense radiation background (which contains heavy nuclei causing single event latch-up in 
					chips).
		\item	The detector can withstand the tremendous vibrations (150 dB) and acceleration (3 g) at launch and the 
					deacceleration (6.5 g) at landing;
	\end{itemize}

	This mission was subsequently referred to as AMS-01.
\end{frame}

\begin{frame}{The Numbers of AMS-02}
	\begin{itemize}
	\small{
	\item[$\circ$]	\textbf{Weight}: 8500 kg
	\item[$\circ$]	\textbf{Volume}: 64 cubic meters
	\item[$\circ$]	\textbf{Power}: 2500 watts
	\item[$\circ$]	\textbf{Data downlink}:  9 Mbps (average)
	\item[$\circ$]	\textbf{Magnetic field intensity}: 0.15 Tesla (4.000 times stronger than the Earth magnetic field)
	\item[$\circ$]	\textbf{Magnetic material}: 1200 kg of Neodymium alloy (Nd$_2$Fe$_{14}$B)
	\item[$\circ$]	\textbf{Subsystems}: 15 among particle detectors and supporting subsystems
	\item[$\circ$]	\textbf{Launch}: 16th May 2011, 08:56 am EDT
	\item[$\circ$]	\textbf{Mission duration}: through the lifetime of the ISS, until 2024 or longer (it will not return back to Earth)
	\item[$\circ$]	\textbf{Construction}: 1999-2010
	\item[$\circ$]	\textbf{Cost}: \$ 1.5 billion (estimated)
	}
	\end{itemize}
\end{frame}

\section{Subdetectors of AMS-02}
\begin{frame}[label=AMS-apparatus_list]
	\frametitle{The AMS-02's subdetectors}
	\framesubtitle{from the top to the bottom}	
	\begin{itemize}
		\justifying
		\item	\hyperlink{TRD-Det}{\textitem{TRD}{Trasition Radiation Detector}}, 
					identifies electrons and positrons among other CRs.
		\item	\hyperlink{ToF-Det}{\textitem{ToF}{Time of Flight}}, 
					warns the sub-detectors of incoming CRs.
		\item	\hyperlink{SM-Det}{\bluetextbf{Magnetic Spectrometer}}, 
					detects the particle charge sign, separating matter from antimatter. Main components:
								\begin{itemize}
									\item[$\circ$]	\hyperlink{MAG-Det}{\textcolor{itemblue}{Magnet}}, 
															bends in opposite directions charged particles/antiparticles.
									\item[$\circ$]	\hyperlink{ST-Det}{\textcolor{itemblue}{Silicon Tracker}}, 
															direct measurement of the trajectory deflection.
								\end{itemize}
		\item	\hyperlink{TRD-Det}{\textitem{RICH}{Ring-Imaging CHerenkov detector}}, 
					measures with high precision the velocity of CRs.
		\item	\hyperlink{TRD-Det}{\textitem{ECAL}{Electromagnetic CALorimeter}}, 
					measures energy of incoming electrons, positrons and $\gamma$-rays.
	\end{itemize}
	\hfill \hyperlink{AMS-apparatus_img}{\textbf{\beamerbutton{link to AMS apparatus scheme}}}
\end{frame}

\subsection{TRD}
\begin{frame}[label=TRD-Det]
	\frametitle{TRD - Transition Radiation Detector}%{Particle ID \& 3D tracking}
	\vspace{-0.55cm}
	\begin{center}
		\includegraphics[height=0.35\paperheight]{TRD-Photo-05.jpeg}
	\end{center}
	\vspace{-0.45cm}
	\begin{block}{\small{The main features of TRD are:}}
		%\vspace{-0.25cm}
		\begin{itemize}\small{
			\item[$\circ$] \textbf{Discriminate between  $e^+$ and $p$}, whit a proton rejection of $10^{-2} \div 10^{-3}$ 
						in the energy range between $10$ and $300 GeV$.$^1$
						
			\item[$\circ$] \textbf{Identify light nuclei}, by measurement of the charge $|Z|$.
		}\end{itemize}
%		\vspace{-0.55cm}
%		\begin{center}
%			\includegraphics[height=0.35\paperheight]{TRD-Photo-05.jpeg}
%		\end{center}
	\end{block}
	\footnotetext[1]{\justifying
						This seems that, even if $e^+$ and $p$ 
						have the same charge, and so they are somewhat similar (i.e. in the other sub-detector, apart for the ECAL, 
						$e^+$ and $p$ with different energy give the same signals and therefore they may be confused), TRD may
						wrongly interpreter a $p$ as an $e^+$ only every $100 \div 1000$ real $p$ events.}
\end{frame}

\begin{frame}{TRD - Transition Radiation Detector}{the exploited physical process}
	\begin{columns}%[T]
	
		\begin{column}{0.5\textwidth}
			\justifying
			\small{The operating principle of TRD is mainly based on the known physical phenomenon, predicted by Ginzburg and 
			Frank in 1946 and observed, for the first time, by Goldsmith and Jelley in 1959, named:}
			
			\vspace{0.25cm}
			\begin{block}{Transition Radiation (TR)}
				\justifying
				TR is the electromagnetic radiation that is emitted whenever a charged particle transverse the interface 
				between two media with different refractive indexes.
			\end{block}
		\end{column}
		
		\begin{column}{0.5\textwidth}
			\begin{center}
%				\vspace{-0.75cm}				%		se l'allignamento delle colonne è impostato su [T]
				\vspace{-0.4cm} 				%		se l'allignamento delle colonne non è impostato su [T]
				\includegraphics[width=0.9\columnwidth]{TransitionRadiation-ColorVers02mid.jpg}
			\end{center}
		\end{column}
		
	\end{columns}
\end{frame}

\begin{frame}[label=TRD-main]
	\frametitle{TRD - Transition Radiation Detector}
	\framesubtitle{the main features of transition radiation process}

	\small{
	Typically, this emission phenomena \textbf{is relevant for ultra-relativistic particles} ($\gamma \geqslant 300$) 
	and the emitted photon are \textbf{soft X-rays}, i.e. TR-photons have an energy of the order of a few $KeV$.
	
		\begin{itemize}
			\justifying
			\item  The \textbf{emission spectrum of  TR depends on} the density of the traversed material $\rho$, on the particle's energy $E$ and on the modulus of its electric charge $|Z|$.
%		ALTERNATIVA alla FRASE PRECEDENTE
%			\vspace{-0.1cm}
%			\begin{itemize} 
%				\item[$\circ$] density of the traversed material, $\rho$;
%				\item[$\circ$] energy of the particle, $E$;
%				\item[$\circ$] modulus of the electric charge of the particle, $|Z|$.
%			\end{itemize}					 % the density of the traversed material, on the particle's energy and on the modulus of its electric charge.
			\item \textbf{Total energy emitted per interface crossed, $W_{TR}$, is proportional to $\gamma$}.
			\item \textbf{Number of emitted TR-photons per interface crossed is small}, 
					 e.g. for a unit charge $N_{TR} \sim (3 \cdot 137)^{-1}$
			\item Exist a minimum distance $D_f$, called \textbf{formation zone}, that the particle has to traverse in order to efficiently emit TR photons.
			\item Due to the very small emission angle, $\theta_{TR} \lesssim \gamma^{-1}$, \textbf{TR-photons stay close to particle's track}.
		
		\end{itemize}
	}
%
%	\vspace{-0.35cm}
%	\hfill \hyperlink{TR-details}{\textbf{\beamerbutton{click here for more details on TR}}}
\end{frame}

\begin{frame}[label=TRD-Design]
	\frametitle{TRD - Transition Radiation Detector}
	\framesubtitle{based on a well-proven design}
	\small{
		\begin{columns}
			\begin{column}{0.675\textwidth}
				\justifying
				The \textbf{AMS-02  TRD is based on a well-proven design}\footnotemark[1] with multiple irregular boundary 
				crossings in a 20-mm fleece radiator and straw tube proportional wire chambers filled with 
				gas mixture ($80 \% \: \text{Xe} - 20\% \: \text{CO}_2$) to detect the TR-photons. 
			
				\vspace{0.15cm}
				The TRD consists of 20 layers of fleece and straw tube modules - in each layers there are several:
			\end{column}
			\begin{column}{0.325\textwidth}
				\vspace{-4ex}
				\begin{center}
					\includegraphics[width=\columnwidth]{TRD-Stack.png}
				
					\hspace{0.4cm}\hyperlink{TRD-Module}{\textbf{\beamerbutton{photo of a TRD module}}}
				\end{center}
			\end{column}
		\end{columns}	
	
		%\vspace{0.15cm}
		\begin{itemize}
			\justifying
			\item \bluetextbf{TRD radiator}\\
				The fleece material LRP 375 BK has a density of $0.06 g/cm^3$ and polypropylene/polyethylene 
				fibers with a thickness of $10 \mu m$.
			\item \bluetextbf{TRD straw modules}\\
				The TR-photons are detected in proportional mode modules, each consisting of 16 straw tubes.
		\end{itemize}
		\footnotetext[1]{This coming from the R\&D work for ground based experiments - ATLAS and HERA}
	}
\end{frame}


\begin{frame}[label=TRD-Straws]
	\frametitle{TRD - Transition Radiation Detector}
	\framesubtitle{more about the straw tubes of a TRD straw module}
	\justifying
	The straw tubes have a diameter of 6 mm and different lengths, up to 2m. Their walls, made of 72 $\mu m$ a 
	double-layer kapton–aluminum foil, work as cathodes. In the center of each tube, there is a 30 $\mu m$ thick fine 
	gold plated wire (tensioned with 1N) that works as anode for the proportional chamber.

	\vspace{-0.15cm}
	\begin{columns}
		\begin{column}{0.125\textwidth}\end{column}
		\begin{column}{0.75\textwidth}
			\begin{figure}
				\centering
				\includegraphics[width=\textwidth]{TRD-StrawMod-01.png}
				\vspace{0.1cm}
				\includegraphics[width=0.75\textwidth]{TRD-StrawMod-02.png}
				\footnotesize{\caption{\justifying An example of a TRD straw module (up) and a schematic view of its section (down),
				 where also are drawn the longitudinal and vertical carbon fiber stiffeners used for the mechanical stabilization of the
				 module.}}
			\end{figure}
		\end{column}
		\begin{column}{0.125\textwidth}\end{column}
	\end{columns}
\end{frame}


\begin{frame}
	\frametitle{TRD - Transition Radiation Detector}
	\framesubtitle{the most critical design issue}
	\justifying
	\footnotesize{The most critical design issue is \textbf{the gas tightness of the straw modules}. In the vacuum of space,
	\textit{gas continuously diffuses out of the straw tubes}.  In order to operate the detector at stable parameters and to 
	compensate the gas gain change due to gas diffusion, a daily high voltage adjustment is performed, and the gas is refilled every 
	month by a gas supplier system.}

	\vspace{-0.65cm}
	\begin{columns}
		\begin{column}{0.1\textwidth}\end{column}
		\begin{column}{0.8\textwidth}
			\begin{figure}
				\centering				
				\includegraphics[height=0.35\paperheight]{TRD-GasSystem-05.png}
				\hfill
				\includegraphics[height=0.35\paperheight]{TRD-GasSystem-01.png}
			\end{figure}
		\end{column}
		\begin{column}{0.1\textwidth}\end{column}
	\end{columns}	
	
	\vspace{0.15cm}
	\begin{columns}
		\begin{column}{0.05\textwidth}\end{column}
		\begin{column}{0.9\textwidth}
			\justifying
			\scriptsize{\textitem{Figure}{TRD gas system}: The gas in the supply boxes is first 
			mixed (BOX-S), visible in the photo. A pump (BOX-C) helps the circulation of the gas through 41 gas circuits, feeding the
			whole TRD detection volume. Each gas circuit is composed of eight straw tubes connected in series. Ten separate 
			manifolds with a shut-off valve control a variable number of gas circuits. Single gas groups can be isolated in case of a gas 
			leak in a tube.}
		\end{column}
		\begin{column}{0.05\textwidth}\end{column}
	\end{columns}
\end{frame}


\begin{frame}
	\frametitle{TRD - Transition Radiation Detector}
	\framesubtitle{gas system location}
	\justifying

	\begin{figure}
		\centering
		\includegraphics[height=0.45\paperheight]{TRD-GasSystem-06.png}
			\vspace{0.075cm}
			
			\footnotesize{\bluetextbf{Figure:} Location of TRD Gas system elements}
	\end{figure}

	\vspace{-0.25cm}
	\small{The available supplies of gas, $49.5 kg$ of Xe and $4.5 kg$ of $\text{CO}_2$, will have to last for three years of 
	operation. 
	Using as standard conditions $1 \, bar$ and 298 $^{\circ}{K}$, this corresponds to $8420 \, l$ of Xe and $2530 \, l$ 
	of $\text{CO}_2$.}
\end{frame}


\begin{frame}
	\frametitle{TRD - Transition Radiation Detector}
	\framesubtitle{safety factor}
	\justifying
	\footnotesize{Since $\text{CO}_2$ molecules are smaller than Xe molecules, they are the component leaking the most.}					
	\vspace{-0.25cm}
	\begin{columns}
		\begin{column}{0.004\textwidth} \end{column}
		\begin{column}{0.575\textwidth} 
		\justifying		
		\footnotesize{The total TRD $\text{CO}_2$ leak rate of $1.5 \cdot 10^{-2} l \, mbar/s$ would correspond 
		(at standard conditions) to a loss of $\text{CO}_2$ over 3 years of $287 \, l$ or a 
		``\textbf{safety factor}''\footnotemark[1] of 8.8 with respect to the $\text{CO}_2$ supply.			
		
		\textit{Fabricated TRD modules are accepted if they have with a $\text{CO}_2$ safety factor better than 4.} 
		This can only be assured by testing each of the 5248 straws individually before producing a module.
		
		As shown in the histogram, \textit{the safety factor averaged over all modules selected for flight is $7.9$}.
		\textbf{This allow up to 20 years of operation on the ISS}.}
		\end{column}
		\begin{column}{0.417\textwidth}
			\begin{figure}
				\centering
				\includegraphics[width=0.975\columnwidth]{TRD-GasSystem-07.png}
				%\footnotesize{Straw module production: safety factor CO2}
			\end{figure}
		\end{column}
		\begin{column}{0.004\textwidth} \end{column}
	\end{columns}

	\footnotetext[1]{\justifying\scriptsize{Safety Factor (SF) describe the load carrying capacity of a system beyond the 
	expected or actual loads. In this case, it is the ratio between the volume of $\text{CO}_2$ stored, $2530 \, l$, and that lost 
	by diffusion in three years, $287 \, l$. Therefore, $\text{SF} = 2530 / 287 \simeq 8.8$.}}
\end{frame}


\begin{frame}[label=TRD-Supp_Struct]
	\frametitle{TRD - Transition Radiation Detector}
	\framesubtitle{the support structure of the TRD}%\hspace{2.2cm}\hyperlink{TRD-Mech_Struct}{\beamerbutton{more about 
%	the Mechanical Structure of the TRD}}}
	\justifying
	\small{In total, 328 modules (overall, 5248 straws) arranged on 20 layers that are supported by a conically shaped octagon
	structure made of aluminum honeycomb walls with carbon-fiber skins and bulkheads.}
	
	\vspace{-0.25cm}
	\begin{columns}
		\begin{column}{0.24\textwidth}\end{column}
		\begin{column}{0.37\textwidth}
			\begin{figure}
				\centering
				\includegraphics[width=\textwidth]{TRD-Support-05.png}
				
				%\footnotesize{\caption{\justifying AAA}}
			\end{figure}
		\end{column}
		\begin{column}{0.3\textwidth}\justifying
			
			\vspace{0.5cm}
			\footnotesize{\bluetextbf{Figure:}
			
			TRD support structure. 
			
			The \textbf{octagonal pyramid shape} has been chosen to optimize the TRD acceptance and to minimize the weight
			and the size of the apparatus.}
		\end{column}
		\begin{column}{0.09\textwidth}\end{column}
	\end{columns}
	
	\vspace{0.25cm}
	\small{The structure has a mechanical precision of 100 $\mu m$ to avoid wire displacement or straw walls deformation. 

	The upper and lower four layers run parallel to the magnetic field, the others perpendicular to \textbf{provide 3D 
	tracking}.}
%The TRD will be fully covered in a multi-layer-insulation (MLI) foil to keep the spatial orbit temperature gradient below 1K. 

	\vspace{-0.25cm}\hfill\hyperlink{TRD-Mech_Struct}{\beamerbutton{more on the TRD's Mechanical Structure}}
\end{frame}


\begin{frame}{TRD - Transition Radiation Detector}{the challenge for the AMS-02 experiment}
	\justifying
	\small{The challenge is to \textbf{build such a detector in a space-qualified way with strict limits} on outgassing, gas tightness, 
	weight and power consumption (less than $185 W$) whilst assuring safety and gas gain homogeneity in a harsh environment 
	during payload lift and in orbit without the possibility of further access to the experiment.}
	
	\vspace{-0.5cm}
	\begin{columns}
		\begin{column}{0.125\textwidth}\end{column}
		\begin{column}{0.65\textwidth}
			\begin{figure}
				\centering
				\includegraphics[width=0.75\textwidth]{TRD-Weights.png}
				
				\scriptsize{The optimized AMS-02 TRD design weighs less than $500 kg$.}
			\end{figure}
		\end{column}
		\begin{column}{0.125\textwidth}\end{column}
	\end{columns}
	
	\vspace{0.25cm}
	\small{This \textbf{involves} detailed \textit{finite-element calculations} as well as \textit{subcomponent vibration and 
	thermo-vacuum-cycle tests}.}
	
%	\footnotetext[1]{AMS-02 has a limited power consumption budget of $2000 W$, from which the TRD takes a part of less
%	than $185 W$.}	
\end{frame}

\begin{frame}[label=TRD-SQ_StrawsMod]
	\frametitle{TRD - Transition Radiation Detector}
	\framesubtitle{space qualification of the straw modules}
	\justifying	
	\small{Space qualification tests have been carried out for eight 0.7m long straw modules.}
	
	\vspace{-0.25cm}
	\begin{columns}
		\begin{column}{0.2\textwidth}\end{column}
		\begin{column}{0.45\textwidth}
			\begin{figure}
				\centering
				\includegraphics[width=0.95\textwidth]{TRD-StrawsQual-01.png}
				
				%\footnotesize{\caption{\justifying AAA}}
			\end{figure}
		\end{column}
		\begin{column}{0.27\textwidth}\justifying
			
			\vspace{0.5cm}
			\footnotesize{\bluetextbf{Figure}
			
			TRD thermal vacuum test of 4 straw detector modules}
		\end{column}
		\begin{column}{0.08\textwidth}\end{column}
	\end{columns}
	
	\vspace{0.25cm}
	\small{They underwent vibration tests (0.5 g sine sweep followed by 6.8 g random test followed by 0.5 g sine sweep) as
	well as thermo-vacuum-cycle tests (from -40 and +60 $^\circ{C}$) followed again by vibration tests. \textbf{No significant
	changes in eigenfrequencies, gas tightness or gas gain were observed.}}
	
	\hfill\hyperlink{TRD-SpaceQual}{\textbf{\beamerbutton{more details on Space Qualification}}}
\end{frame}

\begin{frame}
	\frametitle{TRD - Transition Radiation Detector}
	\framesubtitle{structural verification of the TRD support}
	\vspace{-1.15cm}
	\begin{columns}[t]
		\begin{column}{0.015\textwidth}\end{column}
		\begin{column}{0.425\textwidth}
			\begin{figure}
				\centering
				\small{\bluetextbf{Modal Analysis}}
				
				\includegraphics[width=0.725\textwidth]{TRD-Support-03.png}
				
				\justifying								
				\scriptsize{Finite-elements calculation \textbf{coupled load modal analysis}\footnotemark[1]:\\
								\hspace{0.17cm} - $f_0 = 67.1 Hz$\\
								\textbf{NASA requirement}: $f_0 > 50 Hz$}
			\end{figure}
		\end{column}
		\begin{column}{0.02\textwidth}\end{column}
		\begin{column}{0.425\textwidth}
			\begin{figure}
				\centering
				\small{\bluetextbf{Displacement}}
				\includegraphics[width=0.725\textwidth]{TRD-Support-04.png}
				
				\justifying								
				\scriptsize{\textbf{Main accelerations} (simulating shuttle):\\
								\hspace{0.17cm} - $7.7g$ in $x$ at start; $8.9g$ in $z$ at landing.\\
								\textbf{Maximum displacement}:\\
								\hspace{0.17cm} - $0.73 mm$, well within elastic limits.}
			\end{figure}
		\end{column}
		\begin{column}{0.015\textwidth}\end{column}
	\end{columns}
	\footnotetext[1]{\justifying\scriptsize{
	Coupled loads modal analysis (CLA) is a critical process for many high-technology systems including launch vehicles and 
	satellites. CLA predicts responses caused by major dynamic and quasistatic loading events such as liftoff, gust, buffet, and 
	engine startup and shutdown.\\
	CLA helps to minimize risk and maximize the probability of mission success.}}
	
\end{frame}
%The TRD will be fully covered in a
%multi-layer-insulation (MLI) foil to keep the spatial orbit
%temperature gradient below 1K. 

%\begin{frame}
%	\frametitle{TRD - Transition Radiation Detector}
%	\framesubtitle{structural verification of the TRD gas system}
%	\justifying
%
%	\begin{figure}
%		\centering
%		\includegraphics[height=0.45\paperheight]{TRD-GasSystem-08.png}
%			\vspace{0.075cm}
%			
%			\justifying
%			\footnotesize{\bluetextbf{Figure:} Vibration testing of the Box S mechanics. The figure in the left shows the results of FE modeling and the figure on the right the locations of the accelerometers during the test.}
%	\end{figure}
%\end{frame}

\begin{frame}
	\frametitle{TRD - Transition Radiation Detector}
	\framesubtitle{performance (proton rejection)}
	\justifying
	\small{The proton rejection power of the AMS-02 TRD design has been verified with a full 20 layer prototype in testbeams. 
	Protons are separated from electrons with a combined neural network analysis.
			 
	\vspace{-0.4cm}
	\begin{columns}
		\begin{column}{0.05\textwidth}\end{column}
		\begin{column}{0.45\textwidth}
			\begin{figure}
				\centering
				\hspace{0.5cm}\small{\bluetextbf{Tube-Energy Spectra}}
	
				\includegraphics[height=0.35\paperheight]{TRD-TestBeam.png}
			\end{figure}
		\end{column}
		\begin{column}{0.45\textwidth}
			\begin{figure}
				\centering
				\hspace{0.5cm}\small{\bluetextbf{Proton Rejection}}
				
				\includegraphics[height=0.35\paperheight]{TRD-ProtRej-03.png}
			\end{figure}
		\end{column}
		\begin{column}{0.05\textwidth}\end{column}
	\end{columns}
	
	\vspace{0.1cm}
	The \textbf{tube-energy spectra} are used as probability densities $\rho (E)$. 
	
	The \textbf{proton rejection}\footnotemark[1] is well above 100 at an electron efficiency of 90\% for proton beam energies between 15 and 250 $GeV$.}
	
	\footnotetext[1]{\justifying\scriptsize{TRD performs better in space than expected by rejection studies in beam tests on ground.}}
\end{frame}

\begin{frame}[label=TRD-ChargePerform1]
	\frametitle{TRD - Transition Radiation Detector}
	\framesubtitle{performance (measurement of the charge)}
	\justifying
	\vspace{-0.1cm}
	
	\small{To perform charge measurements is used the dependence of $dE/dx$ on particle charge, $Z$, that is also 
			dependent on path length and rigidity. }
		
	\vspace{-0.05cm}
	\begin{columns}
		\begin{column}{0.01\textwidth}\end{column}
		\begin{column}{0.5\textwidth}
			\justifying
			
			\small{Therefore, in AMS data analysis, \textbf{the distributions of $dE/dx$ in each tube} is parametrized as analytical 
			functions of these variables, called $dE/dx$ \textit{Probability Density Functions} ($dE/dx$ \textit{PDFs}), which are 
			used in determination of $Z$ by a likelihood method.

			\vspace{0.1cm}
			\textbf{For particles with $Z$ larger than 6, the ADC readout of the straw tubes is saturated}. }
			
			\vspace{0.1cm}
			To extend the measurements to 	particles with higher charge, as well as increase the			
		\end{column}
		\begin{column}{0.01\textwidth}\end{column}
		\begin{column}{0.47\textwidth}
			\centering
			\vspace{0.05cm}
			
			\includegraphics[width=\columnwidth]{TRD-ChargeMeas-dEPDFs.png}

			\vspace{-0.1cm}
			\begin{columns}
				\begin{column}{0.15\textwidth}\end{column}
				\begin{column}{0.8\textwidth}
					\justifying			
					\scriptsize{\bluetextbf{Figure:}\\
					$dE/dx$ \textit{PDFs} of He, Li, Be, and B, derived from flight data.}		
					%\vspace{0.1cm}
				\end{column}
				\begin{column}{0.05\textwidth}\end{column}
			\end{columns}						
		\end{column}
		\begin{column}{0.01\textwidth}\end{column}
	\end{columns}		
	
	\vspace{0.1cm}
	\small{resolution for light ions, additional information of the produced delta rays can also is utilized for charge measurements.}
	
	\vspace{-0.2cm}
	\hfill\hyperlink{TRD-DeltaTech1}{\textbf{\beamerbutton{more about delta rays technique}}}
\end{frame}

%% VERSIONE ALTERNATIVA DELLA SLIDE PRECEDENTE
%\begin{frame}
%	\frametitle{TRD - Transition Radiation Detector}
%	\framesubtitle{performance (measurement of the charge)}
%	\justifying
%	\vspace{-0.1cm}
%	\small{
%	To perform charge measurements is used the dependence of $dE/dx$ on particle charge, $Z$, that is also dependent on path 
%	length and rigidity. Therefore, in AMS data analysis, \textbf{the distributions of $dE/dx$ in each tube} is parametrized as analytical 
%	functions of these variables, called $dE/dx$ \textit{Probability Density Functions} ($dE/dx$ \textit{PDFs}), which are used in 
%	determination of $Z$ by a likelihood method.
%	
%	\vspace{-0.35cm}
%	\begin{columns}
%		\begin{column}{0.24\textwidth}\end{column}
%		\begin{column}{0.415\textwidth}
%			\begin{figure}
%				\centering
%				\includegraphics[width=0.95\textwidth]{TRD-ChargeMeas-dEPDFs.png}
%				
%				%\footnotesize{\caption{\justifying AAA}}
%			\end{figure}
%		\end{column}
%		\begin{column}{0.295\textwidth}\justifying
%			
%			\vspace{0.1cm}
%			\footnotesize{\bluetextbf{Figure}
%			
%			$dE/dx$ \textit{PDFs} of Helium, Lithium, Beryllium, and Boron, derived from flight data.}
%		\end{column}
%		\begin{column}{0.14\textwidth}\end{column}
%	\end{columns}
%	
%	\vspace{0.1cm}
%	\textbf{For particles with $Z$ larger than 6, the ADC readout of the straw tubes is saturated}. To extend the measurements to 
%	particles with higher charge, as well as increase the resolution for light ions, additional information of the produced delta rays 
%	can also is utilized for charge measurements.
%	}
%\end{frame}

\begin{frame}
	\frametitle{TRD - Transition Radiation Detector}
	\framesubtitle{performance (measurement of the charge)}
	\justifying
	
	\small{The TRD of AMS is able to identify light cosmic ray nuclei by measuring $dE/dx$ within its ADC dynamic range. This
	measurements can be improved and extended to higher $Z$ counting the number of delta rays 
	produced in AMS.}
	
	\vspace{-0.45cm}
	\begin{columns}[t]
		\begin{column}{0.015\textwidth}\end{column}
		\begin{column}{0.475\textwidth}
			\begin{figure}
				\centering
				%\small{\bluetextbf{Modal Analysis}}
				\includegraphics[width=0.9\textwidth]{TRD-ChargeMeas.png}
				
				%\scriptsize{\bluetextbf{Figure 1:}}\\
				\begin{columns}
					\begin{column}{0.06\textwidth}\end{column}
					\begin{column}{0.9\textwidth}
						\justifying								
						\scriptsize{\bluetextbf{Figure 1:} Distribution of charge of cosmic ray nuclei measured by the TRD alone.}
					\end{column}
					\begin{column}{0.02\textwidth}\end{column}
				\end{columns}
			\end{figure}
		\end{column}
		\begin{column}{0.04\textwidth}\end{column}
		\begin{column}{0.475\textwidth}
			\begin{figure}
				\centering
				%\small{\bluetextbf{Displacement}}
				\includegraphics[width=0.9\textwidth]{TRD-ChargeMeas-Comparation.png}
				
				%\scriptsize{\bluetextbf{Figure 2:}}\\
				\begin{columns}
					\begin{column}{0.05\textwidth}\end{column}
					\begin{column}{0.9\textwidth}
						\justifying								
						\scriptsize{\bluetextbf{Figure 2:} Comparison between charge measured by the TRD and charge measured by the 
						inner Silicon Tracker.}
					\end{column}
					\begin{column}{0.05\textwidth}\end{column}
				\end{columns}
			\end{figure}
		\end{column}
		\begin{column}{0.015\textwidth}\end{column}
	\end{columns}
	
	\vspace{0.3cm}
	\small{The performance studied from ISS flight data shows \textbf{heavy nuclei up to Fe can be identified}.}
	
\end{frame}

\subsection{ToF}
\begin{frame}[label=ToF-Det]
	\frametitle{ToF - Time of Flight}
	\justifying

%    \vspace*{-0.25cm}
	
	\begin{columns}
		\begin{column}{0.05\textwidth}\end{column}
		\begin{column}{0.44\textwidth}\centering
			\includegraphics[width=\columnwidth]{ToF-Photo-UToF-02.png}
		\end{column}
		\begin{column}{0.02\textwidth}\end{column}
		\begin{column}{0.44\textwidth}\centering
			\includegraphics[width=\columnwidth]{ToF-Photo-LToF-02.png}
		\end{column}
		\begin{column}{0.05\textwidth}\end{column}
	\end{columns}
	
	\vspace*{0.1cm}		
		
	\begin{columns}
		\begin{column}{0.13\textwidth}\end{column}
		\begin{column}{0.74\textwidth}\justifying
			\scriptsize{\bluetextbf{Figure:} The completed upper and lower ToF before the final shielding.}
		\end{column}
		\begin{column}{0.13\textwidth}\end{column}
	\end{columns}
	
	%\vspace*{-0.45cm}
	
	\begin{block}{\small{The ToF system has been completely designed and built to provide:}}
		%\vspace{-0.25cm}
		\begin{itemize}\justifying
		\small{
			\item	The \textbf{fast trigger} to the experiment.
			%\vspace{-0.05cm}
			\item	The \textbf{measurement of the time-of-flight} of the particles traversing the detector with a resolution 
						sufficient \textit{to distinguish}:
						\vspace{-0.05cm}
						\begin{itemize}\justifying
							\item[$\circ$]	\textit{upward from downward going particles} at a level of $10^{-9}$;
							\item[$\circ$]	\textit{anti-protons from electrons} up to about $1.2 GeV$.
						\end{itemize}
			%\vspace{-0.05cm}
			\item 	The \textbf{measurement of primary cosmic nuclei velocity}, with a resolution of a few percent, \textbf{and of 
						their absolute charge} up to at least $Z = 15$.
		}
		\end{itemize}
	\end{block}
	%\scriptsize{\footnotetext[1]{\justifying This is based on previous experience and well-established techniques.}}
\end{frame}


\begin{frame}
	\frametitle{ToF - Time of Flight}
	\framesubtitle{parameters taken into account while designing the ToF detector}
	\justifying
	
	\footnotesize{The main parameters taken into account have been the following:}
	
	\begin{itemize}
		\justifying
		\footnotesize{
		\item[$\circ$]	\bluetextbf{Total sensitive area -}
								AMS-02 has been designed to have a large acceptance for cosmic ray tracks. To match the full acceptance 
								of the magnet, each layer of the ToF system has to cover a circular area of about $1.6 m^2$.
		\item[$\circ$]	\bluetextbf{Trigger selection -}
								A particle able to traverse the Upper and Lower ToF is said to be inside to the AMS acceptance. In this case 
								the ToF alerts all the AMS sub-detectors and data from Tracker, TRD, RICH, ECAL, ACC and ToF itself are 
								collected, processed and stored.
		\item[$\circ$]	\bluetextbf{Weight -}
								Given the limitations in the total weight of AMS, the ToF system was allotted $268 kg$ 
								to accommodate for the detector it self and for the support structure.
		\item[$\circ$]	\bluetextbf{Power consumption -}
								The ToF system was allowed to use about $150 W$ for photomultiplier tube operation and signal read-out, 
								out of the $2 kW$ electric power given by NASA to the AMS experiment on the ISS. 
		\item[$\circ$]	\bluetextbf{Time-of-flight resolution -}
								A resolution in the ToF better than $180 ps$ is needed to satisfy the physics requirements. The choice 	was 
								$1 cm$ thick scintillator, as a compromise between the minimum thickness and the light output needed to 
								reach this resolution.
		}
	\end{itemize}	
\end{frame}


\begin{frame}
	\frametitle{ToF - Time of Flight}
	\framesubtitle{the scintillation counters of the ToF detector}
	\justifying
	
	\small{The \textbf{ToF system consists of two planes of scintillator paddles}: the first situated at the entrance of the 
	magnetic volume, the second one at the exit.
	
	Distance from the magnet: $\pm 626 mm$ - distance between the two planes:
	$120 cm$. 
	
	\textbf{Each plane contains two layers of counters}, in x and y directions, respectively.}
	
	\vspace*{-0.2cm}
	\begin{center}
		%\scriptsize{\bluetextit{Top view of the upper (left) and lower (right) ToF paddles during an assembly test.}}
		%\vspace{0.1cm}
		\includegraphics[height=0.25\paperheight]{ToF-Photo-UToF.png}
		\hspace*{0.1cm}
		\includegraphics[height=0.25\paperheight]{ToF-Photo-LToF.png}

		\vspace*{0.1cm}		
		
		\begin{columns}
			\begin{column}{0.1\textwidth}\end{column}
			\begin{column}{0.8\textwidth}\justifying
				\scriptsize{\bluetextbf{Figure (}\bluetextit{from left to right}\bluetextbf{):}  top view of the upper (layer 1 and 2) and 
								lower (layer 3 and 4) ToF planes during an assembly test. These planes are called \textit{UToF} and 
								\textit{LToF}, respectively.}
			\end{column}
			\begin{column}{0.1\textwidth}\end{column}
		\end{columns}
	\end{center}	
	
	\vspace*{-0.2cm}
	\small{Mechanical constraints, due to the support structures of the other AMS-02 detectors, required a \textbf{different 
	design for the UToF}, mechanically connected to the TRD, \textbf{and for the LToF} mechanically connected to the Unique 
	Support Structure.}
\end{frame}

\begin{frame}[label=ToF-Paddles]
	\frametitle{ToF - Time of Flight}
	\framesubtitle{a ToF paddle}
	\justifying
	
	\small{The four ToF layers have, respectively, from the first to the last: 8, 8, 10 and 8 paddles (a paddle consists of a 
	polyvinyltoluene EJ-200 scintillator).}
	\vspace*{-0.3cm}
	
	\begin{columns}
		\begin{column}{0.25\textwidth}\end{column}
		\begin{column}{0.5\textwidth}
			\begin{figure}
				\centering

				\includegraphics[width=\textwidth]{ToF-Paddle-01.png}
			\end{figure}
		\end{column}
		\begin{column}{0.25\textwidth}\end{column}
	\end{columns}	
		\begin{columns}
		\begin{column}{0.1\textwidth}\end{column}
		\begin{column}{0.8\textwidth}
			\begin{figure}		
				\centering
				\vspace*{-0.4cm}
				\scriptsize{\bluetextbf{Figure:}
				Design of a TOF paddle.}
			\end{figure}
		\end{column}
		\begin{column}{0.1\textwidth}\end{column}
	\end{columns}	

	\vspace{0.25cm}
	\small{The internal paddles, of a layer, have a rectangular shape, $1 \, cm$ thick, $12 \, cm$ width and $110 \div 135 \, cm$ 
	length, while the external counters have a trapezoidal shape with $18 \div 26 \, cm$ width and $110 \, cm$ length.
	To ensure $100 \%$ trigger efficiency, neighbouring counters are overlapped by $0.5 \, cm$.}	
	
%	\begin{columns}
%		\begin{column}{0.02\textwidth}\end{column}
%		\begin{column}{0.44\textwidth}
%			\begin{figure}
%				\centering
%				\vspace*{-0.15cm}
%
%				\includegraphics[width=\textwidth]{ToF-Paddle-01.png}
%				%\vspace*{0.1cm}
%			\end{figure}
%		\end{column}
%		\begin{column}{0.01\textwidth}\end{column}
%		\begin{column}{0.45\textwidth}\justifying
%			\vspace*{0.5cm}
%			
%			\includegraphics[width=\textwidth]{ToF-Paddle-02.png}
%			
%			\vspace{0.2cm}\scriptsize{\bluetextbf{Figures:}
%			
%			 Design of a TOF paddle (on the left) and photograph of an assembled paddle (on the top).}
%		\end{column}
%		\begin{column}{0.02\textwidth}\end{column}
%	\end{columns}	
	
	\vspace*{0.2cm}	\hfill	
	\hyperlink{ToF-EJ200}{\textbf{\beamerbutton{more on EJ-200 scintillator}}}
	\hspace{-0.1cm}
	\hyperlink{ToF-CADpannels}{\textbf{\beamerbutton{more on ToF planes design}}}
\end{frame}

\begin{frame}[label=ToF-MagneticField]
	\frametitle{ToF - Time of Flight}
	\framesubtitle{light detectors and the problem of the magnetic field}
	\justifying

	\vspace*{-0.1cm}	
	\small{The Photo-Multiplier Tubes (PMTs) used to collect the light from the ToF scintillators should operate in the strong 
	($2 \div 3 \, kG$) fringing field of the ``superconducting'' dipole magnet.}

	\vspace*{-0.05cm}	
	\begin{columns}
		\begin{column}{0.27\textwidth}\end{column}
		\begin{column}{0.33\textwidth}
			\centering
			\includegraphics[width=\textwidth]{ToF-MagneticField-01.png}
		\end{column}
		\begin{column}{0.26\textwidth}\justifying
			\justifying			
			\footnotesize{\bluetextbf{Figure:} Magnetic field magnitude at the vertical coordinate where the ToF planes are 
			positioned.}
		\end{column}
		\begin{column}{0.14\textwidth}\end{column}
	\end{columns}
	
	\vspace*{0.15cm}
	\small{Because of the field magnitude, and of weight limits, the \textbf{PMTs must withstand the magnetic field without 
	shielding}. 
			
	The choice was fine-mesh PMTs, R5946 Hamamatsu, for their capability of working in high magnetic fields, while keeping 
	good timing characteristics.}

			
%			With a 16 mesh dynode chain the gain is $10^6$ at $2000 \, V$ and zero magnetic field.

%The light guides can be straight, tilted or tilted and twisted to minimize the effect of magnetic field on the PMT signal. 
		
%	The PMTs fine-mesh R5946 Hamamatsu have been chosen for their capability of working in high magnetic fields, while keeping good timing characteristics.
	\hfill	
	\hyperlink{ToF-R5946}{\textbf{\beamerbutton{more on R5946 Hamamatsu}}}
\end{frame}

\begin{frame}
	\frametitle{ToF - Time of Flight}
	\framesubtitle{the PMT R5946 Hamamatsu}
	\justifying
	
	\vspace*{-0.05cm}	
	\small{Each side of a ToF counter is connected to 2 or 3 PMTs through plexiglass light guides that can be straight, tilted or 
	tilted and twisted to minimize the effect of magnetic field on the PMT signal.} 

	%\vspace*{-0.2cm}
	\begin{center}
		%\includegraphics[height=0.325\paperheight]{ToF-PMTphoto01.jpeg}
		\includegraphics[height=0.335\paperheight]{ToF-PMTbasic02.png}		
		\hspace*{0.25cm}
		\includegraphics[height=0.335\paperheight]{ToF-LightGuide.png}
	\end{center}		
	
	\vspace*{-0.1cm}
	\begin{columns}
		\begin{column}{0.05\textwidth}\end{column}
		\begin{column}{0.9\textwidth}\justifying
			\scriptsize{\bluetextbf{Figures:} 
			%Photo of a PMT, 
			%On the left, photo of a PMT in housing with HV divider.
		 	Dimensional outline and basing diagram (Unit: $mm$) on the left.
		 	CAD picture, on the right, showing the light guide shape needed to match a scintillator paddle to the PM oriented 
		 	according to the magnetic field vector, and all other mechanical elements.}
		\end{column}
		\begin{column}{0.05\textwidth}\end{column}
	\end{columns}
	
	\vspace*{0.2cm}
	\small{In total there are 144 PMTs, connected to the 34 ToF paddles, that are powered by 4 power supply 
	(Scintillator High Voltage - \textit{SHV}) which can provide a voltage between $1700 V$ and $2250 V$.}
\end{frame}

\begin{frame}
	\frametitle{ToF - Time of Flight}
	\framesubtitle{mechanical supports for ToF planes}
	\justifying
	
%	The long term operational reliability of the TOF system is assured by its modular design. 
%	Every counter is read at both sides by at least two photomultiplier tubes and it can still trigger with only one tube working. 
%	The same is true for the (unlikely) complete failure of one counter, as the normal trigger logic requires a signal in three layers out of four. 
%	In this way, the system is fault tolerant and somewhat redundant by design.
%	
%	%The powering and read-out electronics are protected with conformal coating from the highly ionizing low-orbit space environment and passed qualification tests to verify the absence of sparks, discharges or leakage. 
%	The one per one redundant design ensures constant operation even in case of a hardware failure in the power supplies or in the read-out electronics.
	\vspace*{-0.05cm}
	\small{Because of the space borne nature of the AMS experiment, the ToF system must \textbf{tolerate the 
	high mechanical stresses and large temperature changes} without breaking or damaging of any part of the detector. 
	
%	Moreover, it must survive for several years in the harsh space environment, the main threats being the large temperature 
%	changes depending of the ISS orbits and %attitudes.
	}
	
	\vspace*{-0.05cm}
	\begin{center}
		\includegraphics[height=0.275\paperheight]{ToF-MechStrut-UToF-02.png}		
		\hspace*{0.2cm}
		\includegraphics[height=0.275\paperheight]{ToF-MechStrut-LToF-02.png}
	\end{center}		
	
	\vspace*{-0.1cm}
	\begin{columns}
		\begin{column}{0.05\textwidth}\end{column}
		\begin{column}{0.9\textwidth}\justifying
			\scriptsize{\bluetextbf{Figures:} 
			Exploded view of the UToF (on the left) and of the LToF (right) where are visible the covers and the mechanical supports 
		 	for ToF planes.}
		\end{column}
		\begin{column}{0.05\textwidth}\end{column}
	\end{columns}
	
	\vspace*{0.25cm}
	
	\small{To accomplish all that, \textbf{the counters are housed in mechanically robust and light-tight covers} with a system for 
	fast depressurisation, supported by a structure conforming to the NASA specifications for Shuttle payloads about resistance to 
	load and vibrations. The use of composite materials allows to keep the weight low.}
\end{frame}

\begin{frame}
	\frametitle{ToF - Time of Flight}
	\framesubtitle{space qualification}
	\justifying
	
	\small{To assess the ToF capability of operating in space, all of its components underwent several thermo-vacuum and 
	vibration tests before their integration. 
	
	Subsequently, the UToF and LToF have been space qualified, passing through strict thermo-vacuum and vibration tests at 
	SERMS laboratories in Terni.}
	
	\vspace*{-0.05cm}
	\begin{center}
		\includegraphics[height=0.315\paperheight]{ToF-SpaceQual-04.jpg}		
		\hspace*{0.45cm}
		\includegraphics[height=0.315\paperheight]{ToF-SpaceQual-05.jpg}
	\end{center}		
	
	\vspace*{-0.1cm}
	\begin{columns}
		\begin{column}{0.1\textwidth}\end{column}
		\begin{column}{0.8\textwidth}\justifying
			\scriptsize{\bluetextbf{Figures:} 
		 	UToF being inserted into the thermo-vacuum chamber (left) and on the sliding table during the vibration test in the (x, y) 
		 	plane (right).}
		\end{column}
		\begin{column}{0.1\textwidth}\end{column}
	\end{columns}
	
	\vspace*{0.2cm}	
	
	\small{\textbf{To ensure a long term reliability, a certain degree of redundancy has been implemented in the ToF system}: each 
	counter is 4(6)-fold redundant in PMTs, and doubly redundant in powering and read-out electronics. }
\end{frame}


\begin{frame}
	\frametitle{ToF - Time of Flight}
	\framesubtitle{performance (velocity measurement)}
	\justifying
	
	\small{Since May 2011, the ToF detector is operating in the space station without major problems.}
		
	\begin{columns}
		\begin{column}{0.13\textwidth}\end{column}
		\begin{column}{0.55\textwidth}
			\centering
			\includegraphics[width=\textwidth]{ToF-VelRes.png}
		\end{column}
		\begin{column}{0.01\textwidth}\end{column}
		\begin{column}{0.27\textwidth}\justifying
			\justifying
			\vspace*{-0.5cm}	
					
			\footnotesize{\bluetextbf{Figure:}\\
			The velocity resolution of the ToF system improves with increasing particle charge $Z$.}
		\end{column}
		\begin{column}{0.04\textwidth}\end{column}
	\end{columns}
	\vspace*{0.2cm}
	
	\small{The particle velocity resolution is 4\% for $Z = 1$ and 1.2\% for $Z = 6$, corresponding to a time of flight resolution 
	from $160 \, ps$ to $48 \, ps$, allowing to distinguish upward to downward going particles at level of $10^9$.}
\end{frame}


\begin{frame}
	\frametitle{ToF - Time of Flight}
	\framesubtitle{performance (trigger)}
	\justifying
	
	\vspace*{-0.15cm}
	\small{The normal trigger configuration in space requires a hit on three ToF layers out of four (in this way, the system is
	tolerant for the complete failure of one counter).}
	
	\vspace*{-0.15cm}
	\begin{center}
		\includegraphics[height=0.375\paperheight]{ToF-Trigger-01.png}		
		\hspace*{0.45cm}
		\includegraphics[height=0.375\paperheight]{ToF-Trigger-02.png}
	\end{center}		
	
	\vspace*{-0.1cm}
	\begin{columns}
		\begin{column}{0.02\textwidth}\end{column}
		\begin{column}{0.96\textwidth}\justifying
			\scriptsize{\bluetextbf{Figures:} 
			Left - the trigger efficiency curve for one ToF counter: the efficiency has been measured at different
			PMT High Voltages (HV), close to the nominal one (up to $\pm 250 \, V$). 
			
			Right - the trigger efficiencies for all counters of the four ToF layers\footnotemark[1]. }
		\end{column}
		\begin{column}{0.02\textwidth}\end{column}
	\end{columns}
	
	\vspace*{0.1cm}		
	
	\footnotetext[1]{\justifying{\scriptsize{
	The ToF counter effiency is computed using the OR of the two sides. 
	The efficiencies are of the order of 99\% for all, with the exception of two external counters.}}}
\end{frame}


\begin{frame}
	\frametitle{ToF - Time of Flight}
	\framesubtitle{performance (charge measurement)}
	\justifying
	
	\vspace*{-0.15cm}
	\small{Once all the dynamic and static calibration procedures are applied, the ToF is capable of measuring all particle charges 
	from $Z = 1$ to $Z = 40$ with high precision. }
	%Both anodes and dynodes have good resolution for charge measurement. Low charges ($1 \leqslant Z \leqslant 3$) are measured using anodes, middle charges ($4 \leqslant Z \leqslant 8$) are measured using both anodes and dynodes, and high charges ($Z > 8$) are measured using dynodes.

	\vspace*{-0.15cm}
	\begin{center}
		\includegraphics[height=0.35\paperheight]{ToF-ChargeMeas-03.png}		
		\hspace*{0.45cm}
		\includegraphics[height=0.35\paperheight]{ToF-ChargeMeas-01.png}
	\end{center}		
	
	\vspace*{-0.2cm}
	\begin{columns}
		\begin{column}{0.02\textwidth}\end{column}
		\begin{column}{0.96\textwidth}\justifying
			\scriptsize{\bluetextbf{Figures:} 
		 	In the left graph are shown ToF charge estimation for different ions. Instead, in the right one are shown anode (dashed line) and dynode (solid line) charge resolution in charge units (c.u.) as a function of $Z$; the error bars represent the standard deviation of the distribution of resolution for all ToF counters \cite{Bindi-3}.}
		\end{column}
		\begin{column}{0.02\textwidth}\end{column}
	\end{columns}
	
	\vspace*{0.2cm}	
	\small{A single ToF counter has a charge resolution $\sim 0.16$ charge unit (c.u.) for Carbon ($Z = 6$) and $\sim 0.4$ charge unit for Iron ($Z = 6$).}
\end{frame}

\subsection{Spectrometer}
\begin{frame}[label=SM-Det]
	\frametitle{Magnetic Spectrometer}
	
	\begin{columns}
		\begin{column}{0.01\textwidth}\end{column}
		\begin{column}{0.66\textwidth}
			\justifying
			\vspace*{-0.1cm}
			
			\small{The Magnetic Spectrometer (MS) is \textbf{the heart of the AMS experiment}.}
			
			\vspace*{0.1cm}			
			
			\small{Thanks to the magnetic field of the MS, AMS is able to measure the sign of the traversing particle and so 
			to discriminate matter from antimatter.} 
			%and from the radius of curvature it is possible to estimate the particle momentum.
			
			\vspace*{0.15cm}

			\begin{block}{\footnotesize{Main component of the MS}}\justifying
				\begin{columns}
					\begin{column}{0.99\textwidth}
						\vspace*{-0.25cm}
						\begin{itemize}\justifying
							\item	\footnotesize{The \hyperlink{MAG-Det}{\bluetextit{Magnet}}, which produces the magnetic field.}
							\item	\footnotesize{The \hyperlink{ST-Det}{\bluetextit{Silicon Tracker}}, that reconstruct the tracks of 
										charged particles and measure, with an excellent resolution, the magnitude of their charges.}
						\end{itemize}
					\end{column}
					\begin{column}{0.01\textwidth}\end{column}
				\end{columns}
			\end{block}
		\end{column}
		\begin{column}{0.02\textwidth}\end{column}	
		\begin{column}{0.3\textwidth}
			\centering	
			\scriptsize{\bluetextbf{Schematic picture of MS}}
						
			\includegraphics[height=0.555\paperheight]{MS-Magnet_Deflection-02.png}
		\end{column}	
		\begin{column}{0.01\textwidth}\end{column}
	\end{columns}
	\vspace*{0.15cm}
	\justifying
			\scriptsize{In the \bluetextit{figure on the right} is shown a schematic representation of a charged particle and its 
			antiparticle that, with the same energy, pass through the MS.
			Thanks to the magnetic field, the trajectories of the two particles are deflected of the same amount (because they have 
			the same rigidity $R = p / |Z|$), 	but in opposite directions due to the different sign of their charges.}
\end{frame}
			
\subsubsection{Magnet}%\pdfbookmark[4]{Magnet}{Magnet}
\begin{frame}[label=MAG-Det]
	\frametitle{Magnet}
	\framesubtitle{the permanent magnet of AMS-02}
	\justifying

%	The AMS-02 magnet is a permanent magnet that designed to generating a uniform dipole magnetic field and it is designed to 
%	get a large acceptance ($\sim 1 m^2 sr$) and strong analysing power ($BL^2 = 0.15 T m^2$) the requirements of a large, 
%	powerful and uniform dipole magnetic field of a flight-qualified and relatively light-weighted system. To realize a device of this 
%	type, the main parameters taken into account have been the following:	
	\footnotesize{The AMS-02 magnet is a \textbf{permanent magnet designed to generating a uniform dipole magnetic field} 	for
	a flight-qualified and relatively light-weighted system.}
	
	\vspace*{0.25cm}
	
	\begin{columns}
		\begin{column}{0.3\textwidth}\end{column}
		\begin{column}{0.37\textwidth}\centering
			\includegraphics[width=\textwidth]{MAGNET-Photo-02.png}
		\end{column}
		\begin{column}{0.01\textwidth}\end{column}
		\begin{column}{0.26\textwidth}\justifying
			\scriptsize{\bluetextbf{Figure:}\\
			Photo shows the permanent magnet of AMS-02, namely the same magnet employed in the AMS-01 prototype.}
		\end{column}
		\begin{column}{0.06\textwidth}\end{column}
	\end{columns}
	\vspace*{0.25cm}
	
	\footnotesize{So, the main parameters taken into account to realize this device have been the following:}
%	To realize a device of this type, the main parameters taken into account have been the following:

	\vspace*{-0.1cm}
	\begin{itemize}\justifying
		\footnotesize{
		\item The magnet will be launched into space, hence its weight must be as low as possible.
		\item To avoid the yaw and pitch caused by the interaction of the magnet with the earth magnetic field, or in other words to 
		ease the burden of the satellite altitude control system, there should be compensating facilities on the magnet so that the 
		overall magnetic dipole moment of the magnet is minimum.
		}
	\end{itemize}
\end{frame}

\begin{frame}[label=MAGNET-Des]
	\frametitle{Magnet}
	\framesubtitle{design of the magnet structure}
	\justifying
	%The commonly known structures of iron-free hollow dipole magnets that minimize the magnet with low leakage flux
	\vspace*{-0.15cm}	
	\footnotesize{
	The magnet is made from 64 high-grade Nd-Fe-B sectors. 
	Each sector is composed of 100 blocks (size of a block: $5.08 \times 5.08 \times 2.54 \, cm^3$). 
	%Figure 1.2 shows the arrangement of the field direction of the 64 sectors.
	This configuration produce a dipole field of $1.5 \, kG$ and a negligible dipole moment\footnotemark[1].
	In addition, the flux leakage\footnotemark[2] at a distance of $2 \, m$ from the center of the magnet is $3 \, G$.}
			
	\begin{columns}
		\begin{column}{0.33\textwidth}\end{column}
		\begin{column}{0.34\textwidth}\centering
			\includegraphics[width=\textwidth]{MAGNET-Struct-05.png}
		\end{column}
		\begin{column}{0.01\textwidth}\end{column}
		\begin{column}{0.23\textwidth}\justifying
			\vspace*{0.15cm}
			
			\scriptsize{\bluetextbf{Figure:}\\
			The image shows the arrangement of the field direction of the 64 sectors.}
			
			\vspace*{0.15cm}\centering
			\hyperlink{MAGNET-FieldDistr}{\textbf{\beamerbutton{more on field distribution}}}
		\end{column}
		\begin{column}{0.09\textwidth}\end{column}
	\end{columns}
	%\vspace*{-0.15cm}	
	
	\footnotetext[1]{\justifying{\scriptsize{The geomagnetic field is $0.5 \, G$ on the surface and varies when on orbit. 
	A strong dipole moment would result in an undesirable force on the space shuttle or the space station.}}}
	\footnotetext[2]{\justifying{\scriptsize{NASA requires the leakage field to be $< 300 \, G$ so as not to interfere with the life 
	support system of the astronauts.}}}
\end{frame}

\begin{frame}
	\frametitle{Magnet}
	\framesubtitle{the magnet and the reference frame of AMS-02}
	\justifying
	%The commonly known structures of iron-free hollow dipole magnets that minimize the magnet with low leakage flux
			
	\small{The orientation of the almost uniform magnetic field defines the whole reference frame of the AMS-02 
	experiment.}
	
	\begin{columns}
		\begin{column}{0.01\textwidth}\end{column}
		\begin{column}{0.53\textwidth}
			\begin{block}{\footnotesize{Reference Frame of AMS-02}}
				\begin{columns}
					\begin{column}{0.99\textwidth}			
						\begin{itemize}\justifying
							\vspace*{-0.25cm}
							\footnotesize{
							\item[$\circ$]	The center is at the center of the magnet. 
							\item[$\circ$]	The $x$ axis is oriented along the direction of the field lines. 
							\item[$\circ$]	The $z$ axis is defined by the magnet cylinder symmetry axis, with positive values toward the 
													top of the instruments.
							\item[$\circ$]	The $y$ axis completes the Cartesian right-handed tern. 
							}
						\end{itemize}
						\vspace*{0.05cm}
					\end{column}
					\begin{column}{0.01\textwidth}\end{column}
				\end{columns}
			\end{block}
		\end{column}
		\begin{column}{0.01\textwidth}\end{column}
		\begin{column}{0.44\textwidth}\centering
			\vspace*{0.25cm}
			
			\scriptsize{\bluetextit{Properties of the AMS-02 flight magnet}}			
						
			\includegraphics[width=0.85\textwidth]{MAGNET-Struct-04.png}			
		\end{column}
		\begin{column}{0.01\textwidth}\end{column}
	\end{columns}

	\vspace*{0.25cm}
	
	\small{\textbf{All the curved motion of a charged particle in the AMS-02 magnetic field is contained in the $z y$ 
	plane.}}	
	
%	\vspace*{-0.15cm}
%	\begin{center}
%		\includegraphics[height=0.35\paperheight]{MAGNET-Struct-04.png}
%	\end{center}		
%	
%	\vspace*{-0.1cm}
%	\begin{columns}
%		\begin{column}{0.02\textwidth}\end{column}
%		\begin{column}{0.96\textwidth}\centering
%			\scriptsize{\bluetextbf{Figure} Properties of the AMS-02 flight magnet.}
%		\end{column}
%		\begin{column}{0.02\textwidth}\end{column}
%	\end{columns}
\end{frame}

\begin{frame}
	\frametitle{Magnet}
	\framesubtitle{the superconducting magnet of AMS-02}
	\justifying
	
	\footnotesize{AMS also developed a \bluetextit{Super-Conducting Magnet (SCM)} which was suitable for a three year mission
	on the ISS as originally planned.}
			
	\vspace*{-0.1cm}
	\begin{columns}
		\begin{column}{0.01\textwidth}\end{column}
		\begin{column}{0.63\textwidth}\justifying

			\begin{block}{\footnotesize{Something about the SCM:}}\justifying
				\begin{columns}
				\begin{column}{0.01\textwidth}\end{column}
					\begin{column}{0.99\textwidth}\justifying
						\footnotesize{The SCM was the same volume as the permanent one but its field was stronger 
						($\sim 8.6 \, kG$ at the center).
						
						Using 2500 litres of superfluid He, \textbf{it operated at an extremely low temperature}, 
						namely $1.8 ^\circ{K}$.
						
						Many new technologies (i.e. thermomechanical pumps, cryocoolers, etc.) were specifically 
						modified for AMS to ensure its functioning, safety and endurance in space.}
					\end{column}
					\begin{column}{0.01\textwidth}\end{column}
				\end{columns}
			\end{block}
		\end{column}
		\begin{column}{0.01\textwidth}\end{column}	
		\begin{column}{0.35\textwidth}
			\vspace*{0.15cm}

			\centering	
			\includegraphics[width=\textwidth]{MAGNET-SupercondMagnet.png}
			
			\scriptsize{\bluetextbf{	Schematic view\\
												of the SCM layout}}
												
			\end{column}	
		%\begin{column}{0.01\textwidth}\end{column}
	\end{columns}
	
	\vspace*{0.3cm}
	
	\footnotesize{
	The SCM underwent extensive testing that validated its performance and endurance 
	in simulated space conditions.
	These tests also confirmed that \textbf{the magnet would lose its superconductivity when the superfluid He boiled off after 
	approximately three years of operation}, exactly the length of the mission for which it was originally designed
	(\textit{because of the retirement of the shuttle fleet, there would have been no means of re-supplying AMS with additional 
	liquid He).}}	
\end{frame}


\begin{frame}[label=MAGNET-Choice]
	\frametitle{Magnet}
	\framesubtitle{why the choice of the permanent magnet?}
	\justifying
	\small{The SCM was not used for flight because the lifetime extension of the ISS made the permanent magnet (PM) 
	with its extended operational capability and promise of higher yield of data, the instrument of choice.}
	
	\vspace*{0.15cm}

	\begin{columns}
		\begin{column}{0.01\textwidth}\end{column}	
		\begin{column}{0.53\textwidth}\justifying
			\vspace*{-0.1cm}
			\begin{block}%{\footnotesize{Improvement for a long duration mission}}
			{\footnotesize{Improvement for a long-term experiment with high rigidity resolution:}}\justifying
				\begin{columns}
				\begin{column}{0.01\textwidth}\end{column}
					\begin{column}{0.98\textwidth}\justifying
						\footnotesize{To offset the lower magnetic field of the PM, \textit{the AMS Collaboration optimized the 
						geometry of the detector by adding one silicon planes and rearranging the existing silicon planes}, 
						\textbf{thus greatly increasing the measurement arm} 
						(specifically: $l \simeq 0.8 \, m \: \div \: L \simeq 3 \, m$).}
					\end{column}
					\begin{column}{0.01\textwidth}\end{column}
				\end{columns}
			\end{block}
			 
			\vspace*{0.1cm}
			
			\small{This recovered the full sensitivity for AMS for the rigidity and for the matter-antimatter separation as was 
			confirmed with a dedicated beam test.}
			
		\end{column}	
		\begin{column}{0.02\textwidth}\end{column}	
		\begin{column}{0.43\textwidth}\centering
			\scriptsize{\bluetextbf{Comparison of the Two Scenarios}}
			
			\vspace*{0.05cm}
			\includegraphics[width=\textwidth]{MAGNET-Comparition-01.png}
			
			\vspace*{0.15cm}\justifying
			\includegraphics[height=0.1\paperheight]{MAGNET-Comparition-03.png}
			\hspace*{0.4cm}
			\includegraphics[height=0.1\paperheight]{MAGNET-Comparition-04.png}
			
			\vspace*{0.2cm}
			
			\centering
			\hyperlink{MAGNET-CompRes}{\textbf{\beamerbutton{comparison of the two rigidity resolutions}}}
		\end{column}	
		\begin{column}{0.01\textwidth}\end{column}	
	\end{columns}
	
%	\vspace*{0.5cm}
%	\hspace*{7.5cm}\hyperlink{MAGNET-CompRes}{\textbf{\beamerbutton{comparison of the two rigidity resolutions}}}
%	
\end{frame}


\begin{frame}
	\frametitle{Magnet}
	\framesubtitle{why the choice of the permanent magnet?}
	\justifying

	\begin{columns}
		\begin{column}{0.01\textwidth}\end{column}
		\begin{column}{0.55\textwidth}\justifying
			\small{As a side effect of the lever arm extension, the number of CRs per second passing through the tracking volume is 
			reduced. 
			
			\vspace*{0.1cm}
	
			This is because longer cylinders, with the same diameter, have narrower angular field-of-views (\textbf{reduction in 
			acceptance}).}
	
			\vspace*{-0.05cm}
			
			\begin{block}{\footnotesize{CRs yield of the two configurations}}\justifying
				\begin{columns}
				\begin{column}{0.01\textwidth}\end{column}
					\begin{column}{0.98\textwidth}\justifying
						\footnotesize{However the number of CRs collected in 10 years in the PM Scenario (the worst case in this 
						scenario) is greater than the number of CRs that could be collected in the best case with the SCM Scenario 
						(at most 3 years).
						
						\vspace*{0.05cm}
			
						\textit{As an example the number of positrons collected in the PM configuration is expected to be a factor 2 to 6 
						times greater to the SCM configuration, depending on the energy.}}
					\end{column}
					\begin{column}{0.01\textwidth}\end{column}
				\end{columns}
			\end{block}
		\end{column}
		\begin{column}{0.01\textwidth}\end{column}		
		\begin{column}{0.42\textwidth}\centering
			\hspace*{0.2cm}\scriptsize{\bluetextbf{The Expected Performances}}
			
			\includegraphics[width=0.85\textwidth]{MAGNET-Comparition-06.png}
			
			\vspace*{0.1cm}
			\begin{columns}
				\begin{column}{0.04\textwidth}\end{column}
				\begin{column}{0.95\textwidth}\justifying
					\scriptsize{\bluetextbf{Figure:} The expected positron fraction in the two AMS-02 configurations, considering a
					particular model of annihilating Dark Matter.}
				\end{column}
				\begin{column}{0.01\textwidth}\end{column}
			\end{columns}
			
		\end{column}
		\begin{column}{0.01\textwidth}\end{column}
	\end{columns}
\end{frame}

\subsubsection{Silicon Tracker}\pdfbookmark[4]{Silicon Tracker}{Silicon Tracker}

\begin{frame}[label=ST-Det]
	\frametitle{Silicon Tracker}
	\framesubtitle{physical goals}
	
	\vspace{-0.2cm}
	\begin{center}
		\includegraphics[height=0.32\paperheight]{TRACKER-Photo-01.png}
	\end{center}
	\vspace{-0.25cm}

	\begin{block}{\small{The physical aims of the Silicon Tracker are:}}
		\vspace*{-0.25cm}
		
		\begin{columns}
			\begin{column}{0.99\textwidth}
				\begin{itemize}\justifying\small{
					\item	\textbf{measure the position of charged particles}, with a precision of few tens of 
								microns, in order to reconstruct their trajectories inside the AMS detector for the determination of:
								\vspace*{-0.15cm}
								
								\begin{itemize}\justifying
									\item[$\circ$] 	\textit{rigidity,} $R = p / |Z|$; 
									\item[$\circ$]	\textit{sign of the charge,} once that is determined with the ToF if they are
															going upward or downward.
								\end{itemize}
								\vspace*{-0.1cm}
																
					\item \textbf{measure the magnitude of the charge}, $|Z|$.
				}\end{itemize}
			\end{column}
			\begin{column}{0.01\textwidth}\end{column}
		\end{columns}
		%\vspace{-0.25cm}
		
%		\vspace{-0.55cm}
%		\begin{center}
%			\includegraphics[height=0.35\paperheight]{TRD-Photo-05.jpeg}
%		\end{center}
	\end{block}
\end{frame}

\begin{frame}[label=TRK-DsidedDet]
	\frametitle{Silicon Tracker}
	\framesubtitle{basic element of the tracking system}
	\justifying
	\small{The basic elements of the Silicon Tracker are the \textbf{double-sided microstrip sensors}, 
	size:  $41.360 \times 72.045 \times 0.300 \, mm^3$. 
	%Its size is $41.360 \times 72.045 \times 0.300 \, mm^3$.		
	The thickness of the silicon ($300 \, \mu m$), as well as the choice of a double-side detector, is made to 
	\textit{minimize the amount of material along the particles path in order to reduce the multiple scattering}.}
	
	\begin{columns}
		\begin{column}{0.01\textwidth}\end{column}
		\begin{column}{0.57\textwidth}\justifying
			\begin{block}{\footnotesize{Dimensions}}\justifying
				\begin{columns}
				\begin{column}{0.01\textwidth}\end{column}
					\begin{column}{0.98\textwidth}\justifying
						\footnotesize{On each face of the sensor, metallic strip implants run in perpendicular directions, providing a 
						\textbf{2D measurement for each sampling}. In particular:}
						\vspace*{-0.05cm}
						\begin{itemize}\justifying{\footnotesize{
							\item[$\triangleright$]	on the \textbf{junction side}, also called $p$-side or S-side, the $p^+$ strips 
								have implantation and readout pitches of 27.5 and $110 \, \mu m$;
								\vspace*{-0.05cm}
							\item[$\triangleright$]	on the \textbf{ohmic side}, also called $n$-side or K-side, the $n^+$ strips have 
								implantation and readout pitches of 104 and $208 \, \mu m$. 
						}}\end{itemize}
					\end{column}
					\begin{column}{0.01\textwidth}\end{column}
				\end{columns}
			\end{block}
		\end{column}
		\begin{column}{0.01\textwidth}\end{column}		
		\begin{column}{0.4\textwidth}\centering
			%\hspace*{0.2cm}\scriptsize{\bluetextbf{Anus}}
			
			\includegraphics[width=0.85\textwidth]{TRACKER-Sensor-04.png}
			
			\vspace*{0.1cm}
			\begin{columns}
				\begin{column}{0.04\textwidth}\end{column}
				\begin{column}{0.92\textwidth}\justifying
					\scriptsize{\bluetextbf{Figure:} Schematic view of a double-sided microstrip detector.}
				\end{column}
				\begin{column}{0.04\textwidth}\end{column}
			\end{columns}
			
		\end{column}
		\begin{column}{0.01\textwidth}\end{column}
	\end{columns}
	
	\vspace*{0.5cm}
	\hfill\hyperlink{TRK-CCreadout}{\textbf{\beamerbutton{readout implementation}}}
\end{frame}

\begin{frame}[label=TRK-DsidedProblem]
	\frametitle{Silicon Tracker}
	\framesubtitle{the problem of the double-sided strip detectors 
	\hspace*{2.8cm}
	\hyperlink{TRK-nstrip}{\textbf{\beamerbutton{the AMS-02's solution}}}}
	\justifying
	\vspace*{-0.05cm}	

	\small{If one simply tries to subdivide the contacts on both sides of a silicon detector, the presence of a 
				positive charge at the $\text{Si} - \text{SiO}_2$ interface induces an \textit{electron accumulation layer in the 
				$n$-type silicon substrate} (see fig. 1).
				This $n$-type channel, which cannot be depleted by any reasonable bias voltage applied across the wafer, 
				dramatically \textit{lowers the resistance between $n^+$ strips, making the signal charge to spread 
				over many electrodes} (\textbf{no position measurement possible}).}
	\vspace*{-0.3cm}	
	
	\begin{columns}
		\begin{column}{0.015\textwidth}\end{column}
		\begin{column}{0.425\textwidth}
			\begin{figure}
				\centering				
				\includegraphics[height=0.2\paperheight]{TRACKER-DSideprom-01.png}
												
				\scriptsize{\bluetextbf{Figure 1:} Illustration of the problem.}
			\end{figure}
		\end{column}
		\begin{column}{0.02\textwidth}\end{column}
		\begin{column}{0.425\textwidth}
			\begin{figure}
				\centering
				\includegraphics[height=0.2\paperheight]{TRACKER-DSideprom-02.png}
												
				\scriptsize{\bluetextbf{Figure 2:} Example of this solution.}
			\end{figure}
		\end{column}
		\begin{column}{0.015\textwidth}\end{column}
	\end{columns}
	\vspace*{-0.05cm}	

	\begin{block}{\footnotesize{Solution to the problem - \textit{$p^+$ blocking strips}}}\justifying
		\begin{columns}
			\begin{column}{0.01\textwidth}\end{column}
			\begin{column}{0.98\textwidth}\justifying
				\footnotesize{To solve this problem, on the ohmic side, \textit{$p^+$ blocking strips between every two $n^+$
				strips are inserted} (see fig. 2). 
				All $p^+$ blocking strips are left floating; their only function is to interrupt the conducting channel. 
				Thus, \textbf{the interstrip resistance is greatly increased.}}
%				when the applied reversed bias voltage approaches the full depletion voltage
			\end{column}
			\begin{column}{0.01\textwidth}\end{column}
		\end{columns}
	\end{block}
\end{frame}

\begin{frame}[label=TRK-CCandPR]
	\frametitle{Silicon Tracker}
	\framesubtitle{capacitive coupling and position resolution}
	\justifying
%The ionization loss of singly charged particles traversing the fully depleted, reverse-biased
%$300 \pm 10 \, \mu m$ thick sensor is described by a Landau distribution. 
%The peak energy loss of a singlycharged, MIP at normal incidence produces 22,000 electron-hole pairs. 
%The opposite sign $+ / -$ charge carriers drift rapidly ($10 \div 25 \,  ns$) in the electric field to the two surfaces ($p / n$)
%where the accumulated charge on the metalized strips ($p^+ / n^+$) is fed to the front-end electronics. 
%The position of the particle is determined by the relative signal levels observed at the readout strip positions.
%At the single sensor level, the position resolution depends on the sampling pitch and the signal-to-noise
%performance.

	\small{The analog readout and the capacitative coupling (CC) between the implanted strips allow to apply a center-of-gravity 
	technique to achieve a position measurement resolution, for $Z = 1$ particles, of $10 \, \mu m$ for $p$-side and 
	$30 \, \mu m$ for $n$-side.
	
	\vspace*{0.1cm}
	Thus, the side used to measure the $y$ coordinate in the bending plane ($yz$) of the particle trajectory is the $p$-side.}

	\vspace*{0.2cm}
	\begin{columns}
		\begin{column}{0.06\textwidth}\end{column}
		\begin{column}{0.55\textwidth}\centering
			\includegraphics[width=\textwidth]{TRK-CapacitiveCoupling-03b.png}
		\end{column}
		\begin{column}{0.01\textwidth}\end{column}
		\begin{column}{0.33\textwidth}\justifying
			\scriptsize{\textitem{Figure}{example of CC}\\
			The scheme shows an example of a detector with two intermediate strips. 
			Only every $3^{rd}$ strip is connected to an electronics channel. 
			The charge from the intermediate strips is capacitive coupled to the neighbour strips.}
			
			\centering
			\vspace*{0.1cm}
			\hyperlink{TRK-PR_prob}{\textbf{\beamerbutton{more on the position resolution}}}
		\end{column}
		\begin{column}{0.05\textwidth}\end{column}
	\end{columns}
\end{frame}

\begin{frame}
	\frametitle{Silicon Tracker}
	\framesubtitle{double-sided strip detector biassing}
	\justifying
	
	\vspace*{-0.05cm}	
	
	\small{The microstrip detectors operate at \textbf{full depletion by punch-through techniques} that permit to ground 
	all the $p^+$ strips and bias, at $+80 \, V$, the $n^+$ without direct contacts to each strip.}

	\vspace*{-0.05cm}	

	\begin{center}
		\includegraphics[width=0.32\textwidth]{TRK-PunchThrough-01.png}	
		\hspace*{0.05cm}
		\includegraphics[width=0.32\textwidth]{TRK-PunchThrough-02.png}
		\hspace*{0.05cm}	
		\includegraphics[width=0.32\textwidth]{TRK-PunchThrough-03.png}	
	\end{center}
	\scriptsize{\bluetextbf{Figures (}\bluetextit{$p^+$ strip punch-through biassing}\bluetextbf{).} 
			The three diagrams show: 
			(\textbf{left}) depletion zones before merging; 
			(\textbf{center}) reach punch-through voltage $V_2$, depletion zones merge; 
			(\textbf{right}) depletion zones after merging - full depletion not yet reached.
			N.B. The $n^+$ strips have a different, slightly more complex, biassing: 
			the FOXFET biassing (\textit{Field OXide Field Effect Transistor}).}
	\vspace*{0.1cm}
	
	\small{Indeed, the simplest way to bias the strips could be using the preamplifier readout channels,
	but this solution has two drawbacks:}
	\vspace*{-0.2cm}
	
	\begin{itemize}\justifying\small{
		\item[$\circ$]	only the strips connected to the readout electronics are biased, forbidding the use of intermediate strips; 
		\vspace*{-0.15cm}
		
		\item[$\circ$]	channels with too high a leakage current may saturate the amplifier.
	}\end{itemize}
\end{frame}

\begin{frame}
	\frametitle{Silicon Tracker}
	\framesubtitle{improvement of the AMS-02 microstrip detectors}
	\justifying
	\small{The AMS-01 microstrip detector results from the R\&D realized from the 1980's until the 1990's 
	to create first the ALEPH detector design, then the L3 detector.}
	\vspace*{0.15cm}
	
	\begin{columns}
		\begin{column}{\textwidth}\centering
			\includegraphics[height=0.49\paperheight]{TRACKER-SensorGeom.png}
			\hspace*{0.5cm}
			%\includegraphics[width=\textwidth]{TRACKER-AMS01vs02-nside.png}
			\includegraphics[height=0.49\paperheight]{TRACKER-AMS01vs02.png}
		\end{column}
	\end{columns}
	
	\vspace*{0.15cm}

	\begin{columns}
		\begin{column}{0.01\textwidth}\end{column}
		\begin{column}{0.98\textwidth}\justifying
			\scriptsize{\bluetextbf{Figures:} For AMS-02, the detector design was upgraded to decrease as much as possible the
			noise transmitted to the readout channel. 
			Indeed, the long silicon modules of AMS (up to 15 miscrostrip detectors) are not common in particle physics
			experiments, due to noise limitations. This kind of configuration introduces effects usually neglected for 	smaller modules.}
		\end{column}
		\begin{column}{0.01\textwidth}\end{column}
	\end{columns}
\end{frame}

\begin{frame}
	\frametitle{Silicon Tracker}
	\framesubtitle{ladder: the basic readout element}
	\justifying
	
	\vspace*{-0.1cm}

	\small{The basic readout element of the Silicon Tracker is called ``\textit{ladder}'', the figures show the two sides 
	of two ladders (sensors: 9 left; 15 right) during their assembly.}
	
	\vspace*{0.2cm}

	\begin{columns}
		\begin{column}{0.01\textwidth}\end{column}
		\begin{column}{0.65\textwidth}\justifying
			\small{Each ladder consists of 7 to 15 wire-bonded sensors (double-sided microstrip detectors). 
			
			The energy deposit is read out by \textbf{1024 readout channels per ladder}: 640 on the $p$-side, 384 on the 
			$n$-side. 
			The readout electronics is placed directly on the front end board at the end of the ladder. 
			
			The AMS tracking system consists of \textbf{$\approx$ 200 ladders} 
			\textit{for a total of about $2 \cdot 10^5$ readout channels} and a total active area of $6.74 m^2$.}
	
			\vspace*{0.3cm}
			\centering
			\includegraphics[width=0.95\textwidth]{TRACKER-Photo_Ladder-01.png}\\
			\bluetextbf{(}\bluetextit{Ohmic side}\bluetextbf{)}			
		\end{column}
		\begin{column}{0.01\textwidth}\end{column}
		\begin{column}{0.32\textwidth}\centering
			\includegraphics[width=\textwidth]{TRACKER-Photo_Ladder-02.png}\\
			\footnotesize{\bluetextbf{(}\bluetextit{Junction side}\bluetextbf{)}}
%			
%			\includegraphics[width=\textwidth]{TRACKER-Photo_Ladder-04.png}
%			
%			\includegraphics[width=\textwidth]{TRACKER-Photo_Ladder-05.png}
		\end{column}
		\begin{column}{0.01\textwidth}\end{column}
	\end{columns}	
	
	\vspace*{0.1cm}
\end{frame}

%\begin{frame}[label=TRK-Ladder1]
%	\frametitle{Silicon Tracker}
%	\framesubtitle{the structure of a ladder}
%	\justifying
%	
%	\small{The array of sensor that compound a ladder are glued on a Upilex cable and mounted on an Airex support (\textit{Ladder Reinforcement}) that confers
%	the mechanical}
%	
%	%\vspace*{0.1cm}
%	
%	\begin{columns}
%		\begin{column}{0.01\textwidth}\end{column}
%		\begin{column}{0.49\textwidth}\justifying
%			\small{stability to the module.
%			
%			The ladder is protected from the light 
%			(source of noise in the silicon substrate) 
%			and from the electromagnetic interferences by a Upilex metallized foil (\hyperlink{TRK-Upilex}{\bluetextit{link 1}}).
%			
%			\vspace*{0.1cm}
%
%			The Upilex foil is separated from the silicon sensor surface by an Airex structure (\textit{Shield Support}).
%			
%			\vspace*{0.2cm}
%			
%			In order to maximize the layer acceptance and to minimize the number of readout channels the front end 
%			electronics is placed on the ladder end for both sensor sides
%			(the bonding schemes of the sensors are shown in the next slide).}
%		\end{column}
%		\begin{column}{0.01\textwidth}\end{column}
%		\begin{column}{0.48\textwidth}\centering
%			\footnotesize{	\bluetextit{Exploded view of a ladder}}
%			\vspace*{0.1cm}		
%											
%			\includegraphics[width=\textwidth]{TRACKER-Ladder-01bis.png}
%		\end{column}
%		\begin{column}{0.01\textwidth}\end{column}
%	\end{columns}
	
%	\vspace*{0.1cm}
%
%	\begin{columns}
%		\begin{column}{0.51\textwidth}	\end{column}
%		\begin{column}{0.48\textwidth}\centering			
%			\footnotesize{\bluetextbf{(}\bluetextit{Exploded view of a ladder}\bluetextbf{)}}
%		\end{column}
%		\begin{column}{0.01\textwidth}\end{column}
%	\end{columns}	
%\end{frame}

\begin{frame}
	\frametitle{Silicon Tracker}
	\framesubtitle{the structure of a ladder I}
	\justifying

	\begin{columns}
		\begin{column}{0.01\textwidth}\end{column}
		\begin{column}{0.52\textwidth}\justifying
			\small{A metallized UF (Upilex Film $50 \, \mu m$ thick), glued directly to the silicon sensors, serves as a routing 
			cable to bring the $n$-side signals to the $n$-side front end hybrid, which is located at the ladder end closest to 
			the magnet wall. 
			A second short UF join the $p$-side strips to their hybrid. 
			
			\vspace*{0.1cm}

			The \textit{flexibility of these UFs} allows the hybrids to be placed back-to-back, perpendicular to the detection plane, 
			thus \textbf{minimizing the material} in the sensitive region of the tracker and \textbf{maximizing the active area} 
			within the magnet bore. 
			
			\vspace*{0.1cm}

			Finally, an electromagnetic shield in the form of a doubly-metallized UF surrounds each ladder.}
		\end{column}
		\begin{column}{0.01\textwidth}\end{column}
		\begin{column}{0.45\textwidth}\centering
			\footnotesize{	\bluetextit{Exploded view of a ladder}}
			\vspace*{0.1cm}		
											
			\includegraphics[width=\textwidth]{TRACKER-Ladder-01bis.png}
			
			\justifying
			\scriptsize{\bluetextbf{Figure:} The figure shows the principal elements of the silicon ladder and the main 
			components of the readout hybrids.}
		\end{column}
		\begin{column}{0.01\textwidth}\end{column}
	\end{columns}
\end{frame}

\begin{frame}[label=TRK-Ladder1]
	\frametitle{Silicon Tracker}
	\framesubtitle{the structure of a ladder II}
	\justifying
	
	\begin{columns}
		\begin{column}{0.01\textwidth}\end{column}
		\begin{column}{0.52\textwidth}\justifying
			\small{The silicon sensors of each ladder are supported by a reinforcement of $5 \, mm$ thick Airex foam that is 
			glued to the $n$-side UF. 
			
			\vspace*{0.1cm}

			The exposed surface of the foam is covered with a $100 \, \mu m$ thick layer of carbon fibre.
%			Small $5 \, mm^3$ aluminium support feet are glued to the carbon fibre surface; the exact number depends on
%			the ladder length. 
			
			\vspace*{0.1cm}

			The feet contain screw fixation holes which are used to attach the ladder to its tracker plane.
			
			\vspace*{0.1cm}
			
			With the aim of \textbf{maximizing the layer acceptance} and \textbf{minimizing the number of readout channels}, 
			\textit{the front end electronics is placed on the ladder end for both sensor sides} 
			(the bonding schemes of the sensors are shown in the next slide).}
		\end{column}
		\begin{column}{0.01\textwidth}\end{column}
		\begin{column}{0.45\textwidth}\centering
			\footnotesize{	\bluetextit{Exploded view of a ladder}}
			\vspace*{0.1cm}		
											
			\includegraphics[width=\textwidth]{TRACKER-Ladder-01bis.png}
			
			\justifying
			\scriptsize{\bluetextbf{Figure:} The figure shows the principal elements of the silicon ladder and the main 
			components of the readout hybrids.}
		\end{column}
		\begin{column}{0.01\textwidth}\end{column}
	\end{columns}
\end{frame}

\begin{frame}[label=TRK-BondSchem]
	\frametitle{Silicon Tracker}
	\framesubtitle{bonding scheme of the readout strips}
	
	\vspace*{-0.1cm}
	
	\begin{center}
		\includegraphics[height=0.55\paperheight]{TRACKER-BondingScheme.png}
	\end{center}
	
	\begin{columns}
		\begin{column}{0.02\textwidth}\end{column}
		\begin{column}{0.96\textwidth}\justifying
			\scriptsize{\bluetextbf{Figures:} The two pictures show the bonding scheme of the readout strips. 
			In order to decrease the numbers of readout channels: on the $p$-side (top) only one strip, every two, is read;
			on the $n$-side (bottom) the even sensors are coupled each other, likewise the odd ones.}		
		\end{column}
		\begin{column}{0.02\textwidth}\end{column}
	\end{columns}
	
	\vspace*{0.2cm}
	\hfill\hyperlink{TRK-BondSchem-nside}{\textbf{\beamerbutton{more details on $n$-side bonding scheme}}}
\end{frame}

\begin{frame}[label=TRK-Fab00]
	\frametitle{Silicon Tracker}
	\framesubtitle{ladder fabrication
	\hspace*{6.2cm}
	\hyperlink{TRK-Fab01}{\textbf{\beamerbutton{assembly process of a ladder}}}}
	\justifying	
	
	\footnotesize{The alignment precision is provided by the mechanical precision of the jigs ($1 - 2 \, \mu m$) and the precision
	of the sensor cut.
	During fabrication, the sensor positions on a ladder are recorded with a 3D semi-automatic measuring machine
	(see the figure below).}
	
	\vspace*{0.2cm}
	
	\begin{columns}
		\begin{column}{0.28\textwidth}\end{column}
		\begin{column}{0.41\textwidth}
			\centering
			\includegraphics[width=\textwidth]{TRACKER-Lad_Probe_Station_Table.png}
		\end{column}
		%\begin{column}{0.01\textwidth}\end{column}
		\begin{column}{0.28\textwidth}
			\justifying
			\scriptsize{\bluetextbf{Figure:} An AMS-02 ladder under test on the probe station table. The probe station is also 
			equipped with micropositioners and a microscope.}	
		\end{column}
		\begin{column}{0.03\textwidth}\end{column}
	\end{columns}

%	\begin{columns}
%		\begin{column}{0.1\textwidth}\end{column}
%		\begin{column}{0.8\textwidth}
%			\centering
%			\includegraphics[height=0.32\paperheight]{TRACKER-Lad_Probe_Station_Table.png}
%			
%			\justifying
%			\scriptsize{\bluetextbf{Figure:} An AMS-02 ladder under test on the probe station table. The probe station is also 
%			equipped with micropositioners and a microscope.}	
%		\end{column}
%		\begin{column}{0.1\textwidth}\end{column}
%	\end{columns}
	
	\begin{block}{\footnotesize{Goals of the ladder fabrication}}
		\begin{columns}
			\begin{column}{0.01\textwidth}\end{column}
			\begin{column}{0.98\textwidth}\justifying
				\footnotesize{The principal goals of the ladder fabrication are to guarantee the required precision for the relative 
				alignment of the silicon sensors ($< 5 \, \mu m$), and minimize the degradation of the electrical performance due to 
				handling and ultra-sonic bonding. }
			\end{column}
			\begin{column}{0.01\textwidth}\end{column}
		\end{columns}
	\end{block}	
\end{frame}

\begin{frame}
	\frametitle{Silicon Tracker}
	\framesubtitle{ladder: results for the sensor alignment}
	
	\vspace*{-0.1cm}

	\small{The results for the sensor alignment for 125 (of 192) AMS-02 ladders are shown in figure below. 
	A particular effort has been made to maintain the low noise by passivation of the silicon and by optimization of the ladder
	assembly procedure.}
	
	\vspace*{0.25cm}
	
	\begin{columns}
		\begin{column}{0.05\textwidth}\end{column}
		\begin{column}{0.9\textwidth}
			\centering
			\includegraphics[height=0.44\paperheight]{TRACKER-Ladder_AssemblyPrecision.png}
			
			\vspace*{0.05cm}
			
			\justifying
			\scriptsize{\bluetextbf{Figure:} Assembly precision of 125 AMS-02 ladders: distribution of the measured differences
			of the distance between adjacent sensors and the nominal distance, $640 \, \mu m$ including a $40 \, \mu m$ gap
			between sensors (left), and the residual distribution of the sensor positions about the line fits defining the ladder axis
			parallel to the magnetic field (right).}
		\end{column}
		\begin{column}{0.05\textwidth}\end{column}
	\end{columns}
\end{frame}


\begin{frame}
	\frametitle{Silicon Tracker}
	\framesubtitle{the structure of the AMS-02 tracker}
	\justifying

	\vspace*{-0.1cm}
	
	\small{The Silicon Tracker is composed by \textit{9 layers} arranged along AMS-02 detector height which allow the
	\textbf{3-D reconstruction} of incoming particle's trajectory.} 

	\vspace*{0.2cm}	
	
	\begin{columns}
		\begin{column}{0.15\textwidth}\end{column}
		\begin{column}{0.7\textwidth}
			\centering
			\includegraphics[height=0.45\paperheight]{TRACKER-02.png}
			
			\vspace*{0.1cm}
			
			\scriptsize{\bluetextbf{Figure:} 
			Each layer contains a variable number of different ladders.}
		\end{column}
		\begin{column}{0.15\textwidth}\end{column}
	\end{columns}
	
	\vspace*{0.2cm}
	
	\small{With a weigh of $186 \, kg$ and very low power consumption ($\sim 0.7 \, mW$ per channel),
	\textbf{the Silicon Tracker of the AMS-02 experiment is the largest tracker for a magnetic spectrometer ever built for 
	space applications}.}
%	\begin{columns}
%		\begin{column}{0.01\textwidth}\end{column}
%		\begin{column}{0.43\textwidth}\justifying
%		\end{column}
%		\begin{column}{0.55\textwidth}
%			\scriptsize{\bluetextbf{Figure:} 
%			Each layer contains a variable number of readout units (around 20), called ladders and each ladder, in turn, has
%			a variable number (from 7 to 15) of silicon sensors.}
%		\end{column}
%		\begin{column}{0.01\textwidth}\end{column}
%	\end{columns}
\end{frame}

\begin{frame}
	\frametitle{Silicon Tracker}
	\framesubtitle{the structure of the AMS-02 tracker}
	\justifying
	
	\vspace*{-0.3cm}

	\footnotesize{\textit{Layers 1 and 9 constitute the} \textbf{external tracker} and are positioned at both ends of the 
	detector to maximize the lever arm in the trajectory determination: the first is above the TRD at $z =155 \, cm$ and the 
	last is under the RICH (before the ECAL) at $z = -135 \, cm$.}

	\vspace*{0.05cm}
	
	\begin{columns}
		\begin{column}{0.01\textwidth}\end{column}
		\begin{column}{0.59\textwidth}\justifying
			\footnotesize{\textit{The layers from 2 to 8 constitute the} \textbf{inner tracker}\footnotemark[1]. 
			
			\vspace*{0.05cm}

			The layer 2 is arranged just above the magnet mechanical structure (at $z = 55 \, cm$), the other seven layers of the 
			inner tracker (from 3 to 8) are all located inside the magnet bore. 
			
			\vspace*{0.05cm}
			
			Three carbon fibre and aluminium honeycomb planes support the layer pairs (namely: 3-4, 5-6, 7-8) inside the magnet
			bore, with a relative distance between layers belonging to the same plane of about $4 \, cm$ along the $z$ axis. 
			The most inner plane (layers 5-6) is placed at the magnet center ($z = 0$) and the two neighbours symmetrically 
			disposed at $z = \pm 27 \, cm$ from the central one.}
		\end{column}
		\begin{column}{0.01\textwidth}\end{column}
		\begin{column}{0.38\textwidth}
			\vspace*{0.1cm}			
			
			\centering
			\includegraphics[width=0.8\textwidth]{TRACKER-03.png}
			
			\vspace*{0.2cm}

			\justifying
			\scriptsize{\bluetextbf{Figure:} measured $369 \, GeV$ $e^+$ event (bending plane view).}
		\end{column}
		\begin{column}{0.01\textwidth}\end{column}
	\end{columns}
	
%	\footnotetext[1]{\justifying{\scriptsize{There is no overlap between the ladders in the planes of the tracker because, 
%	even if this operation involve an improvement of the planes tightness, it has also the drawback of significantly increase the 
%	complication of the mechanical design of the support structure.}}}
	\footnotetext[1]{\justifying{\scriptsize{In view of the marginal increase of the planes tightness, and the very significant 
	complication of the mechanical design, there is no overlap between the ladders in the planes of the tracker.}}}
\end{frame}

\begin{frame}
	\frametitle{Silicon Tracker}
	\framesubtitle{need for a thermal control system for AMS}
	\justifying
	
	Due to the high number of readout channels, the tracker front end electronics (mainly the voltage amplifiers in the hybrids) 
	develop about $200 \, W$ of heat that must be removed from the volume of the inner tracker to prevent overheating of the 
	system: a maximum temperature of $80 \, ^\circ{C}$ can be tolerated by the electronics before a permanent damage, 
	however an optimal constant temperature of about $20 \, ^\circ{C}$ is desirable to keep uniform performance of the overall
	system. 
	
	Temperature inside the magnet bore is thus actively controlled by the \textbf{Tracker Thermal Control System} 
	(TTCS)\footnotemark[1].
	
	\footnotetext[1]{\justifying{The TTCS also provide for the compensation of the tracker temperature variation induced by the 
	variation of the incidence angle of sunlight on the ISS and, therefore, on the AMS-02 module.}}
%	The hybrids are mounted on Thermal Pyrolytic Graphite-Al cooling bars, which evacuate the heat
%generated by the front-end electronics inside the magnet.
	
	
%			\footnotesize{The layers from 2 to 8 constitute the \textbf{inner tracker}. The layer 2 is arranged just above the magnet 
%			mechanical structure (at $z = 55 \, cm$), the other seven layers of the inner tracker (from 3 to 8) are all located inside
%			the magnet bore. 
%			
%			Three composite structure (with two $220 \, \mu m$ thick layers of carbon fibre surrounding a $12 \, mm$ thick, low 
%			density aluminium honeycomb interior, $\rho = 16.02 \, kg/m^3$ - thve diameter of the planes is $1 \, m$) support the 
%			layer pairs inside the magnet bore (3-4, 5-6,7-8), with a relative distance between layers belonging to the same plane of 
%			$\sim 4 \, cm$ along the z axis. 
%			The most inner plane (layers 5-6) is placed at the magnet center ($z = 0$) and the two neighbours symmetrically 
%			disposed at $z = \pm 27 \, cm$ from the central one.

%As also depicted in
%fig.1 the Silicon Tracker is made of 9 layers covering the
%full span of the AMS detector. Three support planes each
%equipped with ladders on both sides are placed inside the
%magnet bore, another support plane holding a single layer
%of silicon detectors is placed at the top end of the magnet.
%Two additional single layer planes are placed on top of the
%TRD and the ECAL to get the largest lever arm in measuring
%the particle rigidity. Collectively layers from 2 to 8 are
%referred as the inner tracker.
\end{frame}

\begin{frame}
	\frametitle{Silicon Tracker}
	\framesubtitle{need for a thermal control system for AMS}
	
	
	\begin{columns}
		\begin{column}{0.01\textwidth}\end{column}
		\begin{column}{0.62\textwidth}\justifying
			\small{Due to the high number of readout channels, the tracker front end electronics 
			(mainly the voltage amplifiers in the hybrids) 
			develop about $200 \, W$ of heat that must be removed from the volume of the inner tracker to prevent overheating of
			the system: a maximum temperature of $80 \, ^\circ{C}$ can be tolerated by the electronics before a permanent 
			damage, however an optimal constant temperature of about $20 \, ^\circ{C}$ is desirable to keep uniform performance 
			of the overall system. 
			
			\vspace*{0.15cm}			
	
			Temperature inside the magnet bore is thus actively controlled by the:\\ 
			$\quad \triangleright$ \textbf{Tracker Thermal Control System} (TTCS)\footnotemark[1].}
		\end{column}
		\begin{column}{0.01\textwidth}\end{column}

		\begin{column}{0.35\textwidth}
			\centering
			\bluetextit{\footnotesize{Tracker thermal bars}}

			\vspace*{0.1cm}			
			
			\includegraphics[width=0.7\textwidth]{TRACKER-TTCS-Thermal_Bars-cut.png}
			
			\vspace*{0.05cm}
						
			\justifying
			\scriptsize{\bluetextbf{Figure:} The ladder hybrids are thermally connected to a system of thermal bars to keep the whole 
			tracker at the same temperature.}
		\end{column}
		\begin{column}{0.01\textwidth}\end{column}
	\end{columns}
	
	\footnotetext[1]{\justifying{\scriptsize{The TTCS also provide for the compensation of the tracker temperature variation 
	induced by the variation of the incidence angle of sunlight on the ISS and, therefore, on the AMS-02 module.}}}
\end{frame}

\begin{frame}
	\frametitle{Silicon Tracker}
	\framesubtitle{the Tracker Thermal Control System I}
	\justifying
	
	\begin{columns}
		\begin{column}{0.5\textwidth}\justifying			
			\small{\bluetextbf{\hspace{0.15cm} TTCS operating principle}}
			
			\vspace*{-0.15cm}

			\begin{itemize}\justifying
			\footnotesize{
				\item[$\blacktriangleright$]	A pump forces the liquid $\text{CO}_2$ into a loop passing trough the thermal bars, 
							working as evaporator. 
							In this evaporator the hybrids heat makes the $\text{CO}_2$ boiling and becoming two phases.
														
							\vspace*{-0.05cm}
							 
				\item[$\blacktriangleright$]	Exiting from the tracker  the two phases $\text{CO}_2$ is thermally connected to the 
							liquid entering the evaporator, bringing the second as close as possible to the boiling point. 
							
							\vspace*{-0.05cm}
							
				\item[$\blacktriangleright$] 	After this step the two phases $\text{CO}_2$ goes toward a two radiators system,
							facing the outer space, working as condenser.
			}
			\end{itemize}
		\end{column}
		\begin{column}{0.5\textwidth}
			\includegraphics[width=\textwidth]{TRACKER-TTCS_Op_Principle.jpg}
		\end{column}
	\end{columns}
\end{frame}

\begin{frame}
	\frametitle{Silicon Tracker}
	\framesubtitle{the Tracker Thermal Control System II}
	\justifying
	
	\begin{columns}
		\begin{column}{0.5\textwidth}\justifying
		
			\small{\bluetextbf{\hspace{0.15cm} TTCS operating principle}}
			
			\vspace*{-0.15cm}

			\begin{itemize}\justifying
			\footnotesize{
				\item[$\blacktriangleright$]	To adjust the temperature set point of the system, a reservoir regulated by a 
							thermo-electric (Peltier) element controls the amount of $\text{CO}_2$ in the loop.
														
							\vspace*{-0.05cm}
							 
				\item[$\blacktriangleright$]	The boiling of the liquid, and so the use of latent heat, makes the system more efficient 
							with respect to a single phase device.
							
							\vspace*{-0.05cm}
							
				\item[$\blacktriangleright$] 	The whole TTCS system (loop, pumps, accumulators, etc.) is completely doubled to have 
							fully redundancy.
			}
			\end{itemize}
			
			\begin{columns}
				\begin{column}{0.04\textwidth}\end{column}
				\begin{column}{0.96\textwidth}\justifying
					\footnotesize{\textbf{N.B.} The layer 1, facing almost directly the outer space, has no over-heating problem and
					doesn't need a cooling system.}
				\end{column}
			\end{columns}
			
		\end{column}
		\begin{column}{0.5\textwidth}
			\includegraphics[width=\textwidth]{TRACKER-TTCS_Op_Principle.jpg}
		\end{column}
	\end{columns}
\end{frame}

\begin{frame}
	\frametitle{Silicon Tracker}
	\framesubtitle{the performance of the TTCS}
	\justifying
	
	\small{The TTCS keeps the Inner Tracker temperature under control within $1 \, ^\circ{C}$ fluctuation of its 
	nominal operational temperature. 
	This guarantees the Tracker stability in terms of tracking efficiency and position measurement performance.}
	
	\vspace*{0.3cm}

%	\begin{columns}
%		\begin{column}{0.03\textwidth}\end{column}
%		\begin{column}{0.61\textwidth}
%			\includegraphics[width=\textwidth]{TRACKER-TempVar-03.png}			
%		\end{column}
%		\begin{column}{0.33\textwidth}
%			\scriptsize{\textitem{Figure}{Tracker Temperature}}
%			
%			\justifying
%			\scriptsize{Time variation of the tracker temperature measured\footnotemark[1] by different 
%			sensors in the period between May 2011 and September 2014. 
%			Inner Tracker is more stable since it is contained in the magnet bore while the two outer layers are more exposed
%			to the space environment.}
%		\end{column}
%		\begin{column}{0.04\textwidth}\end{column}
%	\end{columns}
%
%	\footnotetext[1]{\justifying{\scriptsize{The temperatures are measured at different locations of the detector, either close to 
%	the electronics, either in the carbon fibre support structures, by a chain of sensors.}}}
	
	\begin{columns}
		\begin{column}{0.05\textwidth}\end{column}
		\begin{column}{0.9\textwidth}
			\centering
			\includegraphics[height=0.45\paperheight]{TRACKER-TempVar-01.png}			

			\vspace*{0.1cm}
			
			\justifying
			\scriptsize{\bluetextbf{Figure:} Time variation of the tracker temperature measured, between May 2011 and September 
			2014, by different sensors. 
			Inner Tracker is more stable since it is contained in the magnet bore while the two outer layers are more exposed
			to the space environment.}
		\end{column}
		\begin{column}{0.05\textwidth}\end{column}
	\end{columns}
\end{frame}

\begin{frame}
	\frametitle{Silicon Tracker}
	\framesubtitle{tracker displacements due to the thermal variations}
	\justifying
	\footnotesize{In order to maximize the Tracker resolution, it is necessary to know the position of each sensor with an accuracy 
	of few $\mu m$, but the variation of the mean temperature and gradients across the mechanical structure lead to shifts of 
	several hundred $\mu m$.}
	
	\vspace*{0.25cm}
	
	\begin{columns}
		\begin{column}{0.12\textwidth}\end{column}
		\begin{column}{0.62\textwidth}
			\includegraphics[width=\textwidth]{TRACKER-Thermal_Disp-02.png}			
		\end{column}
		\begin{column}{0.21\textwidth}
			\justifying
			\scriptsize{\bluetextbf{Figures:}\\Shift of external layers based on nominal position in the bending side ($y$) and 
			in the non-bending one ($x$).}
		\end{column}
		\begin{column}{0.04\textwidth}\end{column}
	\end{columns}
	
	\vspace*{0.25cm}
	
	\footnotesize{Indeed, due to the temperature variation on orbit, displacements of the outer Tracker layers were observed at a 
	time scale of tens of minutes. 
	A long-term movement of the sensors correlates with the evolution of the solar beta angle ($\sim 2$ months cycle), 
	while a shorter term variation is caused by the temperature cycle within a single orbit.}
\end{frame}

\begin{frame}
	\frametitle{Silicon Tracker}
	\framesubtitle{tracker displacements due to the thermal variations}
	\justifying
	
	\begin{columns}
		\begin{column}{0.01\textwidth}\end{column}
		\begin{column}{0.49\textwidth}
			\justifying
			
			\small{To take into account the thermal movements and correct for time dependencies, two independent dynamic 
			procedures of track alignment have been developed. 
			
			\vspace*{0.15cm}
			
			The alignment was based on the minimization of the proton and helium track residuals in the external planes. 
			
			\vspace*{0.15cm}
			
			The stability in the bending coordinate after the alignment is shown in figures, no time dependent structure is visible
			and a dispersion at $\sim 3 \, \mu m$ level, well below the spatial resolution of the sensors, is reached.}
		\end{column}
		\begin{column}{0.01\textwidth}\end{column}
		\begin{column}{0.5\textwidth}
			\centering
			\small{\bluetextit{Tracker displacements}}
			
			\vspace*{0.1cm}
			
			\includegraphics[width=0.95\textwidth]{TRACKER-Alignment.png}	
			
			\vspace*{0.1cm}
			
			\justifying
			\scriptsize{\bluetextbf{Figures:} $y$ coordinate stability after alignment for the layers 1 and 9.}		
		\end{column}
	\end{columns}
	
%	It is necessary to estimate alignment parameters to be applied during the off-line track reconstruction algorithm. 
%	First, a static alignment of all the silicon sensors is performed using cosmic 
%	protons to correct for the sensor shifts. Moreover, a dynamical alignment procedure is applied to the external layers: their 
%	supporting structures suffer in fact deformations and shifts due to the large temperature gradient which they are exposed to 
%	(up to $\pm 80 \, ^\circ{C}$ in a 2 month beta angle cycle). Using cosmic protons to provide a time-dependent correction, the 
%	outer plane alignment is known with a precision $\sim 3 \, \mu m$. 
%	In parallel to the alignment procedure, the 7 inner planes movements are monitored by the Tracker Alignment System (TAS), 
%	which consists of 5 laser beams produced by diodes installed on the layer 2 support structures.
%
%	Laser alignment will be performed coincident with data taking. This allows any possible changes in the tracker geometry, from 
%	rapid thermal deformations to long term drift, to be identified and corrected offline.
\end{frame}

\section{Others sub-system of AMS-02}
\begin{frame}{Others sub-system of the AMS-02 experimet}{}
	\begin{itemize}
		\item	\textitem{ACC}{Anti-Coincidence Counter}, rejects CRs traversing the magnet walls.
		\item	\textitem{TAS}{Tracker Alignment System}, checks the Tracker alignment stability.
		\item	\bluetextbf{Star Tracker and GPS}, defines the position and orientation of the AMS-02 experiment.
		\item	\bluetextbf{Electronics}, transform the signals detected by the various particle detectors into digital information to be analysed by computers.
	\end{itemize}
\end{frame}


\section{Bibliography}
\begin{frame}[allowframebreaks]
	\frametitle{Bibliography}
	\justifying
		
	\bluetextbf{General information on the AMS-02 experiment:}
	
	\begin{thebibliography}{99}
	\setbeamertemplate{bibliography item}[text]
	\scriptsize{
	\bibitem{AMSCollab}
	AMS Collaboration, ``\textit{AMS on ISS - Construction of a particle physics detector on the International Space Station}'' AMS 
	Internal Note.	
	
	\bibitem{AMS-website}
	AMS Collaboration, ``\textit{AMS-02 - The Alpha Magnetic Spectrometer Experiment}'' \url{http://www.ams02.org/}
	
	\bibitem{Duranti}
	M. Duranti, ``\textit{Measurement of the Atmospheric Muon Flux on Ground with the AMS-02 Detector}'' PhD thesis, University 
	of Perugia	(2011).
	
	\bibitem{Kounine} 
	A. Kounine, ``\textit{THE ALPHA MAGNETIC SPECTROMETER ON THE INTERNATIONAL SPACE STATION}'' Int. J. Mod. Phys. E, 
	21, 1230005 (2012).
	
	\bibitem{Ting} 
	S. Ting, ``\textit{The Alpha Magnetic Spectrometer on the International Space Station}'' Nuclear Physics B (Proc. Suppl.) 
	243-244 (2013) 12–24.
		
	\bibitem{Vagelli}
	V. Vagelli, ``\textit{Measurement of the cosmic $e^+ + e^-$ Flux from 0.5 GeV to 1 TeV with the Alpha Magnetic 
	Spectrometer (AMS-02) on the International Space Station}'' PhD thesis, Karlsruher Institut f\"{u}r Technologie (2014).
	}

	\framebreak
	\hspace{-0.7cm}\normalsize{\bluetextbf{On the Transition Radiation Detector (TRD):}}		
	
	\scriptsize{
	\bibitem{Kirn-1}
	T. Kirn, ``\textit{The AMS-02 transition radiation detector}'' Nuclear Instruments and Methods in Physics Research A 581 
	(2007) 156-159.
	
	\bibitem{Kirn-2}
	T. Kirn, ``\textit{The AMS-02 TRD on the international space station}'' Nuclear Instruments and Methods in Physics Research A 
	706 (2013) 43-47.
	
	\bibitem{Sun}
	W. Sun, A. Kounine and Z. Weng, ``\textit{Measurement of the absolute charge of cosmic ray nuclei with the AMS Transition 
	Radiation Detector}'' 33rd International Cosmic Ray Conference, Rio de Janeiro (2013).
	}
	
	\vspace{0.25cm}
	\hspace{-0.7cm}\normalsize{\bluetextbf{On the Time of Flight detector (ToF):}}
	
	\scriptsize{
	\bibitem{Bindi-1}
	V. Bindi et al., ``\textit{The scintillator detector for the fast trigger and time-of-flight (TOF) measurement of the space 
	experiment AMS-02}'' Nuclear Instruments and Methods in Physics Research A 623 (2010) 968-981.
	
	\bibitem{Bindi-2}
	V. Bindi et al., ``\textit{The AMS-02 time of flight (TOF) system: construction and overall performances in space}'' 33rd 
	International Cosmic Ray Conference, Rio de Janeiro (2013).
	
	\bibitem{Bindi-3}
	V. Bindi et al., ``\textit{Calibration and performance of the AMS-02 time of flight detector in space}'' Nuclear Instruments and 
	Methods in Physics Research A 743 (2014) 22-29.
	}
	
	\framebreak
	\hspace{-0.7cm}\normalsize{\bluetextbf{On the Magnetic Spectrometer}}		
	
	\vspace{0.25cm}
	\hspace{-0.7cm}\small{\bluetextbf{$\triangleright$ Magnet:}}
	
	\scriptsize{
	\bibitem{Ahlen}
	S. Ahlen et al., ``\textit{An antimatter spectrometer in space}'' Nuclear Instruments and Methods in Physics Research A 350 
	(1994) 351-367.
	
	\bibitem{Lubelsmeyer}
	K. L\"{u}belsmeyer et al., ``\textit{Upgrade of the Alpha Magnetic Spectrometer (AMS-02) for long term operation on the 
	International Space Station (ISS)}'' Nuclear Instruments and Methods in Physics Research A 654 (2011) 639-648.
	}
	
	\vspace{0.25cm}
	\hspace{-0.7cm}\small{\bluetextbf{$\triangleright$ Silicon Tracker:}}
	
	\scriptsize{
	\bibitem{Alcaraz}
	J. Alcaraz et al., ``\textit{The alpha magnetic spectrometer silicon tracker: Performance results with protons and helium
	nuclei}'' Nuclear Instruments and Methods in Physics Research A 593 (2008) 376-398.

	\bibitem{Ambrosi-01}
	G. Ambrosi et al., ``\textit{AMS-02 Track reconstruction and rigidity measurement}'' 33rd International Cosmic Ray
	Conference, Rio de Janeiro (2013).
	
	\bibitem{Ambrosi-02}
	G. Ambrosi et al., ``\textit{Nuclear Charge Measurement With the AMS-02 Silicon Tracker}'' 33rd International Cosmic Ray
	Conference, Rio de Janeiro (2013).

	\bibitem{Ambrosi-03}
	G. Ambrosi et al., ``\textit{In-flight operations and status of the AMS-02 silicon tracker}'' PoS (ICRC2015) 690.
	
	\bibitem{Ambrosi-04}
	G. Ambrosi et al., ``\textit{Nuclei Charge measurement with the AMS-02 Silicon Tracker}'' PoS (ICRC2015) 429.	

	\bibitem{Azzarello}
	P. Azzarello, ``\textit{Tests And Production Of The AMS-02 Silicon Tracker Detectors}'' PhD thesis, University of Geneva 
	(2004).
	
	\bibitem{Batignani}
	G. Batignani et al., ``\textit{DOUBLE-SIDED READOUT SILICON STRIP DETECTORS FOR THE ALEPH MINIVERTEX}'' Nuclear 
	Instruments and Methods in Physics Research A 277 (1989) 147-153.

	\bibitem{Haas}
	D. Haas, ``\textit{The Silicon Tracker of AMS02}'' Nuclear Instruments and Methods in Physics Research A 530 (2004) 
	173-177.
	
	\bibitem{Krammer}
	M. Krammer, ``\textit{Silicon Detectors}'' XI ICFA School on Instrumentation (2010).
	
	\bibitem{Lutz}
	G. Lutz, ``\textit{Semiconductor Radiation Detectors}'' Springer 1st ed. (1999).
	
	\bibitem{Peisert}
	A. Peisert, ``\textit{Silicon microstrip detectors}'' Instrumentation in High Energy Physics (1992) 1-79.
	
	\bibitem{Pohl}
	M. Pohl, ``\textit{AMS tracking in-orbit performance}'' PoS (VERTEX2015) 022

	\bibitem{Zuccon}
	P. Zuccon, ``\textit{The AMS silicon tracker: Construction and performance}'' Nuclear Instruments and Methods in Physics 
	Research A 596 (2008) 74-78.
	}
	
	\end{thebibliography}
\end{frame}

\appendix
\setbeamertemplate{footline}
{
	\leavevmode%
	\hbox{%
  		% "ht" è lo spazio sopra il carattere partendo dalla base del carattere, "dp"  è lo spazio sotto il carattere partendo sempre dalla base del carattere
  		\begin{beamercolorbox}[wd=.333333\paperwidth,ht=2.5ex,dp=1ex,left,leftskip=4ex]{author in head/foot}%
   			\usebeamerfont{author in head/foot}\insertshortauthor
  		\end{beamercolorbox}%
  		\begin{beamercolorbox}[wd=.333333\paperwidth,ht=2.5ex,dp=1ex,center]{institute in head/foot}%
    		\usebeamerfont{title in head/foot}\insertshortinstitute
 		 \end{beamercolorbox}%
 		 \begin{beamercolorbox}[wd=.333333\paperwidth,ht=2.5ex,dp=1ex,right,rightskip=4ex]{date in head/foot}%
 		 	\usebeamerfont{date in head/foot}\insertshortdate{}\hspace*{0.25cm}(deepenings)
 		 \end{beamercolorbox}
 		 }%
 	\vskip0pt%
}


\section*{Appendix}\pdfbookmark[2]{Appendix}{Append}
\begin{frame}[noframenumbering]
	\vspace*{1cm}
	\hbox{
		\begin{beamercolorbox}[wd=.23\textwidth]{white}\end{beamercolorbox}%
  		\begin{beamercolorbox}
  		[wd=.54\textwidth,ht=.115\paperheight,dp=0.2cm,center,rounded=true]{institute in head/foot}%
   			\centering
   			\textcolor{titleframe_red}{\LARGE{\textbf{Appendix}}}
   			
   			\vspace*{0.05cm}
			\textcolor{titleframe_red}{\large{\textit{deepenings cited in the presentation}}}			
  		\end{beamercolorbox}%
 	}
\end{frame}

%% POSSIBILI SLIDE SULLA TERIA DELLA RADIAZIONE DI TRANSIZIONE
\begin{frame}<presentation:0>[label=TR-details,noframenumbering]
	\frametitle{TRD - Transition Radiation Detector}
	\framesubtitle{Transition Radiation - Angular Distribution\hspace{4.35cm}\hyperlink{TRD-main}{\beamerbutton{back to TRD}}}
	\justifying
	\footnotesize{
	The frequency spectrum of TR emitted by a particle with charge $Z e$ upon perpendicular traversal through a single interface 
	between two media with dielectric constants $\epsilon_f$ and $\epsilon_g$ has been calculated by Ginzburg and Frank and
	by Garibian. These calculations show that for highly relativistic particles most of the radiation is emitted in the x-ray region.}
	
	\vspace{-0.39cm}
	%\vspace{0.05cm}
	
	\begin{columns}
		\begin{column}{0.01\textwidth}\end{column}
		\begin{column}{0.51\textwidth}
			\justifying
		
			%\vspace{-0.4cm}
			\footnotesize{

			In this case $\epsilon_{f,g} \simeq 1 - \omega_{f,g}^2/\omega^2$ and the expression for the 
			\textbf{differential radiation intensity} is:
		
			\vspace{-0.15cm}
			$$ \frac{dW}{d(\hbar \omega) \: d\theta} = \frac{2 Z^2 \alpha}{\pi} f_0(\theta) $$				
		
			\vspace{0.1cm}
			
			where $\omega_{f,g}$ are the plasma frequencies of the two media, i.e.:
			 
			 \vspace{-0.4cm}
			$$ \omega_{f,g} = 28.8 \: \sqrt{\rho_{f,g} \, \frac{Z_{f,g}}{A_{f,g}}} \: eV $$
		
			with $\rho_{f,g}$ that are the densities of the two materials, $Z_{f,g}$ the atomic numbers and $A_{f,g}$ the 
			atomic weights.
			}
		\end{column}
		\begin{column}{0.48\textwidth}
			\centering
			\vspace{0.45cm}
			
			\hspace{0.25cm}\scriptsize{\bluetextit{Angular distribution of a single surface yield.}}	
			
			\vspace{0.05cm}			
			
			\begin{columns}
				\begin{column}{0.03\textwidth}\end{column}
				\begin{column}{0.9\textwidth}
					\centering
					\includegraphics[width=\textwidth]{TR-AngularDistr.png}		
				\end{column}
				\begin{column}{0.07\textwidth}\end{column}
			\end{columns}												
		\end{column}
		\begin{column}{0.01\textwidth}\end{column}
	\end{columns}
\end{frame}

\begin{frame}<presentation:0>[noframenumbering]
	\frametitle{TRD}
	\framesubtitle{Transition Radiation - Angular Distribution\hspace{4.35cm}\hyperlink{TRD-main}{\beamerbutton{back to TRD}}}
	\justifying
	
	\footnotesize{The distribution has a peak at the narrow angle $\theta \simeq \gamma^{-1}$ and extends to the angle of order
	$\sqrt{\gamma^{-2} + \omega_f^2 / \omega^2}$.
	Because the TR angle  $\theta \simeq \gamma^{-1}$ is very small, the practical application of the angle measurements is very 
	difficult.}
		
	\vspace{-0.5cm}
	\begin{columns}
		\begin{column}{0.01\textwidth}\end{column}
		\begin{column}{0.32\textwidth}
			\begin{figure}
				\centering
				\includegraphics[width=\textwidth]{TR-EnergySpecrum.png}
				
				%\scriptsize{\bluetextbf{Figure:} The radiated TR spectrum from a polyethylene surface}
			\end{figure}
		\end{column}
		\begin{column}{0.67\textwidth}
		\justifying
		
		\footnotesize{
		The integration of $dW/(\hbar d\omega \: d\theta)$ respect to $\theta$ yields the energy spectrum: 
		\begin{equation*} 
		\frac{dW}{d(\hbar \omega)} = \frac{Z^2 \alpha}{\pi} \left[ \frac{\omega_f^2 + \omega_g^2 + 2 \omega^2 \gamma^{-2}}{\omega_f^2-\omega_g^2}	\ln \left( \frac{\omega^2 \gamma^{-2} + \omega_f^2}{\omega^2 \gamma^{-2} + \omega_g^2} \right) - 2 \right]
	\end{equation*}
		and, integrating this spectrum in $\hbar d\omega$, the total radiation intensities is obtained, namely:
\begin{equation*}
		W \simeq Z^2 \cdot \frac{\hbar \alpha}{3} \cdot \frac{(\omega_f - \omega_g)^2}{\omega_f + \omega_g} \gamma \propto Z^2 \cdot \frac{E}{m c^2}
	\end{equation*}		
%	The differential energy spectrum can be broken up in essentially three regions having respectively, constant, logarithmic, and power-law dependence on $\omega$ and $\gamma$
%		\begin{equation*}
%		\frac{dW}{d\omega} \simeq 
%		\left\{
%		\begin{array}{ll}
%		\frac{2 Z^2 \alpha \hbar}{\pi}  \left( \ln \frac{\omega_f}{\omega_g} - 1 \right) \quad \quad \omega < \gamma \omega_g
%		\vspace{0.15cm}\\
%		\frac{2 Z^2 \alpha \hbar}{\pi}  \ln \frac{\omega_f}{\omega} \quad \quad \quad \quad \quad \gamma \omega_g < \omega < \gamma \omega_f \vspace{0.15cm}\\
%		\frac{Z^2 \alpha \hbar}{6 \pi}  \left( \frac{\omega_f}{\omega} \right)^4 \quad \quad \quad \quad \quad \gamma \omega_f < \omega
%		\end{array}
%		\right.
%	\end{equation*}
		}
		\end{column}
		\begin{column}{0.01\textwidth}\end{column}
	\end{columns}
		
		
\end{frame}

\begin{frame}<presentation:0>[noframenumbering]
	\begin{equation*} 
		\frac{dW}{d(\hbar\omega)} = \frac{Z^2 \alpha}{\pi} \left[ \frac{\omega_f^2 + \omega_g^2 + 2 \omega^2 \gamma^{-2}}{\omega_f^2-\omega_g^2}	\ln \left( \frac{\omega^2 \gamma^{-2} + \omega_f^2}{\omega^2 \gamma^{-2} + \omega_g^2} \right) - 2 \right]
	\end{equation*}
	
	\begin{equation*}
		\frac{dW}{d\omega} \simeq 
		\left\{
		\begin{array}{ll}
		\frac{2 Z^2 \alpha \hbar}{\pi}  \left( \ln \frac{\omega_f}{\omega_g} - 1 \right) \quad \quad \omega < \gamma \omega_g
		\vspace{0.15cm}\\
		\frac{2 Z^2 \alpha \hbar}{\pi}  \ln \frac{\omega_f}{\omega} \quad \quad \quad \quad \quad \gamma \omega_g < \omega < \gamma \omega_f \vspace{0.15cm}\\
		\frac{Z^2 \alpha \hbar}{6 \pi}  \left( \frac{\omega_f}{\omega} \right)^4 \quad \quad \quad \quad \quad \gamma \omega_f < \omega
		\end{array}
		\right.
	\end{equation*}
	
	\begin{equation*}
		W_{TR} \simeq Z^2 \cdot \frac{\hbar \alpha}{3} \cdot \frac{(\omega_f - \omega_g)^2}{\omega_f + \omega_g} \gamma \propto Z^2 \cdot \frac{E}{m c^2}
	\end{equation*}
	\begin{equation*}
		\theta_{TR} \lesssim \gamma^{-1}
	\end{equation*}
	\begin{equation*}
		D_{f} \simeq \frac{2 c}{\omega \left( \gamma^{-2} + \theta_{TR}^2 + \xi^2 \right)}
	\end{equation*}
\end{frame}



\begin{frame}[label=TRD-Module,noframenumbering]
	\frametitle{TRD - Transition Radiation Detector}
	\framesubtitle{photo of a radiator and a straw module}%\hspace{4.125cm}\hyperlink{TRD-Design}{\beamerbutton{back to TRD design}}}
	\begin{center}
		\vspace{-0.1cm}
		\includegraphics[height=0.65\paperheight]{TRD-RadEStraws-02mid.png}
	\end{center}
	
	\vspace{-0.1cm}\hfill\hyperlink{TRD-Design}{\beamerbutton{back to TRD design}}
\end{frame}



\begin{frame}[label=TRD-Mech_Struct,noframenumbering]
	\frametitle{TRD - Transition Radiation Detector}
	\framesubtitle{mechanical structure \hspace{6.85cm}\hyperlink{TRD-Supp_Struct}{\beamerbutton{back to TRD design}}}
	\begin{center}
		\vspace{-0.32cm}
		\includegraphics[height=0.775\paperheight]{TRD-Struct-04.png}
	\end{center}
\end{frame}



\begin{frame}[label=TRD-SpaceQual,noframenumbering]
	\frametitle{TRD - Transition Radiation Detector}
	\framesubtitle{more details on the space qualification of the straw modules}
	\justifying
	
	\vspace{-0.25cm}
	\scriptsize{Space qualification tests include vibration tests to simulate a space shuttle launch and thermal-vacuum tests, as 
	well as electromagnetic interference and corona tests. Gasgain and gastightness of the straw modules has 
	been	monitored during these tests.}
	
	\begin{itemize}
		\justifying
		\scriptsize{
		\item	For the \bluetextbf{vibration test}, a test jigg with 4 modules with a maximum length of $659 \, mm$ was used. 
					The test was performed on a vibration table, test cycles include a sine sweep $(0.5 \, g)$ from 10 to $2000 \, Hz$ 
					to	determine the first eigenfrequency of the structure, a random spectrum with a mean acceleration of $6.8 \, g$ to 
					simulate launch or landing with a space shuttle, and a second sine sweep to verify the first eigenfrequency has not 
					changed after performing the random  spectrum. 
					The test results show no such change. The measured first eigenfrequency of a module (length: $40 \, cm$) is 
					$f_0 = 128.5 \, Hz$. 
					A Comparison of this result with a FEC\footnotemark[1] coupled load modal analysis 	show good agreement.
		\item	For the \bluetextbf{thermal-vacuum test}, the same 4-layer test jigg has been inserted into a thermal-vacuum 
					tank at the MPE\footnotemark[2].
					During 70 hours, 9 thermocycles have been performed, with temperatures ranging from $-40 \: ^\circ{C}$ to
					$+60 \: ^\circ{C}$. 
					The test jigg was kept in vacuum at $10^{-6} mbar$. After thermal cycling, the 4 modules of the jigg were tested
					again 	for gastightness and gasgain, with no significant changes in both values.
		}
	\end{itemize}

	\hfill\hyperlink{TRD-SQ_StrawsMod}{\textbf{\beamerbutton{back to the previous slide}}}
	\footnotetext[1]{\scriptsize{Finite Elements Calculation}}
	\footnotetext[2]{\scriptsize{Max-Planck-Institut fur Extraterrestrische Physik, Munchen, Germany}}
\end{frame}


\begin{frame}[label=TRD-DeltaTech1,noframenumbering]
	\frametitle{TRD - Transition Radiation Detector}
	\framesubtitle{delta rays technique - 1}%\hspace{2.8cm}
%			\hyperlink{TRD-ChargePerform1}{\textbf{\beamerbutton{back to TRD performance}}}
%			\hspace{0.1cm}\hyperlink{TRD-DeltaTech2}{\textbf{\beamerbutton{more about this technique}}}
%			}
	\justifying
	\vspace{-0.15cm}
	\small{When charged particles pass through the TRD, the produced $keV$ scale delta rays travel almost perpendicular to the particle velocity, and some of them can be detected by the drift tubes not passed by the ions but in the vicinity (a few $cm$) of the particle track. 
	These tubes are tagged as \textit{delta ray tubes}, whereas the tubes passed by ions are defined as $dE/dx$ \textit{tubes}. 
	The signals in \textit{delta ray tubes} are typically far below the ADC saturation threshold. Particles with higher $Z$ corresponds to more delta rays, hence a higher ADC signal in \textit{delta ray tubes}.}

	\vspace{-0.35cm}
	\begin{columns}
		\begin{column}{0.05\textwidth}\end{column}
		\begin{column}{0.55\textwidth}
			\begin{figure}
				\centering
				\includegraphics[width=0.95\textwidth]{TRD-ChargeMeas-DeltaAmpl.png}
				
				%\footnotesize{\caption{\justifying AAA}}
			\end{figure}
		\end{column}
		\begin{column}{0.35\textwidth}\justifying
			
			\vspace{0.1cm}
			\footnotesize{\bluetextbf{Delta ray amplitudes}}
			
			\footnotesize{Figure shows the average signal in delta ray tubes as a function of charge measured by the Silicon 
			Tracker. Clear differences of signals between different ion species can be seen.}
		\end{column}
		\begin{column}{0.05\textwidth}\end{column}
	\end{columns}

%	\begin{columns}
%		\begin{column}{0.05\textwidth}\end{column}
%		\begin{column}{0.9\textwidth}
%			\begin{figure}
%				\centering				
%				\includegraphics[height=0.3\paperheight]{TRD-ChargeMeas-DeltaAmpl.png}
%			\end{figure}
%		\end{column}
%		\begin{column}{0.05\textwidth}\end{column}
%	\end{columns}	
%	
%	\vspace{0.15cm}
%	\begin{columns}
%		\begin{column}{0.05\textwidth}\end{column}
%		\begin{column}{0.9\textwidth}
%			\justifying
%			\footnotesize{Figure shows the average signal in delta ray tubes as a function of charge measured by the Silicon 
%			Tracker. Clear differences of signals between different ion species can be seen.}
%		\end{column}
%		\begin{column}{0.05\textwidth}\end{column}
%	\end{columns}
%	\vspace{0.15cm}
%	\hfill\hyperlink{TRD-ChargePerform1}{\textbf{\beamerbutton{back to TRD performance}}}
%	\hspace{1ex}\hyperlink{TRD-DeltaTech2}{\textbf{\beamerbutton{more about this technique}}}

	\hfill
	\hyperlink{TRD-ChargePerform1}{\textbf{\beamerbutton{back to TRD performance}}}
	\hspace{-0.1cm}
	\hyperlink{TRD-DeltaTech2}{\textbf{\beamerbutton{more about this technique}}}
\end{frame}

\begin{frame}[label=TRD-DeltaTech2,noframenumbering]
	\frametitle{TRD - Transition Radiation Detector}
	\framesubtitle{delta rays technique - 2}%\hspace{2.8cm}
%			\hyperlink{TRD-DeltaTech1}{\textbf{\beamerbutton{back to the previous slide}}}
%			\hspace{0.1cm}\hyperlink{TRD-ChargePerform1}{\textbf{\beamerbutton{return to TRD performance}}}
%			}
	\justifying
	\small{The dependences of \textit{delta ray tube} amplitude on layer and inclination are studied and no obvious dependence is 
	found. However, the \textit{delta ray tube} amplitude does change with increasing energy of the incoming particle.}

	\vspace{-0.2cm}
	\begin{columns}
		\begin{column}{0.02\textwidth}\end{column}
		\begin{column}{0.5\textwidth}
			\begin{figure}
				\centering
				\includegraphics[width=\textwidth]{TRD-ChargeMeas-DeltaSat.png}
				
				%\footnotesize{\caption{\justifying AAA}}
			\end{figure}
		\end{column}
		%\begin{column}{0.0\textwidth}\end{column}
		\begin{column}{0.46\textwidth}\justifying
			
			\vspace{0.1cm}
			\footnotesize{\bluetextbf{Delta ray amplitudes}}
			
			\footnotesize{Plot of the amplitude as a function of rigidity for Z = 6 particles, from which we observed the number of 
			delta rays increases with rigidity for rigidity below $25 GV$, and is almost constant in higher rigidity ranges.}
		\end{column}
		\begin{column}{0.02\textwidth}\end{column}
	\end{columns}
	
	\vspace{0.2cm}
	\small{Similar to $dE/dx$ \textit{PDFs}, we get \textit{Delta Ray PDFs} by parametrization on \textit{delta ray tube} 
	amplitude distributions.}
	
%	\vspace{0.3cm}
%	\hfill\hyperlink{TRD-DeltaTech2}{\textbf{\beamerbutton{back to the previous slide}}}
%	\hspace{1ex}\hyperlink{TRD-DeltaTech2}{\textbf{\beamerbutton{return to TRD performance}}}
	\hfill
	\hyperlink{TRD-DeltaTech1}{\textbf{\beamerbutton{back to the previous slide}}}
	\hspace{-0.1cm}
	\hyperlink{TRD-ChargePerform1}{\textbf{\beamerbutton{return to TRD performance}}}
\end{frame}

\begin{frame}[label=ToF-CADpannels,noframenumbering]
	\frametitle{ToF - Time of Flight}
	\framesubtitle{scintillator counters and light guides CAD design for UToF and LToF}
	
	%\vspace*{-0.25cm}
	\begin{columns}
	\begin{column}{0.08\textwidth}\end{column}
	\begin{column}{0.4\textwidth}
		\centering
		\includegraphics[height=0.58\paperheight]{ToF-PadsScheme_UToF.png}
	\end{column}
	\begin{column}{0.04\textwidth}\end{column}
	\begin{column}{0.4\textwidth}
		\centering
		\includegraphics[height=0.58\paperheight]{ToF-PadsScheme_LToF.png}
	\end{column}
	\begin{column}{0.08\textwidth}\end{column}
	\end{columns}
	\begin{columns}
	\begin{column}{0.11\textwidth}\end{column}
	\begin{column}{0.34\textwidth}
		\includegraphics[width=\columnwidth]{ToF-PadsScheme_UToF-side.png}
	\end{column}
	%\begin{column}{0.01\textwidth}\end{column}
	\begin{column}{0.51\textwidth}
		\justifying
		
		\vspace*{-0.25cm}\hspace*{0.35cm}\scriptsize{\bluetextbf{Figure:}}
		\begin{itemize}\justifying
			\tiny{
			\item[$\triangleright$]	Top view of the scintillator counters and light guides CAD design for the UToF (top left) and LToF 
												(right).
			\item[$\triangleright$]	Side view of layer 1 (bottom left).
			}
		\end{itemize}
						
	\end{column}
	\begin{column}{0.04\textwidth}\end{column}
	\end{columns}
	
	\vspace*{0.1cm}\hfill
	\hyperlink{ToF-Paddles}{\textbf{\beamerbutton{back to the previous slide}}}
	\hspace{-0.1cm}
	\hyperlink{ToF-LISTpannels}{\textbf{\beamerbutton{list of all counter types}}}
\end{frame}

\begin{frame}[label=ToF-LISTpannels,noframenumbering]
	\frametitle{ToF - Time of Flight}
	\framesubtitle{counter types}
	\justifying
	\vspace*{-0.15cm}
	
	\scriptsize{List of all counter types with light guide shape, counter shape, dimensions, number of PMTs per side and efficiency 
	of the light guides computed with the simulation\footnotemark[1].}
	\vspace*{-0.3cm}
	
	\begin{center}
		\includegraphics[height=0.505\paperheight]{ToF-ListOfCounters.png}
	\end{center}		
	\vspace*{-0.3cm}	
	
	\footnotetext[1]{\tiny{The light transmission efficiency of the 10 different light guide types was also simulated with a ray 
	tracing program taking into account the angular distribution at the light guide entrance and the angular acceptance on the 
	cathode glass window. For the trapezoidal counters, the simulation took into account also the counter geometry, therefore the 
	efficiency is much smaller than for the rectangular counters.}}
	
	\hfill	
	\hyperlink{ToF-CADpannels}{\textbf{\beamerbutton{back to the previous slide}}}
	\hspace{-0.1cm}
	\hyperlink{ToF-Paddles}{\textbf{\beamerbutton{return to the presentation}}}
\end{frame}

\begin{frame}[label=ToF-EJ200,noframenumbering]
	\frametitle{ToF - Time of Flight}
	\framesubtitle{property of EJ-200 Plastic Scintillator}
	\justifying
	\vspace{-0.1cm}
	
	\scriptsize{EJ-200 combines the two important properties of long optical attenuation length and fast timing which make it particularly useful for time-of-flight systems using scintillators greater than one meter long. 
	%It is the detector of choice for many industrial applications, such as gauging and environmental protection, where high sensitivity and signal uniformity are critical operating requirements.
	}%\vspace{-0.2cm}
	
	\begin{columns}
		\begin{column}{0.02\textwidth}\end{column}
		\begin{column}{0.39\textwidth}
			\includegraphics[width=\textwidth]{ToF-EJ200spectrum.png}
			
			\vspace*{0.1cm}
			\includegraphics[width=\textwidth]{ToF-EJ200resp03.png}
		\end{column}
		\begin{column}{0.47\textwidth}
			\vspace{-0.2cm}
			\includegraphics[width=\textwidth]{ToF-EJ200prop.png}
			
			\vspace*{0.1cm}
			\includegraphics[width=\textwidth]{ToF-EJ200chem.png}
		\end{column}
		\begin{column}{0.02\textwidth}\end{column}
	\end{columns}	
	
	\vspace{-0.3cm}
	\hfill
	\hyperlink{ToF-Paddles}{\textbf{\beamerbutton{return to the presentation}}}
\end{frame}
	
\begin{frame}[label=ToF-R5946,noframenumbering]
	\frametitle{ToF - Time of Flight}
	\framesubtitle{property of Hamamatsu R5946 PMTs}
	\justifying
	
	\vspace*{-0.1cm}
	\scriptsize{Hamamatsu R5946 PMTs, using 16 fine mesh dynodes, are able to operate even in strong magnetic fields, up 
	to magnetic fields of $1 \, T$ and with a gain of $10^6$ in absence of magnetic field when it is powered at $2000 V$.}
	\vspace*{0.1cm}
%	\begin{columns}
%		\begin{column}{0.01\textwidth}\end{column}
%		\begin{column}{0.48\textwidth}\centering
%			\footnotesize{\bluetextbf{Typical Spectral Response}}
%		\end{column}
%		\begin{column}{0.02\textwidth}\end{column}
%		\begin{column}{0.48\textwidth}
%			\footnotesize{\bluetextbf{Typical Gain in Magnetic Fields}}		
%		\end{column}
%		\begin{column}{0.01\textwidth}\end{column}
%	\end{columns}

% 
	\begin{columns}
		\begin{column}{0.01\textwidth}\end{column}
		\begin{column}{0.32\textwidth}
			\centering
			\includegraphics[height=0.4\paperheight]{ToF-PMTspectrum.png}
		\end{column}
		\begin{column}{0.34\textwidth}
			\centering
			\includegraphics[height=0.4\paperheight]{ToF-PMTgain.png}
		\end{column}
		\begin{column}{0.01\textwidth}\end{column}
		\begin{column}{0.3\textwidth}
			\centering
			\includegraphics[height=0.25\paperheight]{ToF-PMTangle.png}
		\end{column}
		\begin{column}{0.02\textwidth}\end{column}
	\end{columns}	
	
	\vspace*{0.1cm}
	\scriptsize{As can be seen in the \textbf{first graph}, the spectral response ranges of R5946 is from 300 to 600 $nm$ with a 
	maximum response at 420 $nm$ (corresponding to a quantum efficiency of about 20\%).
	%, which matches the scintillator output.
	
	The \textbf{second and the third ones} show the change of the gain due to the magnetic field. 
	
	Specifically, in the third are drawn the measured (squares) and the simulated (points) relative gain as a 
	function of the angle between the PMT axis and the magnetic field, at two values of magnetic field: $1.5 \, kG$ (black) and 
	$3 \, kG$ (white).}
%	
%	\begin{columns}
%		\begin{column}{0.02\textwidth}\end{column}
%		\begin{column}{0.06\textwidth}
%		
%		\end{column}
%		\begin{column}{0.02\textwidth}\end{column}
%	\end{columns}
	 
	\vspace*{-0.15cm}\hfill	
	\hyperlink{ToF-MagneticField}{\textbf{\beamerbutton{return to the presentation}}}
\end{frame}

\begin{frame}[label=MAGNET-FieldDistr,noframenumbering]
	\frametitle{Magnet}
	
	\begin{columns}
		\begin{column}{0.01\textwidth}\end{column}
		\begin{column}{0.65\textwidth}\justifying
			\scriptsize{The magic ring is a Hollow Cylindrical Flux Source (HCFS). An HCFS is a cylindrical permanent magnet shell with 
			its magnetization vector constant in magnitude and oriented according to the formula:}
			
			\vspace*{-0.05cm}\centering$\alpha = 2\phi + \pi/2$

			\justifying\vspace*{0.15cm}
			\scriptsize{where $\phi$ is the angular cylindrical coordinate. Such a distribution gives an interior field of:}

			\vspace*{0.15cm}\centering$B = B_r In(r_2/r_1)$

			\vspace*{0.15cm}\justifying
			\scriptsize{where $B_r$ is the residual magnetic flux density of the ring and $r_1$ and $r_2$ are its inner and outer 			
			radius respectively.
			
			\vspace*{0.15cm}
	
			In practice, it is not possible to make it vary continuously, so its value is made to change abruptly by 
			$4 \pi / N$ between adjacent wedge-shaped sections of equal size and uniform magnetization, $N$ is the number of such 
			segments into which the ring is divided.
	
			\vspace*{0.15cm}			
			
			The field suffers surprisingly little from the approximation by segmentation. For example, if the ring is divided into 16 
			segments, it still produces a magnetic field of over 97\% of that produced by a continuous ring.}
	
			\vspace*{0.4cm}
			\hyperlink{MAGNET-Des}{\textbf{\beamerbutton{back to the previous slide}}}
		\end{column}
		\begin{column}{0.01\textwidth}\end{column}
		\begin{column}{0.32\textwidth}\centering
			\vspace*{-0.15cm}
			
			\includegraphics[width=0.85\textwidth]{MAGNET-FiledOr-02.png}
			
			\justifying
			\tiny{\bluetextbf{Figure:} Magnetic field distribution at a cross-section of the center of the magnet.}

			\vspace*{0.15cm}\centering
			\includegraphics[width=0.85\textwidth]{MAGNET-FiledOr-03.png}
			
			\justifying
			\tiny{\bluetextbf{Figure:} Field distribution at the cross-section of the center of a magic ring with 8 segments.}
		\end{column}
		\begin{column}{0.01\textwidth}\end{column}
	\end{columns}
\end{frame}

\begin{frame}[label=MAGNET-CompRes,noframenumbering]
	\frametitle{Magnet}
	\framesubtitle{comparison of the two rigidity resolutions}
	
	\vspace*{0.15cm}
	\begin{columns}
		\begin{column}{0.11\textwidth}\end{column}
		\begin{column}{0.47\textwidth}
			\includegraphics[width=\textwidth]{MAGNET-Comparition-05.png}
		\end{column}
		%\begin{column}{0.01\textwidth}\end{column}
		\begin{column}{0.35\textwidth}
			\justifying
			\scriptsize{\bluetextbf{Figure:}\\
			The rigidity resolution for the two AMS-02 magnetic designs. 
			The green line is the difference between the blue (PM) and red (SCM) resolutions. 
			At high energies the PM and SCM accuracies are equivalent; at low energies the difference in accuracy is only 10\%.}
			
		\end{column}
		\begin{column}{0.07\textwidth}\end{column}
	\end{columns}
	\vspace*{-0.15cm}

	\hfill
	\hyperlink{MAGNET-Choice}{\textbf{\beamerbutton{back to the previous slide}}}
\end{frame}

\begin{frame}[label=TRK-CCreadout,noframenumbering]
	\frametitle{Silicon Tracker}
	\framesubtitle{advantage of a capacitively coupled readout}
	\justifying
	
	\small{Capacitively coupled readout (right) has the obvious advantage of shielding the electronics from dark current, 
	which with direct coupling (left) can lead to pedestal shifts, a reduction of the dynamic range, and may
	even drive the electronics into saturation.}
	
	\vspace*{-0.15cm}
%
%	\begin{center}
%		\includegraphics[height=0.4\paperheight]{TRK-CapacitiveCoupling.png}		
%	\end{center}
%	
%	\vspace*{-0.25cm}
	
	\begin{columns}
		\begin{column}{0.1\textwidth}\end{column}
		\begin{column}{0.8\textwidth}\centering
			\includegraphics[height=0.4\paperheight]{TRK-CapacitiveCoupling.png}		
		\end{column}
		\begin{column}{0.1\textwidth}\end{column}
	\end{columns}
	
	\vspace*{0.1cm}
	
	\begin{columns}
		\begin{column}{0.05\textwidth}\end{column}
		\begin{column}{0.9\textwidth}
			\justifying
			\scriptsize{\textitem{Figure}{Direct and capacitive coupling of electronics to the detector}\bluetextbf{:}
			With direct coupling (left) the detector reverse bias current $I_r$ has to be absorbed by the electronics. 
			With capacitive coupling (right), only the AC part of the detector current reaches the electronics, while the
			DC part goes into a bias circuit, here shown as a simple resistor.}
		\end{column}
		\begin{column}{0.05\textwidth}\end{column}
	\end{columns}
	
	\vspace*{0.1cm}
	\hfill\hyperlink{TRK-DsidedDet}{\textbf{\beamerbutton{back to the presentation}}}
	\hspace{-0.1cm}
	\hyperlink{TRK-Integration}{\textbf{\beamerbutton{integrated detector}}}
\end{frame}

\begin{frame}[label=TRK-Integration,noframenumbering]
	\frametitle{Silicon Tracker}
	\framesubtitle{integrate elements into the detector}
	\justifying
	
	\small{As it is difficult to fabricate high ohmic resistors and almost impossible to produce sufficiently large capacitors on 
	LSI electronics (\textit{Large Scale Integration}), it seemed natural to integrate these elements into the detector.  
%	The detectors gave very satisfactory results. 
%	Detectors of this design have been used in the vertex detector of the DELPHI experiment at the electron-positron collider 
%	(LEP) at the European Center of Nuclear Research (CERN) in Geneva.
	}
	
	\vspace*{0.05cm}

	\begin{columns}
		\begin{column}{0.08\textwidth}\end{column}
		\begin{column}{0.4\textwidth}\centering
			\includegraphics[height=0.4\paperheight]{TRACKER-PitchStrips.png}
		\end{column}
		\begin{column}{0.04\textwidth}\centering
			\vspace*{0.3cm}
			
			$\Longrightarrow$		
		\end{column}
		\begin{column}{0.4\textwidth}\centering
			\vspace*{-0.2cm}
			\hspace*{0.2cm}\includegraphics[height=0.4\paperheight]{TRACKER-IntegratedCond-01.png}
		\end{column}
		\begin{column}{0.08\textwidth}\end{column}
	\end{columns}
	
	\vspace*{0.25cm}

	\small{Capacitances have been built by separating implantation and metallization of the strips by a thin oxide layer 
	(biassing resistors, if needed, were made in polysilicon - in the case of AMS's detectors the biassing method does 
	not use them).}
	
	\vspace*{0.15cm}
	\hfill\hyperlink{TRK-CCreadout}{\textbf{\beamerbutton{back to the previous slide}}}
	\hspace{-0.1cm}
	\hyperlink{TRK-DsidedExample}{\textbf{\beamerbutton{example of a strip detector}}}
\end{frame}

\begin{frame}[label=TRK-DsidedExample,noframenumbering]
	\frametitle{Silicon Tracker}
	\framesubtitle{example of a double-sided strip detector}
	\justifying
	
	\vspace*{-0.05cm}
	
	\small{Example of a double-sided strip detector with integrated biassing structures and 
	coupling capacitors. The strips on the two detector sides have orthogonal directions.}

	\vspace*{0.15cm}
	
	\begin{columns}
		\begin{column}{\textwidth}\centering
			\includegraphics[height=0.35\paperheight]{TRK-IntegratedStruct-p.png}
			\hspace*{1cm}
			\includegraphics[height=0.37\paperheight]{TRK-IntegratedStruct-n.png}
		\end{column}
	\end{columns}
	
	\vspace*{0.2cm}
	
	\begin{columns}
		\begin{column}{0.05\textwidth}\end{column}
		\begin{column}{0.9\textwidth}\justifying
			\scriptsize{\bluetextbf{Figures:} top view of the $p$ strip side (top left); cross-section along the middle of a $p$ strip 
			(bottom left); bottom view of the $n$ strip side (middle) and cross-section along the middle of an $n$ strip (right). 
			The symbol $C$ indicates the coupling capacitance built by strip implant, insulator and strip metal, while $R$ stands for 
			the biassing structure, which on the $p$ side is a punch-through structure and on the $n$ side is an 
			electron-accumulation layer resistor whose dimensions are defined by the enclosing $p$-type insulation structure.}
		\end{column}
		\begin{column}{0.05\textwidth}\end{column}
	\end{columns}

	\vspace*{0.15cm}
	\hfill\hyperlink{TRK-Integration}{\textbf{\beamerbutton{back to the previous slide}}}
	\hspace{-0.1cm}
	\hyperlink{TRK-DsidedDet}{\textbf{\beamerbutton{return to the presentation}}}
\end{frame}

\begin{frame}[label=TRK-nstrip,noframenumbering]
	\frametitle{Silicon Tracker}
	\framesubtitle{ohmic side of a AMS-02 microstrip sensor: close-up view}
	\justifying
	
	\begin{columns}
		\begin{column}{0.15\textwidth}\end{column}
		\begin{column}{0.85\textwidth}
				\includegraphics[height=0.65\paperheight]{TRACKER-nside.png}
		\end{column}
	\end{columns}
	
	\vspace*{0.55cm}
	\hfill\hyperlink{TRK-DsidedProblem}{\textbf{\beamerbutton{back to the presentation}}}
\end{frame}

\begin{frame}[label=TRK-PR_prob,noframenumbering]
	\frametitle{Silicon Tracker}
	\framesubtitle{more about position resolution}
	\justifying
	
	\vspace*{-0.1cm}

	\small{The strip pitch determines to a large extend the position resolution. It's clear that with small pitch a better resolution is 
	achievable, but there are several drawbacks:
	
	\vspace*{-0.25cm}
	
	\begin{itemize}\justifying{\small{
		\item	small strip pitch requires large number of electronic channels;
		
		\vspace*{-0.15cm}
		
		\item	cost increase;
		
		\vspace*{-0.15cm}

		\item	power dissipation increase.
	}}\end{itemize}

	\vspace*{-0.1cm}
	
	With an analogue readout, a possible solution is the implementation of intermediate strips. These are strips not connected to
	the readout electronics located between readout strips (for this reason, they are also called: \textit{floating strips}).

	\vspace*{0.1cm}

	The signal from floating strips is transferred by capacitive coupling (through the naturally present
	strip-strip capacitances) to the readout strips:\footnotemark[1]

	\vspace*{-0.25cm}
	
	\begin{itemize}\justifying{\small{
		\item	more hits with signals on more than one strip;

		\vspace*{-0.15cm}
		
		\item	improved resolution with smaller number of readout channels obtained by interpolation of the signals
								(charge center-of-gravity technique).
								}}
	\end{itemize}
	}
	
	\vspace*{-0.15cm}
	\hfill\hyperlink{TRK-CCandPR}{\textbf{\beamerbutton{back to the presentation}}}
	\hspace{-0.1cm}
	\hyperlink{TRK-CentOfGrav}{\textbf{\beamerbutton{center-of-gravity technique}}}
		
	\footnotetext[1]{\scriptsize{Important: the floating strips must held on the same voltage as the readout ones.}}
\end{frame}

%\begin{frame}[label=TRK-CentOfGrav,noframenumbering]
%	\frametitle{Silicon Tracker}
%	\framesubtitle{the charge center-of-gravity technique}
%
%	\vspace*{0.1cm}
%	
%	\begin{columns}
%		\begin{column}{0.03\textwidth}\end{column}
%		\begin{column}{0.4\textwidth}\centering
%			\includegraphics[width=\textwidth]{TRK-CapacitiveCoupling-03.png}
%		\end{column}
%		\begin{column}{0.01\textwidth}\end{column}
%		\begin{column}{0.53\textwidth}\justifying
%			\vspace*{1.2cm}
%
%			\scriptsize{\textitem{Figure}{example of the capacitive coupling}\\
%			The scheme shows a detector with two intermediate strips. 
%			Only every $3^{rd}$ strip is connected to an electronics channel. 
%			The charge from the intermediate strips is capacitive coupled to the neighbour strips.}
%		\end{column}
%		\begin{column}{0.03\textwidth}\end{column}
%	\end{columns}
%	
%	\begin{block}{\scriptsize{The charge center-of-gravity technique in the simplest case}}\justifying
%		\begin{columns}
%			\begin{column}{0.01\textwidth}\end{column}
%			\begin{column}{0.98\textwidth}\justifying
%			%	\scriptsize{Thank to the analogue readout and to the implementation of intermediate strips, it is possible to improve,  
%			%	by interpolation, the position measurements.}	
%				\scriptsize{Taking into account the signals collected by the two amplifiers ($h_1$ and $h_2$), it is possible to
%				evaluate the original position $x$, by computation of the \textit{charge center-of-gravity}:}
%				\vspace*{-0.15cm}
%	
%				\begin{equation*}
%					x 	= \frac{h_1 x_1 + h_2 x_2}{h_1 + h_2} 
%						= x_1 + \frac{h_2}{h_1 + h_2} (x_1 - x_2) 
%						= x_1 + \frac{h_2}{h_1 + h_2} p_{read}
%				\end{equation*}
%				\scriptsize{where $x_1$ and $x_2$ are the positions of $1^{st}$ and $2^{nd}$ readout strip and $p_{read}$ 
%				is their pitch. 
%	
%				The position resolution $\sigma_x$ is proportional to the ratio between the readout pitch and the $SNR$ 
%				(Signal to Noise Ratio), namely $\sigma_x \approx p_{read}/SNR$.}
%			\end{column}
%			\begin{column}{0.01\textwidth}\end{column}
%		\end{columns}
%	\end{block}
%\end{frame}
%
%\begin{frame}[label=TRK-CentOfGrav,noframenumbering]
%	\frametitle{Silicon Tracker}
%	\framesubtitle{position resolution as a function of the SNR}
%	
%	\begin{columns}
%		\begin{column}{0.12\textwidth}\end{column}
%		\begin{column}{0.78\textwidth}\justifying
%			\includegraphics[height=0.5\paperheight]{TRK-CapacitiveCoupling-05.png}
%		\end{column}
%	\end{columns}
%	
%	\vspace*{0.15cm}
%	
%	\begin{columns}
%		\begin{column}{0.13\textwidth}\end{column}
%		\begin{column}{0.74\textwidth}\justifying
%			\scriptsize{\bluetextbf{Figure:} Example of a detector with strip pith of $25 \, \mu m$ and analogue readout. 
%			The position resolution, $\sigma_x$, is plotted as a function of the $SNR$ in two different cases:}
%			
%			\vspace*{-0.15cm}
%
%			\begin{itemize}\scriptsize{\justifying
%				\item[$\rhd$]	\bluetextit{Top curve}: every 2nd strip is connected, one intermediate strip.
%	
%				\vspace*{-0.1cm}
%
%				\item[$\rhd$]	\bluetextit{Bottom curve}: every strip is connected to the readout electronics.
%			}\end{itemize}
%		\end{column}
%		\begin{column}{0.13\textwidth}\end{column}
%	\end{columns}
%\end{frame}
%
%\begin{frame}
%	\frametitle{Silicon Tracker}
%	\framesubtitle{capacitive coupling}
%	
%	\begin{columns}
%		\begin{column}{0.12\textwidth}\end{column}
%		\begin{column}{0.78\textwidth}\justifying
%			\includegraphics[height=0.5\paperheight]{TRK-CapacitiveCoupling-03.png}
%		\end{column}
%	\end{columns}
%	
%	\vspace*{0.25cm}
%	
%	\begin{columns}
%		\begin{column}{0.13\textwidth}\end{column}
%		\begin{column}{0.74\textwidth}\justifying
%			\scriptsize{\textitem{Figure}{example of the capacitive coupling}\\
%			The scheme shows an example of a detector with two intermediate strips. 
%			Only every $3^{rd}$ strip is connected to an electronics channel. 
%			The charge from the intermediate strips is capacitive coupled to the neighbour strips.}
%		\end{column}
%		\begin{column}{0.13\textwidth}\end{column}
%	\end{columns}
%\end{frame}

\begin{frame}[label=TRK-CentOfGrav,noframenumbering]
	\frametitle{Silicon Tracker}
	\framesubtitle{the charge center-of-gravity technique}
	
	\vspace*{-0.1cm}
	\begin{block}{\scriptsize{An example of the center-of-gravity technique}}\justifying
		\begin{columns}
			\begin{column}{0.01\textwidth}\end{column}
			\begin{column}{0.98\textwidth}\justifying
			%	\scriptsize{Thank to the analogue readout and to the implementation of intermediate strips, it is possible to improve,  
			%	by interpolation, the position measurements.}	
				\scriptsize{Taking into account the signals collected by the two amplifiers ($h_1$ and $h_2$), it is possible to
				evaluate the original position $x$, by computation of the \textit{charge center-of-gravity}:}
				\vspace*{-0.15cm}
	
				\begin{equation*}
					x 	= \frac{h_1 x_1 + h_2 x_2}{h_1 + h_2} 
						= x_1 + \frac{h_2}{h_1 + h_2} (x_1 - x_2) 
						= x_1 + \frac{h_2}{h_1 + h_2} p_{read}
				\end{equation*}
				\scriptsize{where $x_1$ and $x_2$ are the positions of $1^{st}$ and $2^{nd}$ readout strip and $p_{read}$ 
				is their pitch. 
	
				The position resolution $\sigma_x$ is proportional to the ratio between the readout pitch and the $SNR$ 
				(Signal to Noise Ratio), namely $\sigma_x \approx p_{read}/SNR$.}
			\end{column}
			\begin{column}{0.01\textwidth}\end{column}
		\end{columns}
	\end{block}
	
	\vspace*{-0.15cm}
	
	\begin{columns}
		\begin{column}{0.05\textwidth}\end{column}
		\begin{column}{0.35\textwidth}\centering
			\vspace*{0.2cm}
			\includegraphics[width=\textwidth]{TRK-CapacitiveCoupling-05.png}
		\end{column}
		%\begin{column}{0.01\textwidth}\end{column}
		\begin{column}{0.55\textwidth}\justifying
			\scriptsize{\bluetextbf{Figure:}
			Plot of $\sigma_x$ as a function of the $SNR$ for detector 
			with strip pith of $25 \, \mu m$ and analogue readout in two different cases:}
			
			\vspace*{-0.15cm}
			
			\begin{itemize}\scriptsize{\justifying{
				\item[$\rhd$]	\bluetextit{Top curve -} every 2nd strip is connected, one intermediate strip.
 			
 				\vspace*{-0.1cm}				
				
				\item[$\rhd$]	\bluetextit{Bottom curve -} every strip is connected to the readout electronics.
			}}\end{itemize}
			
%			\centering
%			\hyperlink{TRK-PR_prob}{\textbf{\beamerbutton{back to the previous slide}}}
%			\hspace{-0.1cm}
%	 		\hyperlink{TRK-CCandPR}{\textbf{\beamerbutton{return to the presentation}}}
		\end{column}
		\begin{column}{0.05\textwidth}\end{column}
	\end{columns}

	\vspace*{-0.3cm}
	\hfill
	\hyperlink{TRK-PR_prob}{\textbf{\beamerbutton{back to the previous slide}}}
	\hspace{-0.1cm}
 	\hyperlink{TRK-CCandPR}{\textbf{\beamerbutton{return to the presentation}}}
\end{frame}

%\begin{frame}[label=TRK-Upilex,noframenumbering]
%	\frametitle{Silicon Tracker}
%	\framesubtitle{upilex wrapping}
%	\vspace*{-0.45cm}
%
%	\begin{columns}
%		\begin{column}{0.02\textwidth}\end{column}
%		\begin{column}{0.46\textwidth}
%			\begin{figure}
%				\centering		
%				\small{\bluetextit{Airex spacer gluing}}
%				\vspace*{0.1cm}
%
%				\includegraphics[height=0.485\paperheight]{TRACKER-Ladder-06.png}
%			\end{figure}
%		\end{column}
%		\begin{column}{0.04\textwidth}\end{column}
%		\begin{column}{0.46\textwidth}
%			\begin{figure}
%				\centering
%				\small{\bluetextit{Upilex wrapping}}
%				\vspace*{0.1cm}
%				
%				\includegraphics[height=0.485\paperheight]{TRACKER-Ladder-05.png}
%			\end{figure}
%		\end{column}
%		\begin{column}{0.02\textwidth}\end{column}
%	\end{columns}
%	
%	\vspace*{-0.3cm}
%	
%	\begin{columns}[t]
%		\begin{column}{0.02\textwidth}\end{column}
%		\begin{column}{0.46\textwidth}
%			\begin{figure}\justifying
%				\scriptsize{\bluetextbf{Figure 1:}
%				Once that the glue (Araldite 2011) is dispensed on the spacer surface,
%				%in similar way as for the ladder reinforcement, 
%				the spacer is aligned with a plexiglas guide.
%				A plexiglas lid is then laid on the spacer surface.
%				}
%			\end{figure}
%		\end{column}
%		\begin{column}{0.04\textwidth}\end{column}
%		\begin{column}{0.46\textwidth}
%			\begin{figure}\justifying									
%				\scriptsize{\bluetextbf{Figure 2:} 
%				The Upilex foil, then the ladder are placed on a table. 
%				The foil is bent over the ladder with pushers.
%				The shielding is closed with rosin-free solder.}
%			\end{figure}
%		\end{column}
%		\begin{column}{0.02\textwidth}\end{column}
%	\end{columns}
%	
%	\vspace*{0.1cm}
%	\hfill
%	\hyperlink{TRK-Ladder1}{\textbf{\beamerbutton{back to the previous slide}}}
%\end{frame}

\begin{frame}[label=TRK-BondSchem-nside,noframenumbering]
	\frametitle{Silicon Tracker}
	\framesubtitle{bonding scheme of the readout strips
	\hspace*{3.8cm}
	\hyperlink{TRK-BondSchem}{\textbf{\beamerbutton{back to the previous slide}}}}
	
	\small{The two figures show the signal routing for a ladder with 12 sensors (in particular the K5 design - top), and for one with
	15 sensors (K7 design - bottom)\footnotemark[1].}
	
	\vspace*{0.2cm}

	\centering
	\includegraphics[height=0.25\paperheight]{TRACKER-BondingScheme-nside.png}
	
	\vspace*{0.2cm}

	\includegraphics[height=0.25\paperheight]{TRACKER-BondingScheme-nside02.png}

	\vspace*{0.3cm}

	\footnotetext[1]{\justifying{\scriptsize{The x coordinate is reconstructed with a multiplicity that can be solved only from the
	complete track knowledge.}}}
\end{frame}

\begin{frame}[label=TRK-Fab01,noframenumbering]
	\frametitle{Silicon Tracker}
	\framesubtitle{ladder fabrication  
	\hspace*{6.75cm}
	\hyperlink{TRK-Fab02}{\textbf{\beamerbutton{next}}}
	\textcolor{itemblue}{$\parallel$}
	\hyperlink{TRK-Fab00}{\textbf{\beamerbutton{presentation}}}}
	
	\vspace*{-0.45cm}

	\begin{columns}
		\begin{column}{0.02\textwidth}\end{column}
		\begin{column}{0.98\textwidth}
			\begin{figure}
				\includegraphics[height=0.52\paperheight]{TRACKER-LadFabrication-01.png}
				\hspace*{0.5cm}
				\includegraphics[height=0.52\paperheight]{TRACKER-LadFabrication-01b.jpg}
			\end{figure}
		\end{column}
		\begin{column}{0.02\textwidth}\end{column}
	\end{columns}
	
	\vspace*{-0.3cm}
	
	\begin{columns}[t]
		\begin{column}{0.02\textwidth}\end{column}
		\begin{column}{0.98\textwidth}
			\begin{figure}\justifying
				\scriptsize{\bluetextbf{Figures 1 and 2:}
				Sensor positioning. The detectors are longitudinally aligned with two pins, and transversely with one pin 
				(the pins have an geometrical accuracy of $5 \, \mu m$). 
				The mechanical precision required for the sensors alignment is $10 \, \mu m$ (actually, the precision reached for the 
				assembled ladders, is $6 \div 9 \, \mu m$).}
			\end{figure}
		\end{column}
		\begin{column}{0.02\textwidth}\end{column}
	\end{columns}
\end{frame}

\begin{frame}[label=TRK-Fab02,noframenumbering]
	\frametitle{Silicon Tracker}
	\framesubtitle{ladder fabrication  
	\hspace*{5.5cm}
	\hyperlink{TRK-Fab01}{\textbf{\beamerbutton{previous}}}
	\hspace{-0.1cm}
	\hyperlink{TRK-Fab03}{\textbf{\beamerbutton{next}}}
	\textcolor{itemblue}{$\parallel$}
	\hyperlink{TRK-Fab00}{\textbf{\beamerbutton{presentation}}}}
	
	\vspace*{-0.45cm}

	\begin{columns}
		\begin{column}{0.02\textwidth}\end{column}
		\begin{column}{0.46\textwidth}
			\begin{figure}
				\includegraphics[height=0.25\paperheight]{TRACKER-LadFabrication-02.png}
			\end{figure}
		\end{column}
		\begin{column}{0.04\textwidth}\end{column}
		\begin{column}{0.46\textwidth}
			\begin{figure}
				\includegraphics[height=0.25\paperheight]{TRACKER-LadFabrication-03.png}
			\end{figure}
		\end{column}
		\begin{column}{0.02\textwidth}\end{column}
	\end{columns}
	
	\vspace*{-0.3cm}
	
	\begin{columns}[t]
		\begin{column}{0.02\textwidth}\end{column}
		\begin{column}{0.46\textwidth}
			\begin{figure}\justifying
				\scriptsize{\bluetextbf{Figure 3:}
				Jigs used to glue the long Upilex.}
			\end{figure}
		\end{column}
		\begin{column}{0.04\textwidth}\end{column}
		\begin{column}{0.46\textwidth}
			\begin{figure}\justifying									
				\scriptsize{\bluetextbf{Figure 4:} 
				Upilex on jig.
				The glass plates used to flatten the Upilex are visible.}
			\end{figure}
		\end{column}
		\begin{column}{0.02\textwidth}\end{column}
	\end{columns}
	
	\begin{columns}
		\begin{column}{0.02\textwidth}\end{column}
		\begin{column}{0.46\textwidth}
			\begin{figure}
				\includegraphics[height=0.25\paperheight]{TRACKER-LadFabrication-04.png}
			\end{figure}
		\end{column}
		\begin{column}{0.04\textwidth}\end{column}
		\begin{column}{0.46\textwidth}
			\begin{figure}
				\includegraphics[height=0.25\paperheight]{TRACKER-LadFabrication-05.png}
			\end{figure}
		\end{column}
		\begin{column}{0.02\textwidth}\end{column}
	\end{columns}
	
	\vspace*{-0.3cm}
	
	\begin{columns}[t]
		\begin{column}{0.02\textwidth}\end{column}
		\begin{column}{0.46\textwidth}
			\begin{figure}\justifying
				\scriptsize{\bluetextbf{Figure 5:}
				Upilex ready for glue deposition.}
			\end{figure}
		\end{column}
		\begin{column}{0.04\textwidth}\end{column}
		\begin{column}{0.46\textwidth}
			\begin{figure}\justifying									
				\scriptsize{\bluetextbf{Figure 6:} 
				After glue deposition, the transfer jig is placed on the silicon jig.}
			\end{figure}
		\end{column}
		\begin{column}{0.02\textwidth}\end{column}
	\end{columns}
\end{frame}

\begin{frame}[label=TRK-Fab03,noframenumbering]
	\frametitle{Silicon Tracker}
	\framesubtitle{ladder fabrication  
	\hspace*{5.5cm}
	\hyperlink{TRK-Fab02}{\textbf{\beamerbutton{previous}}}
	\hspace{-0.1cm}
	\hyperlink{TRK-Fab04}{\textbf{\beamerbutton{next}}}
	\textcolor{itemblue}{$\parallel$}
	\hyperlink{TRK-Fab00}{\textbf{\beamerbutton{presentation}}}}
	
	\vspace*{-0.45cm}

	\begin{columns}
		\begin{column}{0.02\textwidth}\end{column}
		\begin{column}{0.46\textwidth}
			\begin{figure}
				\includegraphics[height=0.52\paperheight]{TRACKER-LadFabrication-07.png}
			\end{figure}
		\end{column}
		\begin{column}{0.04\textwidth}\end{column}
		\begin{column}{0.46\textwidth}
			\begin{figure}
				\includegraphics[height=0.52\paperheight]{TRACKER-LadFabrication-08.png}
			\end{figure}
		\end{column}
		\begin{column}{0.02\textwidth}\end{column}
	\end{columns}
	
	\vspace*{-0.3cm}
	
	\begin{columns}[t]
		\begin{column}{0.02\textwidth}\end{column}
		\begin{column}{0.46\textwidth}
			\begin{figure}\justifying
				\scriptsize{\bluetextbf{Figure 7:}
				The glue is spread with a spatula, the glue thickness will be close to $80 \, \mu m$.}
			\end{figure}
		\end{column}
		\begin{column}{0.04\textwidth}\end{column}
		\begin{column}{0.46\textwidth}
			\begin{figure}\justifying									
				\scriptsize{\bluetextbf{Figure 8:} 
				The Airex surface of the reinforcement is laid on the glue film.}
			\end{figure}
		\end{column}
		\begin{column}{0.02\textwidth}\end{column}
	\end{columns}
\end{frame}

\begin{frame}[label=TRK-Fab04,noframenumbering]
	\frametitle{Silicon Tracker}
	\framesubtitle{ladder fabrication  
	\hspace*{5.5cm}
	\hyperlink{TRK-Fab03}{\textbf{\beamerbutton{previous}}}
	\hspace{-0.1cm}
	\hyperlink{TRK-Fab05}{\textbf{\beamerbutton{next}}}
	\textcolor{itemblue}{$\parallel$}
	\hyperlink{TRK-Fab00}{\textbf{\beamerbutton{presentation}}}}
	
	\vspace*{-0.45cm}

	\begin{columns}
		\begin{column}{0.02\textwidth}\end{column}
		\begin{column}{0.46\textwidth}
			\begin{figure}
				\includegraphics[height=0.35\paperheight]{TRACKER-LadFabrication-09b.png}
			\end{figure}
		\end{column}
		\begin{column}{0.04\textwidth}\end{column}
		\begin{column}{0.46\textwidth}
			\begin{figure}
				\includegraphics[height=0.35\paperheight]{TRACKER-LadFabrication-10b.png}
			\end{figure}
		\end{column}
		\begin{column}{0.02\textwidth}\end{column}
	\end{columns}
	
	\vspace*{-0.3cm}
	
	\begin{columns}[t]
		\begin{column}{0.02\textwidth}\end{column}
		\begin{column}{0.46\textwidth}
			\begin{figure}\justifying
				\scriptsize{\bluetextbf{Figure 9:}
				Alignment bars are inserted, to guide the reinforcement application.}
			\end{figure}
		\end{column}
		\begin{column}{0.04\textwidth}\end{column}
		\begin{column}{0.46\textwidth}
			\begin{figure}\justifying									
				\scriptsize{\bluetextbf{Figure 10:} 
				The last bar is fixed with screws.}
			\end{figure}
		\end{column}
		\begin{column}{0.02\textwidth}\end{column}
	\end{columns}
	
	\begin{columns}
		\begin{column}{0.02\textwidth}\end{column}
		\begin{column}{0.96\textwidth}
			\begin{figure}
				\includegraphics[height=0.18\paperheight]{TRACKER-LadFabrication-11b.png}
			\end{figure}
		\end{column}
		\begin{column}{0.02\textwidth}\end{column}
	\end{columns}
	
	\vspace*{-0.3cm}
	
	\begin{columns}[t]
		\begin{column}{0.1\textwidth}\end{column}
		\begin{column}{0.8\textwidth}
			\begin{figure}\justifying									
				\scriptsize{\bluetextbf{Figure 11:} 
				Lead blocks on glass plates are placed on the reinforcement.}
			\end{figure}
		\end{column}
		\begin{column}{0.1\textwidth}\end{column}
	\end{columns}
\end{frame}

\begin{frame}[label=TRK-Fab05,noframenumbering]
	\frametitle{Silicon Tracker}
	\framesubtitle{ladder fabrication  
	\hspace*{5.5cm}
	\hyperlink{TRK-Fab04}{\textbf{\beamerbutton{previous}}}
	\hspace{-0.1cm}
	\hyperlink{TRK-Fab06}{\textbf{\beamerbutton{next}}}
	\textcolor{itemblue}{$\parallel$}
	\hyperlink{TRK-Fab00}{\textbf{\beamerbutton{presentation}}}}
	
	\vspace*{-0.4cm}
	
	\begin{columns}
		\begin{column}{0.02\textwidth}\end{column}
		\begin{column}{0.96\textwidth}
			\begin{figure}
				\includegraphics[height=0.18\paperheight]{TRACKER-LadFabrication-12.png}
			\end{figure}
		\end{column}
		\begin{column}{0.02\textwidth}\end{column}
	\end{columns}
	
	\vspace*{-0.3cm}
	
	\begin{columns}[t]
		\begin{column}{0.06\textwidth}\end{column}
		\begin{column}{0.88\textwidth}
			\begin{figure}\justifying									
				\scriptsize{\bluetextbf{Figure 12:} 
				Ladder on the K-side bonding jig. The ladder is aligned with help of pins and an alignment tool (on the left). 
				The ladder is fixed with clamps, and the hybrid is screwed on the extension (on the right).}
			\end{figure}
		\end{column}
		\begin{column}{0.06\textwidth}\end{column}
	\end{columns}
	
	\begin{columns}
		\begin{column}{0.02\textwidth}\end{column}
		\begin{column}{0.46\textwidth}
			\begin{figure}
				\includegraphics[height=0.25\paperheight]{TRACKER-LadFabrication-13.png}
			\end{figure}
		\end{column}
		\begin{column}{0.04\textwidth}\end{column}
		\begin{column}{0.46\textwidth}
			\begin{figure}
				\includegraphics[height=0.25\paperheight]{TRACKER-LadFabrication-14.png}
			\end{figure}
		\end{column}
		\begin{column}{0.02\textwidth}\end{column}
	\end{columns}
	
	\vspace*{-0.3cm}
	
	\begin{columns}[t]
		\begin{column}{0.02\textwidth}\end{column}
		\begin{column}{0.46\textwidth}
			\begin{figure}\justifying
				\scriptsize{\bluetextbf{Figure 13:}
				Blue tape is used to protect the bonding pads, in case the glue flows over them.}
			\end{figure}
		\end{column}
		\begin{column}{0.04\textwidth}\end{column}
		\begin{column}{0.46\textwidth}
			\begin{figure}\justifying									
				\scriptsize{\bluetextbf{Figure 14:} 
				Last step of K-hybrid gluing: the presser is screwed.}
			\end{figure}
		\end{column}
		\begin{column}{0.02\textwidth}\end{column}
	\end{columns}
\end{frame}

\begin{frame}[label=TRK-Fab06,noframenumbering]
	\frametitle{Silicon Tracker}
	\framesubtitle{ladder fabrication  
	\hspace*{5.5cm}
	\hyperlink{TRK-Fab05}{\textbf{\beamerbutton{previous}}}
	\hspace{-0.1cm}
	\hyperlink{TRK-Fab07}{\textbf{\beamerbutton{next}}}
	\textcolor{itemblue}{$\parallel$}
	\hyperlink{TRK-Fab00}{\textbf{\beamerbutton{presentation}}}}
	
	\vspace*{-0.45cm}

	\begin{columns}
		\begin{column}{0.02\textwidth}\end{column}
		\begin{column}{0.46\textwidth}
			\begin{figure}
				\includegraphics[height=0.25\paperheight]{TRACKER-LadFabrication-15.png}
			\end{figure}
		\end{column}
		\begin{column}{0.04\textwidth}\end{column}
		\begin{column}{0.46\textwidth}
			\begin{figure}
				\includegraphics[height=0.25\paperheight]{TRACKER-LadFabrication-16.png}
			\end{figure}
		\end{column}
		\begin{column}{0.02\textwidth}\end{column}
	\end{columns}
	
	\vspace*{-0.3cm}
	
	\begin{columns}[t]
		\begin{column}{0.02\textwidth}\end{column}
		\begin{column}{0.46\textwidth}
			\begin{figure}\justifying
				\scriptsize{\bluetextbf{Figure 15:}
				K6 Upilex with protection tape on the bonding pads.}
			\end{figure}
		\end{column}
		\begin{column}{0.04\textwidth}\end{column}
		\begin{column}{0.46\textwidth}
			\begin{figure}\justifying									
				\scriptsize{\bluetextbf{Figure 16:} 
				K6 Upilex on gluing jig. The metallized surface is facing down.}
			\end{figure}
		\end{column}
		\begin{column}{0.02\textwidth}\end{column}
	\end{columns}
	
	\begin{columns}
		\begin{column}{0.02\textwidth}\end{column}
		\begin{column}{0.46\textwidth}
			\begin{figure}
				\includegraphics[height=0.25\paperheight]{TRACKER-LadFabrication-17.png}
			\end{figure}
		\end{column}
		\begin{column}{0.04\textwidth}\end{column}
		\begin{column}{0.46\textwidth}
			\begin{figure}
				\includegraphics[height=0.25\paperheight]{TRACKER-LadFabrication-18.png}
			\end{figure}
		\end{column}
		\begin{column}{0.02\textwidth}\end{column}
	\end{columns}
	
	\vspace*{-0.3cm}
	
	\begin{columns}[t]
		\begin{column}{0.02\textwidth}\end{column}
		\begin{column}{0.46\textwidth}
			\begin{figure}\justifying
				\scriptsize{\bluetextbf{Figure 17:}
				Presser used to flatten hybrid.}
			\end{figure}
		\end{column}
		\begin{column}{0.04\textwidth}\end{column}
		\begin{column}{0.46\textwidth}
			\begin{figure}\justifying									
				\scriptsize{\bluetextbf{Figure 18:} 
				Gluing jig screwed on the S-extension.}
			\end{figure}
		\end{column}
		\begin{column}{0.02\textwidth}\end{column}
	\end{columns}
\end{frame}

\begin{frame}[label=TRK-Fab07,noframenumbering]
	\frametitle{Silicon Tracker}
	\framesubtitle{ladder fabrication  
	\hspace*{5.5cm}
	\hyperlink{TRK-Fab06}{\textbf{\beamerbutton{previous}}}
	\hspace{-0.1cm}
	\hyperlink{TRK-Fab08}{\textbf{\beamerbutton{next}}}
	\textcolor{itemblue}{$\parallel$}
	\hyperlink{TRK-Fab00}{\textbf{\beamerbutton{presentation}}}}
	
	\vspace*{-0.45cm}
	
	\begin{columns}
		\begin{column}{0.05\textwidth}\end{column}
		\begin{column}{0.9\textwidth}
			\begin{figure}
				\includegraphics[width=\textwidth]{TRACKER-LadFabrication-18-2.png}
			\end{figure}
		\end{column}
		\begin{column}{0.05\textwidth}\end{column}
	\end{columns}
%	
%	\vspace*{-0.4cm}
%	
%	\begin{columns}
%		\begin{column}{0.25\textwidth}\end{column}
%		\begin{column}{0.5\textwidth}
%			\begin{figure}
%				\includegraphics[height=0.23\paperheight]{TRACKER-LadFabrication-18bbb.png}
%				\hspace*{0.1cm}
%				\includegraphics[height=0.23\paperheight]{TRACKER-LadFabrication-18bb.png}
%			\end{figure}
%		\end{column}
%		\begin{column}{0.25\textwidth}\end{column}
%	\end{columns}
	
	\vspace*{-0.3cm}
	
	\begin{columns}[t]
		\begin{column}{0.02\textwidth}\end{column}
		\begin{column}{0.96\textwidth}
			\begin{figure}\justifying									
				\scriptsize{\bluetextbf{Figure 19:} 
				Bounding: electrically connection between the strip in a ladder with a $25 \, \mu m$ Al wire through ultrasonic 
				bondings (in particular, the photos shown the $p$-side of a ladder where the strip are daisy-chained)
				\footnotemark[1].}
			\end{figure}
		\end{column}
		\begin{column}{0.02\textwidth}\end{column}
	\end{columns}
	
	\footnotetext[1]{\justifying{\scriptsize{A ladder composed of $n$ sensors needs $(n + 1) \cdot 640$ bonds on the S-side 
	and $n\cdot 384$ bonds on the K-side, for a total of $1024 \cdot 1024 + 640 \simeq 10^6$ bonds. The need of automatic 
	bonding procedures is obvious.}}}
\end{frame}

\begin{frame}[label=TRK-Fab08,noframenumbering]
	\frametitle{Silicon Tracker}
	\framesubtitle{ladder fabrication  
	\hspace*{5.5cm}
	\hyperlink{TRK-Fab07}{\textbf{\beamerbutton{previous}}}
	\hspace{-0.1cm}
	\hyperlink{TRK-Fab09}{\textbf{\beamerbutton{next}}}
	\textcolor{itemblue}{$\parallel$}
	\hyperlink{TRK-Fab00}{\textbf{\beamerbutton{presentation}}}}
	
	%\vspace*{-0.45cm}

	\begin{columns}
		\begin{column}{0.02\textwidth}\end{column}
		\begin{column}{0.48\textwidth}
			\centering
			\includegraphics[height=0.25\paperheight]{TRACKER-LadFabrication-19.png}
			
			\vspace*{0.13cm}
			
			\justifying
			\scriptsize{\bluetextbf{Figure 20:}
			Airex spacer glued on S-hybrid. The left and right edges are machined to be compatible with the S-box.}
			
			\vspace*{0.4cm}

			\centering
			\includegraphics[height=0.25\paperheight]{TRACKER-LadFabrication-20.png}
			
			\vspace*{0.13cm}

			\justifying									
			\scriptsize{\bluetextbf{Figure 21:} 
			Airex spacer glued on the K-hybrid.}
		\end{column}
		\begin{column}{0.04\textwidth}\end{column}
		\begin{column}{0.44\textwidth}
			\centering
			\includegraphics[height=0.48\paperheight]{TRACKER-LadFabrication-21.png}
		
			\vspace*{0.15cm}

			\justifying
			\scriptsize{\bluetextbf{Figure 22:} 
			Thermal grease is dispensed between both hybrids, at the VAs location. 
			The aim is to exhaust the heat produced by the VAs to the thermal bars to which is connected
			the K-hybrid box.}
		\end{column}
		\begin{column}{0.02\textwidth}\end{column}
	\end{columns}
\end{frame}

	
\begin{frame}[label=TRK-Fab09,noframenumbering]
	\frametitle{Silicon Tracker}
	\framesubtitle{ladder fabrication  
	\hspace*{5.5cm}
	\hyperlink{TRK-Fab08}{\textbf{\beamerbutton{previous}}}
	\hspace{-0.1cm}
	\hyperlink{TRK-Fab10}{\textbf{\beamerbutton{next}}}
	\textcolor{itemblue}{$\parallel$}
	\hyperlink{TRK-Fab00}{\textbf{\beamerbutton{presentation}}}}
	
	\vspace*{-0.45cm}

	\begin{columns}
		\begin{column}{0.02\textwidth}\end{column}
		\begin{column}{0.96\textwidth}
			\begin{figure}
				\includegraphics[height=0.23\paperheight]{TRACKER-LadFabrication-22.png}
			\end{figure}
		\end{column}
		\begin{column}{0.02\textwidth}\end{column}
	\end{columns}
	
	\vspace*{-0.3cm}
	
	\begin{columns}
		\begin{column}{0.09\textwidth}\end{column}
		\begin{column}{0.82\textwidth}
			\begin{figure}\justifying									
				\scriptsize{\bluetextbf{Figure 23:} 
				Feet gluing jig: the feet are screwed on the jig. Feet at the $\leftrightarrow$ position are rotated by $90^\circ$. 
				The jig is then laid into the glue dispensing machine.}
			\end{figure}
		\end{column}
		\begin{column}{0.09\textwidth}\end{column}
	\end{columns}
	
	\begin{columns}
		\begin{column}{0.02\textwidth}\end{column}
		\begin{column}{0.46\textwidth}
			\begin{figure}
				\includegraphics[height=0.23\paperheight]{TRACKER-LadFabrication-23.png}
			\end{figure}
		\end{column}
		\begin{column}{0.04\textwidth}\end{column}
		\begin{column}{0.46\textwidth}
			\begin{figure}
				\includegraphics[height=0.23\paperheight]{TRACKER-LadFabrication-24.png}
			\end{figure}
		\end{column}
		\begin{column}{0.02\textwidth}\end{column}
	\end{columns}
	
	\vspace*{-0.3cm}
	
	\begin{columns}[t]
		\begin{column}{0.02\textwidth}\end{column}
		\begin{column}{0.46\textwidth}
			\begin{figure}\justifying
				\scriptsize{\bluetextbf{Figure 24:}
				Base jig: the ladder is aligned with Delrin pins and fixed with vacuum.}
			\end{figure}
		\end{column}
		\begin{column}{0.04\textwidth}\end{column}
		\begin{column}{0.46\textwidth}
			\begin{figure}\justifying									
				\scriptsize{\bluetextbf{Figure 25:} 
				Once the glue is dispensed on the feet, the feet gluing jig is inserted onto the base jig.}
			\end{figure}
		\end{column}
		\begin{column}{0.02\textwidth}\end{column}
	\end{columns}
\end{frame}


\begin{frame}[label=TRK-Fab10,noframenumbering]
	\frametitle{Silicon Tracker}
	\framesubtitle{ladder fabrication  
	\hspace*{5.5cm}
	\hyperlink{TRK-Fab09}{\textbf{\beamerbutton{previous}}}
	\hspace{-0.1cm}
	\hyperlink{TRK-Fab11}{\textbf{\beamerbutton{next}}}
	\textcolor{itemblue}{$\parallel$}
	\hyperlink{TRK-Fab00}{\textbf{\beamerbutton{presentation}}}}
	
	\vspace*{-0.45cm}

	\begin{columns}
		\begin{column}{0.02\textwidth}\end{column}
		\begin{column}{0.46\textwidth}
			\begin{figure}
				\includegraphics[height=0.51\paperheight]{TRACKER-LadFabrication-25.png}
			\end{figure}
		\end{column}
		\begin{column}{0.04\textwidth}\end{column}
		\begin{column}{0.46\textwidth}
			\begin{figure}
				\includegraphics[height=0.51\paperheight]{TRACKER-LadFabrication-26.png}
			\end{figure}
		\end{column}
		\begin{column}{0.02\textwidth}\end{column}
	\end{columns}
	
	\vspace*{-0.3cm}
	
	\begin{columns}[t]
		\begin{column}{0.02\textwidth}\end{column}
		\begin{column}{0.46\textwidth}
			\begin{figure}\justifying
				\scriptsize{\bluetextbf{Figure 26:}
				Airex spacer gluing. Once that the glue (Araldite 2011) is dispensed on the spacer surface,
				in similar way as for the ladder reinforcement, 
				the spacer is aligned with a plexiglas guide.
				A plexiglas lid is then laid on the spacer surface.}
			\end{figure}
		\end{column}
		\begin{column}{0.04\textwidth}\end{column}
		\begin{column}{0.46\textwidth}
			\begin{figure}\justifying									
				\scriptsize{\bluetextbf{Figure 27:} Shielding.
				The doubly-metallized Upilex foil, then the ladder are placed on a table. 
				The foil is bent over the ladder with pushers.
				The shielding is closed with rosin-free solder.}
			\end{figure}
		\end{column}
		\begin{column}{0.02\textwidth}\end{column}
	\end{columns}
\end{frame}

\begin{frame}[label=TRK-Fab11,noframenumbering]
	\frametitle{Silicon Tracker}
	\framesubtitle{ladder fabrication  
	\hspace*{6.25cm}
	\hyperlink{TRK-Fab10}{\textbf{\beamerbutton{previous}}}
	\textcolor{itemblue}{$\parallel$}
	\hyperlink{TRK-Fab00}{\textbf{\beamerbutton{presentation}}}}
	
	\vspace*{-0.45cm}

	\begin{columns}
		\begin{column}{0.02\textwidth}\end{column}
		\begin{column}{0.96\textwidth}
			\begin{figure}
				\includegraphics[height=0.23\paperheight]{TRACKER-LadFabrication-27.png}
			\end{figure}
		\end{column}
		\begin{column}{0.02\textwidth}\end{column}
	\end{columns}
	
	\vspace*{-0.35cm}
	
	\begin{columns}
		\begin{column}{0.09\textwidth}\end{column}
		\begin{column}{0.82\textwidth}
			\begin{figure}\centering									
				\scriptsize{\bluetextbf{Figure 28:} 
				Two ladders ready to be mounted on a support plane.}
			\end{figure}
		\end{column}
		\begin{column}{0.09\textwidth}\end{column}
	\end{columns}
	
	\begin{columns}
		\begin{column}{0.02\textwidth}\end{column}
		\begin{column}{0.96\textwidth}
			\begin{figure}
				\includegraphics[height=0.34\paperheight]{TRACKER-LadFabrication-28.png}
				\hspace*{0.5cm}			
				\includegraphics[height=0.34\paperheight]{TRACKER-LadFabrication-29.png}
			\end{figure}
		\end{column}
		\begin{column}{0.02\textwidth}\end{column}
	\end{columns}
	
	\vspace*{-0.35cm}
	
	\begin{columns}
		\begin{column}{0.09\textwidth}\end{column}
		\begin{column}{0.82\textwidth}
			\begin{figure}\centering									
				\scriptsize{\bluetextbf{Figures 29 and 30:} 
				Tracker layer 2 equipped with its 24 ladders.}
			\end{figure}
		\end{column}
		\begin{column}{0.09\textwidth}\end{column}
	\end{columns}
\end{frame}

\end{document}
